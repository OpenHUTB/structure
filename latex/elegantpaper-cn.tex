%!TEX program = xelatex
% !BIB program = biber
% 完整编译: xelatex -> biber/bibtex -> xelatex -> xelatex
\documentclass[lang=cn,a4paper,newtx]{elegantpaper}

% 参考:https://weibo.com/ttarticle/p/show?id=2309404908550236537030
\title{大脑功能的几何约束}
%\author{作者1 \\ 某某大学/机构 \and 作者2 \\ 某某大学/机构}
%\institute{\href{https://elegantlatex.org/}{Elegant\LaTeX{} 项目组}}

%\version{0.11}
\date{}

% 本文档命令
\usepackage{array}
\newcommand{\ccr}[1]{\makecell{{\color{#1}\rule{1cm}{1cm}}}}
\addbibresource[location=local]{reference.bib} % 参考文献,不要删除

\begin{document}

\maketitle

\begin{abstract}
脑部的解剖结构必然限制其功能,但具体的限制方式尚不清楚。在神经科学中,传统且占主导地位的范式是,神经元动态是由复杂的轴突纤维网络连接的离散、功能专一的细胞群体之间的交互作用驱动的。然而,从神经场理论(一个用于模拟大脑大规模活动的既定数学框架)的预测来看,大脑的几何形状可能比复杂的区域间连接性更基本地限制了动态性。在这里,我们通过分析在自发条件和多种任务诱发条件下获取的人类磁共振成像数据来确认这些理论预测。具体来说,我们显示,皮质和皮质下活动可以被简明地理解为大脑几何形状(即其形状)的基本共振模式的激发,而不是像传统观念中所假设的复杂区域间连接性的模式。然后,我们使用这些几何模式来显示,在超过10,000个大脑绘图中,任务诱发的激活并不局限于焦点区域(如广泛认为的那样),而是激发了波长超过60毫米的全脑模式。最后,我们证实了几何形状和功能之间紧密关联是由波动活动的主导作用解释的预测,显示波动动力学可以再现自发和诱发记录的许多典型的空间时间属性。我们的发现挑战了现有的观点,并确定了一个以前被低估的几何形状在塑造功能中的作用,正如大脑全局动态的统一和物理原理模型所预测的那样。
%\keywords{Elegant\LaTeX{},工作论文,模板}
\end{abstract}

\section{主题}

许多自然系统的动态特性在根本上受到其潜在结构的限制。例如,鼓的形状影响其声学特性,河床的形态塑造了水下的流动,蛋白质的几何形状决定了它可以与哪些分子进行相互作用。神经系统也不例外,由轴突互连网支持的解剖分布的神经元群体的丰富和复杂的时空动态特性就是这样。有几个研究已经显示出大脑连接性和活动的各种特性之间的相关性,但神经动态的时空模式如何受到相对稳定的神经解剖支架的限制,这一点尚不清楚。

在物理和工程的多个领域中,系统动态的结构约束可以通过系统的特征模式来理解,这些特征模式是对应系统自然共振模式的基本空间模式。在线性阶段,如大脑在正常(即,非癫痫样)条件下的活动,特征模式(以下也称为模式)为连接大脑解剖结构和塑造活动的物理过程提供了一种特别强大和严格的形式主义。通过这个视角,神经动态的时空模式源自大脑结构特征模式的激发,就像拨动的小提琴弦的和音源自其自身共振模式的振动。

关键的是,就像小提琴弦的共振频率由其长度、密度和张力决定一样,大脑的特征模式由其结构——物理、几何和解剖——属性决定。这些特定的结构属性中是否有任何一个对动态产生主导贡献?在这里,我们测试了两种具有影响力且竞争的理论,这两种理论对大脑结构的哪些关键元素塑造功能做出了不同的预测。

一个传统的观点,代表了神经科学中的主导范式,它源于Ramon y Cajal的神经元学说,Brodmann的细胞结构学以及一个世纪以来将功能定位到特定大脑区域的工作。根据这种观点,神经动态的时空模式源于离散、功能专一的细胞群体之间的交互,这些细胞群体由拓扑复杂的短程和长程轴突连接组成。在人类中,这些连接可以通过扩散磁共振成像(dMRI)在宏观尺度上进行估计,以得到一个称为连接组的结构连接矩阵。这种方法已被广泛用于理解大脑组织和动态,最近的工作提出,从这样的离散连接组中得到的特征模式——在这里被称为连接组特征模式——可以用来重构人类皮层的典型功能网络的空间模式,这些网络是通过功能性磁共振成像(fMRI)映射的。

这种基于离散连接组的观点的一个限制是,它依赖于一个不直接考虑大脑解剖学的物理属性和空间嵌入(即,几何和拓扑)的抽象表示。这些特征被明确地纳入到一大类神经场理论(NFTs)中,这些理论描述了在0.5毫米以上的空间尺度上的平均场神经动态(补充信息1)。一种受生理限制的NFT形式已经统一了各种不同的实证现象,它将皮层活动视为通过物理连续的神经组织片层传播的行波的叠加。在这个理论中,不同皮质位置之间的神经交互被近似为一个随距离大致指数衰减的均匀空间核,这与实验证据一致,证据显示,无数种物种的神经系统的组织都受到连接性的指数距离规则(EDR)的普遍管理。

鉴于波动动态和EDR样的连接性,NFT(神经场理论)的一个关键预测是,大脑的内在几何形状物理上塑造并对出现的动态施加边界条件。这个观点的一个显著的推论是,如果我们优先考虑大脑解剖学的空间和物理限制,我们只需要考虑大脑的形状,而不是它的全系列拓扑复杂的轴突互连性,就可以理解空间模式化的活动。更正式地说,这个理论预测,从大脑几何形状中得到的特征模式——在这里被称为几何特征模式——比连接组更基本的解剖学约束动态。这个观点与传统观点形成鲜明对比,传统观点认为区域间解剖连接的复杂模式塑造大脑活动。

在这里,我们测试了这些关于大脑的竞争观点,目的是确定人脑动态的主要结构约束。与来自NFT(神经场理论)的理论预测一致,我们展示了从人类新皮层的自发和任务诱发记录中获取的多样化的实验性fMRI数据,可以更简洁地由从皮层几何形状(几何特征模式)得出的特征模式解释,而不是由大脑连接性(连接组特征模式)的测量得出的那些。我们进一步确认,刺激诱发的活动主要由具有长空间波长的几何特征模式的激发主导,这挑战了传统观点,即这种活动局限于局部、空间隔离的簇。为了直接将这些结构约束与驱动大脑动态的物理过程联系起来,我们使用了一个生成模型,展示了如何在皮层的几何形状上展开的波动动态可以解释功能性大脑组织的多样性特征。最后,我们展示了通过特征模式捕获的几何与功能之间的紧密关系扩展到非新皮层结构,表明这种链接是大脑组织的普遍属性。

\section{几何模型约束皮层激活}
我们首先检查几何特征模式能在多大程度上解释人类新皮层活动的多样性。
为了推导出这些特征模式,我们使用一个群体平均模板的新皮层表面网格表示(图\ref{fig:1}a和方法\ref{sec:derivation}中的皮层几何特征模式的推导)。
然后,我们从这个表面网格构建拉普拉斯-贝尔特拉米算子(Laplace–
Beltrami operator, LBO),该算子捕捉了局部顶点到顶点的空间关系和曲率,并解决特征值问题\footnote{在微分几何中,拉普拉斯算子可以推广为定义在曲面,或更一般地黎曼流形与伪黎曼流形上,函数的算子。这个更一般的算子叫做拉普拉斯-贝尔特拉米算子},

% 有*号就没公式标号
\begin{equation} \label{eq:1}
	\nabla^2 \psi = \Delta\psi = -\lambda \psi,
\end{equation}


% \nabla (那勃勒)是哈密顿算子(在量子力学中为一个可观测量,对应于系统的总能量);矢量微分算子;
% \psi是振幅
其中 $ \nabla $ 是梯度算子,$ \Delta $ 是拉普拉斯-贝尔特拉米算子,$ \psi = \{\psi_1(r), \psi_2(r),...\} $ 是一组几何特征模式,对应的特征值为 $ \lambda = \{ \lambda_1, \lambda_2, ... \} $。
特征值按照每个模式的空间频率或波长的顺序进行排序(图\ref{fig:1}a和扩展数据图\ref{fig:extended_fig_1}),因此$ \psi_1 $是波长最长的模式。
得到的特征模式是正交的,形成一个完整的基础集,可以将在皮层上展开的时空动态分解为具有不同波长的模式的加权和(图\ref{fig:1}b和方法\ref{sec:modal_decomposition}中的大脑活动的模式分解)。
除非另有说明,我们在本研究中使用$ N = 200 $个模式\footnote{"模式" (modes) 是一个技术术语,它是指振动系统的特殊状态,这些状态可以独立地进行简谐振动。在这里,它们是指从大脑结构特性(如几何形状)中推导出的特定空间模式,这些模式可能对大脑活动产生影响。}。)


\begin{figure}[!htb]
	\centering
	\includegraphics[width=0.9\textwidth]{fig/fig_1.pdf}
	\caption{\textbf{使用几何特征模式重建新皮层活动}。
	\textbf{a}, 通过解决特征值问题,$ \Delta \psi = -\lambda \psi $(公式(\ref{eq:1})),从皮层表面网格得到几何特征模式。
	模式$ \psi_1, \psi_2, \psi_3, ... \psi_N $ 按空间频率从低到高(空间波长从长到短)排序。
	负值、零和正值分别用蓝色、白色和红色标记。
	\textbf{b}, 大脑活动数据的模态分解。
	示例展示了如何将给定时间$ t $的空间图 $ y(r,t) $ 分解为加权的模式 $ \psi_j $ 之和。
	\textbf{c}, 左图,我们使用一系列刺激对比的激活空间图重建任务诱发的数据。
	右图,我们通过在每个时间帧分解空间图并生成区域到区域的功能耦合(,functional coupling)矩阵来重建自发活动。
	\textbf{d}, 重建七个关键HCP任务对比图(补充信息2.1)和静息状态功能耦合作为模式数量函数的准确性。
	插图显示皮层表面重建,展示了前10个、100个和200个模式所对应的空间尺度,分别对应大约120毫米、40毫米和30毫米的空间波长。
	 \textbf{e}, 使用10个、100个和200个模式得到的7个关键HCP任务对比的群体平均实验任务激活图和重建图(recon)。
	 黑色箭头指示出当使用短波长模式时能更准确地重建的局部激活模式。 
	 \textbf{f}, 使用10个、100个和200个模式的群体平均实验静息状态功能耦合矩阵和重建。
	} \label{fig:1}
\end{figure}


利用这种分解,我们评估了几何特征模式在捕获任务诱导和自发大脑活动(图\ref{fig:1}c)方面的准确性,这些活动是从人类连接体项目(Human Connectome Project, HCP; 方法和补充信息\ref{sec:sup_2}中的HCP数据)中的255个健康个体中测量出来的。
对于任务诱导的活动,我们绘制了从七种不同任务中获取的47种基于任务的对比,这些任务代表了不同的诱导活动模式。
然后,我们使用逐渐增加的模式数量(最多200个)来重建每个个体的激活图(图\ref{fig:1}d)。
对于自发的、无任务的(所谓的静息态)活动,我们在每个时间帧重建活动的空间图,然后生成一个区域到区域的功能耦合矩阵,描述180个每半球离散大脑区域之间的活动相关性。
为了使任务诱导和自发记录之间能够直接进行比较,我们将相同的区域划分应用于任务诱导数据(方法中的皮层划分)。
最后,我们通过计算实证和重建任务诱导激活图以及自发功能耦合矩阵之间的相关性来量化重建的准确性(图\ref{fig:1}d-f)。


我们观察到,在所有任务对比和静息状态下,随着模式数量的增加,重构的准确性也随之增加,仅使用$ N=10 $模式,相关性系数$ r $就已经达到或超过0.38(图\ref{fig:1}d)。
不同的任务也会调动不同的大尺度模式,这表明特定的刺激会激发特定的模式(图\ref{fig:1}e)。
在10个模式之后,重构准确性的改善变得缓慢,大约在$ N=100 $模式时达到$ r \geq 0.80 $,之后的重构准确性只有微小的增加。
因为前100个模式的波长都超过40毫米(补充表1),而包含更短波长的模式只会细化局部模式的重构(图\ref{fig:1}e中的箭头),我们的发现表明,数据主要由具有长空间波长的空间模式组成\footnote{在这里,波长是指空间模式或模式的尺度。在这篇文章中,研究者们发现,脑活动的重构主要依赖于具有较长空间波长(大约40毫米或更大)的模式。这个发现表明,大脑的功能模式具有较大的空间尺度,这与以往对于脑活动主要在局部区域发生的观点形成了鲜明的对比。}(下一节将进行更详细的分析)。


这些结果在所有47个HCP任务对比(补充图\ref{fig:supp_1})和各种分辨率的划分(补充图\ref{fig:supp_2})中都是一致的,但是高分辨率划分的数据需要更多的模式才能达到高重建精度,这是由于粗略划分的低通空间滤波效应。
我们的结果也不受使用群体平均皮层表面模板(而不是个体特定的表面)来推导几何特征模式的影响(补充图\ref{fig:supp_3}-\ref{fig:supp_5}和补充信息\ref{sec:individual_specific})。
总的来说,这些发现表明皮层几何特征模式形成了一个紧凑的表示,捕获了任务引发和自发皮层活动的各种方面。
此外,他们显示出这样的活动主要由长波长、大规模的特征模式主导。


我们接下来测试几何特征模式是否比基于图论的连接体近似导出的特征模式提供了更简洁且基础的动态描述。
为此,我们将几何特征模式的重构精度与三个替代的连接体导出的特征模式基准集进行比较(参见图\ref{fig:2}a的示意图)。
第一个基准集是根据dMRI追踪图在顶点分辨率上映射的连接体并经过阈值处理得到的,以获得0.10\%的连接密度,就像我们之前做过的那样(在方法中导出连接体的特征模式)。
第二个基准集是根据均匀的随机接线过程构建的,该过程受到指数级的距离依赖连接概率的控制,以模仿简单的,类似指数距离规则的连接性(在方法\ref{sec:derivation}皮层几何特征模式推导中)。
因为实证连接体和指数距离规则连接体的连接密度不同,我们评估了一个第三个基准集,该基准集来自经过1.55\%的阈值处理的实证连接体,以匹配指数距离规则连接体的密度。
上述连接体,指数距离规则和密度匹配的连接体特征模式是从其各自的连通性矩阵的图拉普拉斯算子(拉普拉斯-贝尔特拉米算子的离散对应物)导出的(见图\ref{fig:2}b和扩展数据图\ref{fig:extended_fig_1})。


\begin{figure}[!htb]
	\centering
	\includegraphics[width=0.9\textwidth]{fig/fig_2.pdf}
	\caption{\textbf{几何特征模态与基于连接组的特征模态的对比评估}。
		\textbf{a}, 用于推导皮层几何,连接体和指数距离规则(exponential distance rule, EDR)连接体的特征模式的解剖属性示意图。
		几何特征模式依赖于局部表面网格信息,如相邻表面网格顶点(点)之间的连接(蓝色)和曲率。
		连接体特征模式依赖于网格顶点(蓝色)之间的局部链接,以及从dMRI实证重构的短程和长程连接(品红色)。
		指数距离规则特征模式依赖于从随机接线过程生成的连接(红色),其中顶点之间的连接概率作为它们距离的函数呈指数衰减。
		\textbf{b},连接体和指数距离规则特征模式的示例。
		负值,零值和正值分别以蓝色,白色和红色着色。
		尽管存在一些相似之处,这些模式的空间模式与使用皮层几何学导出的模式不同(与图\ref{fig:1}a比较)。
		\textbf{c},由几何,EDR和两种变体的连接体特征模式实现的静息态功能耦合矩阵的重构精度:
		一种使用以前的方法\cite{naze2021robustness}定义的连接体,另一种与指数距离规则连接体具有相同的连接密度,以便公平比较(对于其他密度,请参见补充图\ref{fig:supp_6}和\ref{fig:supp_7})。
		\textbf{d},由几何特征模式和其他基准集实现的所有47个HCP任务对比图的重构精度之间的差异,如每个面板上方的文本所示。
		每一行代表一种不同的任务对比,这里按照广泛的类型进行分组(补充信息2.1);
		红色表示几何特征模式的性能优越。
		请注意,虽然在与几何特征模式相比的重构中,对于包含少于十个模式的连接体特征模式似乎有性能优势,但重构精度通常较低(不同任务的平均$ r = 0.42 $),与100个模式的情况(平均$ r = 0.71 $)相比。
	} \label{fig:2}
\end{figure}


总的来说,几何特征模式考虑了皮层表面的内在曲率和表面网格中的局部顶点到顶点的关系;
连接体特征模式并未考虑曲率,但捕捉了网格顶点之间的局部空间关系,以及使用dMRI测量的短程和长程连接;
而指数距离规则特征模式考虑了均匀的,随机的,距离依赖的连接规则的影响,但并未完全捕捉到皮层几何(图\ref{fig:2}a)。
因此,比较这些不同的基准集使我们能够区分皮层几何与结构连接对大脑动态的贡献。


直接比较这些不同基准集的重构精度显示,几何特征模式在自发(图\ref{fig:2}c)和任务诱发(图\ref{fig:2}d)数据中始终显示出最高的重构精度。
指数距离规则特征模式的表现几乎与几何特征模式一样好,而连接体特征模式的精度最低。
无论使用哪种划分(扩展数据图2和3),用于生成连接体特征模式的特定连接密度(补充图6和7以及补充信息4),以及我们是否使用离散的区域划分而不是顶点分辨率生成连接体(补充图8和补充信息4),这一发现都成立。我们还发现,几何特征模式显示出比功能数据本身的主成分(通过主成分分析(PCA)计算;补充图9,扩展数据图4和补充信息5)更强的样本外泛化能力,并且比傅立叶空间基准集的表现更好(扩展数据图5,补充信息6和方法中的统计基准集的比较)。      总的来说,这些结果展示了几何特征模式作为大脑功能基准集的简洁性,稳健性和普遍性。它们还支持NFT的预测,即大脑活动最好用直接从皮质形状导出的特征模式来表示,从而强调了几何形状在约束动态中的基本作用。
长波长主导皮层活动      使用几何特征模式对自发和任务诱发的数据进行重构显示,大脑活动的空间组织主要由空间波长约40毫米或更长的模式主导(图1d-f)。这个结果反驳了神经影像数据分析的经典方法,其中通过阈值化统计图来映射刺激诱发的激活,以识别高度活动的局部、孤立的区域。这种经典方法基于这样的假设,即焦点区域代表了刺激可能引发的离散的大脑区域,而其他区域的次阈值活动是可以忽略的。任务激活数据的令人惊讶的长波长内容(图1d-e)表明,经典程序只关注冰山的尖端,并掩盖了任务诱发的底层空间延展和结构化的活动模式(参见扩展数据图6以获取涉及原因的解释)。这些观察符合NFT的理论预测以及先前对任务诱发的电脑图形 (EEG) 信号的分析。(编者注:"冰山"在这里是个比喻,意指经典神经影像数据分析方法只揭示了大脑活动的一小部分(即"冰山的尖端"),而忽略了由任务诱发的底层空间延展和结构化的活动模式,这些可能类似于隐藏在水面下的冰山的大部分。)      为此,我们分析了使用HCP中的47个任务对比中的群体平均未阈值化激活图的几何模式分解获得的空间功率谱(方法中的任务诱发激活图的模式功率谱)。作为独立的复制,我们还分析了NeuroVault存储库中1,178个独立实验的10,000个未阈值化激活图,从而提供了在人脑中映射的刺激诱发激活模式的多样性的全面绘图。      尽管获取这些激活图使用了广泛的刺激、范例和数据处理方法,但我们观察到图中的大部分功率集中在前50个模式中,这些模式对应的空间波长大于约60毫米(图3a;对HCP关键任务对比图的每个单独结果也类似;扩展数据图7)。使用替代数据,我们确认这些发现不能由典型的fMRI处理流程引发的空间平滑解释,该流程可以过滤出活动的短波长空间模式(扩展数据图8和补充信息7)。我们进一步观察到,递增地、顺序地移除长波长模式对重构精度的影响比移除短波长模式的影响要大得多(图3b和方法中的长和短波长模式的贡献)。例如,在七个关键的HCP任务对比中,移除前25\%的长波长模式(模式1-50)导致重构精度下降约40-60\%,而移除前25\%的短波长模式(模式151-200)仅导致下降约2-4\%(图3b,插图)。这些结果表明,在fMRI可接触的时间和空间尺度上,诱发的皮层活动包括大尺度,几乎是大脑全宽的空间模式,挑战了这样的经典观点,即这种活动应该以离散的、孤立的和解剖学上定位的激活簇来描述。




\section{方法} \label{sec:method}

\subsection{皮层几何特征模式推导} \label{sec:derivation}

% Each eigenmode
每个特征模式包括具有特定空间波长的空间图案。
根据参考文献\cite{robinson2016eigenmodes},我们使用理想化的球形情况来近似特征模式波长,因为它在拓扑上与人类皮层相当。
通过在球体上求解方程\ref{eq:1},存在退化解,使得某些特征模式具有相同的特征值和空间波长——这类似于量子物理学中的球谐函数。
事实上,由于特征模式将在皮层折叠消失的极限下接近球谐函数,因此前者可以组合成具有空间波长的本征群,

\begin{equation}
	\text{wavelength} = \frac{2 \pi R_s}{\sqrt{l(l+1)}},
\end{equation}

其中$R_s$是球体的半径(对于本研究中使用的fsaverage总体平均模板,$R_s \approx 67.0 mm$),$l$是本征群数(原子物理学中的角动量量子数)。
前15个本征群的波长和本征群中包括的本征模如补充表1所示。


\subsection{大脑活动的模式分解} \label{sec:modal_decomposition}





\section{扩展数据的图}


\begin{figure}[!htb] 
	\centering
	\includegraphics[width=0.9\textwidth]{fig/extended_fig_1.pdf}
	\caption{\textbf{特征模式基组}。
		从左到右的基组是几何特征模式、连接组特征模式、使用与指数距离规则特征模式使用的密度相匹配的连接矩阵的连接组特征模式,和距离指数规则特征模式。 
		负值-零值-正值被着色为蓝-白-红。
	} \label{fig:extended_fig_1}
\end{figure}


\begin{figure}[!htb] 
	\centering
	\includegraphics[width=0.9\textwidth]{fig/extended_fig_2.pdf}
	\caption{对于不同的分割分辨率,通过不同的基集实现静息状态功能耦合的重建精度。 
		基组是几何特征模式、连接组特征模式、使用密度与指数距离规则特征模式使用的连接矩阵相匹配的连接组特征模式和指数距离特征模式。
		Schaefer100、Schaefer200、Glasser360、Schaefer400、Schaefer600、Schaefer800 和 Schaefer1000 分别在两个半球拥有 100、200、360、400、600、800 和 1000 个块。
	} \label{fig:extended_fig_2}
\end{figure}


\begin{figure}[!htb] 
	\centering
	\includegraphics[width=0.9\textwidth]{fig/extended_fig_3.pdf}
	\caption{\textbf{在不同分割分辨率下,通过几何特征模式和连接体特征模式实现的所有47个HCP任务对比图的重建精度差异}。
	每行代表不同的任务对比,此处按广泛类型进行分组(补充信息 2.1)。
	wm = 工作记忆。
	红色表示几何特征模式的卓越性能。
	Schaefer100、Schaefer200、Glasser360、Schaefer400、Schaefer600、Schaefer800 和 Schaefer1000 分别在两个半球拥有 100、200、360、400、600、800 和 1000 个块。}
	\label{fig:extended_fig_3}
\end{figure}


\begin{figure}[!htb] 
	\centering
	\includegraphics[width=0.9\textwidth]{fig/extended_fig_4.pdf}
	\caption{\textbf{几何特征模式和主成分分析实现的重建精度}。
	(\textbf{a}) 静息态功能耦合的重建精度。
	(\textbf{b}) 比较所有 47 个 HCP 任务对比图的重建精度,这些图已按广泛类型进行了分组(补充信息 2.1)。
	wm = 工作记忆。
	每行代表不同的任务对比。
	红色表示几何特征模式的卓越性能。
	星号表示用于训练主成分分析相关任务中的对比(即七个关键 HCP 任务对比)。}
	\label{fig:extended_fig_4}
\end{figure}


\begin{figure}[!htb] 
	\centering
	\includegraphics[width=0.9\textwidth]{fig/extended_fig_5.pdf}
	\caption{\textbf{几何特征模式与傅立叶基集的比较}。
	\textbf{a},具有单位系数的六个不同傅里叶基组的模式1、2、3、4、10、50和100的空间图。
	术语 reg 和 irreg 表示 $ x $、$ y $ 和 $ z $ 方向上的模式的空间波长分别以规则和不规则增量间隔开。
	详情请参阅补充信息 6。
	\textbf{b},七个关键 HCP 任务对比图和静息状态功能耦合的重建准确性。
	有关对比图的详细信息,请参阅补充信息 2.1。
	wm = 工作记忆。}
	\label{fig:extended_fig_5}
\end{figure}


\begin{figure}[!htb] 
	\centering
	\includegraphics[width=0.9\textwidth]{fig/extended_fig_6.pdf}
	\caption{\textbf{阈值统计图的经典神经成像方法}。
	\textbf{a},简单的一维示例,说明不同阈值如何仅捕获激活的焦点簇并忽略激活的底层结构化模式。
	\textbf{b},使用未阈值和二值化阈值地图描述的概念的空间嵌入式演示。}
	\label{fig:extended_fig_6}
\end{figure}


\begin{figure}[!htb] 
	\centering
	\includegraphics[width=0.9\textwidth]{fig/extended_fig_7.pdf}
	\caption{\textbf{七个关键HCP任务对比图中每一个的归一化功率谱}。
	有关对比图的详细信息,请参阅补充信息 2.1。
	wm = 工作记忆。}
	\label{fig:extended_fig_7}
\end{figure}


\begin{figure}[!htb] 
	\centering
	\includegraphics[width=0.9\textwidth]{fig/extended_fig_8.pdf}
	\caption{\textbf{经验任务激活映射和代理映射的功率谱}。
	\textbf{a},47 个 HCP 任务对比图(上)和来自 NeuroVault 数据库的 10,000 个对比图(下)的归一化平均功率谱。
	彩色线对应于应用具有不同半高全宽 (full-width at
	half-maximum, FWHM) 的空间平滑滤波器后的替代数据的功率谱。
	\textbf{b},平均均方对数误差 (MSLE) 作为 HCP 和 NeuroVault 对比图的归一化平均功率谱与平滑替代数据之间的 FWHM 的函数。
	\textbf{c},分别在 47 个 HCP 和 10,000 个 NeuroVault 对比图和平滑替代数据之间的功率谱之间获得 MSLE。
	每行代表不同的任务对比图。
	这些线对应于每张图的 MSLE 最小的 FWHM。}
	\label{fig:extended_fig_8}
\end{figure}


\begin{figure}[!htb] 
	\centering
	\includegraphics[width=0.9\textwidth]{fig/extended_fig_9.pdf}
	\caption{\textbf{波动力学模型和神经质量模型在捕捉功能磁共振成像数据时滞特性方面的比较}。
	\textbf{a},来自实证数据、使用波模型的模拟数据和使用左半球神经质量模型的模拟数据的时滞矩阵。
	负-零-正值被着色为蓝-白-红。
	\textbf{b},投影到皮质表面的 a 矩阵(每列的平均值)的平均滞后。
	负-零-正值被着色为蓝-白-红。
	散点图显示了来自 180 个大脑区域的两个模型的经验数据和模拟数据的平均滞后的关系。
	红线表示与皮尔逊相关系数 $ r $ 和单边旋转测试 $ p $ 值 pspin 的线性拟合,根据 10,000 个排列进行估计。 
	\textbf{c},与 b 类似,但基于 a 中矩阵的第一个主成分 (PC1)。
	表面上方的数字 (var) 对应于 PC 解释的方差。
	\textbf{d},与 c 类似,但位于 a 中矩阵的第二个 PC (PC2) 上。}
	\label{fig:extended_fig_9}
\end{figure}



\section{补充信息}

\subsection{人类连接体项目数据} \label{sec:sup_2}

\subsubsection{任务诱发的数据}

我们分析了在 7 个任务域中测量的任务诱发功能磁共振成像数据,这些数据已被证明可以可靠地组成各种神经系统\cite{barch2013function}。
这 7 个任务是:社交、运动、赌博、工作记忆、语言、情感和关系。
补充表 2 显示了每个任务领域涉及的具体对比以及本研究中调查的关键对比。
我们总共分析了 47 个对比,其中包括 7 个关键对比。
关键对比代表了文献中常用的那些来映射任务引发的主要激活模式。
有关每项任务和对比的详细信息,请参阅\cite{barch2013function}。
该分析是在通过\href{https://fsl.fmrib.ox.ac.uk/}{FSL} 交叉运行(2 级)FEAT 分析\cite{woolrich2004multilevel}计算的各个任务激活图上进行的。
我们使用 HCP 提供的任务图,通过多模态表面匹配\cite{robinson2018multimodal},将最小平滑 (2 mm) 映射到 fsLR-32k CIFTI 空间,每个半球有 32,492 个顶点。


\subsubsection{无任务静息态数据}
我们使用从左到右 (LR) 编码方向分析了在一次扫描过程中获得的无任务静息态 fMRI。
扫描持续 14.4 分钟,总共 1200 个时间帧。
简而言之,静息态 fMRI 采集参数为:各向同性体素大小为 2 mm,重复时间 (TR) 为 720 ms,回波时间 (TE) 为 33.1 ms。
所有其他采集参数都可以在\cite{van2013wu}中找到。
每个人的数据均由 HCP 通过其最小预处理管道进行预处理\cite{glasser2013minimal},并且还经过 ICA-FIX 来纠正结构噪声和残余混杂\cite{salimi2014automatic}。
没有执行额外的平滑。
与任务诱发数据类似,静息态数据被映射到 fsLR-32k CIFTI 空间。
因此,每个人的数据被表示为每个半球上大小为 32,492 个顶点 × 1200 个时间帧的矩阵。


\subsubsection{连接体数据}

为了导出连接组本征模式,我们使用了通过概率纤维束成像从扩散 MRI (dMRI) 数据导出的单个连接组,如\cite{tian2021high}中提供的。
简而言之,dMRI 采集参数为:各向同性体素大小为 1.25 mm,TR 为 5520 ms,TE 为 89.5 ms,b 权重为 1000、2000、3000 s/mm2 和 6 次 b0 扫描。
所有其他采集参数均可在\cite{van2013wu}中找到。
每个人的数据均由 HCP 通过其扩散预处理流程 (v3.19.0) \cite{glasser2013minimal}进行预处理。
为了生成连接组,使用 MRtrix 和概率纤维束描记术生成纤维束图,500 万条流线连接解剖学上不同的流线 脑区域、多壳多组织 (MSMT) 约束球形反卷积 (CSD) 解剖约束纤维束成像 (ACT) 和纤维方向分布算法的二阶积分 (iFOD2)。
详细信息请参见\cite{tian2021high}。
标准 MNI 空间中的 fsLR-32k 皮质表面网格用于定义灰质-白质界面。
对于每个个体,每个半球生成的流线被映射到表面网格上最近的顶点,以构建高分辨率加权连接组(32,492 × 32,492 矩阵大小和代表流线总数的权重)。
详细信息请参见\cite{tian2021high}。


\subsection{个体特异性皮层特征模式} \label{sec:individual_specific}
亥姆霍兹方程\ref{eq:1}可用于求解任何皮质表面网格模型的特征模式,但网格几何形状的变化可能会改变生成的特征值 $ \lambda $ 和特征模式本征模态 $ \psi $。
因此,从各个对象表面导出的特征值和特征模式,特别是在非常短的波长下,将有所不同,并且不能直接比较\cite{henderson2022empirical,chen2022individuality}。
为了简单起见,这里我们使用从公共模板表面生成的特征模式(参见方法\ref{sec:derivation}中的“皮层几何本征模式的推导”)。
虽然这允许比较不同个体的数据重建,但它掩盖了与皮层几何个体差异相关的任何潜在影响。
然而,补充图 \ref{fig:supp_3} 显示,使用从各个皮层表面导出的几何特征模式不会改变研究的总体结果,这表明,就目前的目的而言,从总体平均模板表面导出的特征模式代表了基本几何特征模态的良好近似。
尽管如此,为了强调个体几何结构的某些细微差别,我们在补充图\ref{fig:supp_4} 中表明,在某些个体中,200 个个体特定的特征模式的表现略好于模板导出的本征模式,特别是在重建任务激活图方面,但在重建静息状态活动方面则不然。 。 
然而,补充图\ref{fig:supp_5} 显示,个体特定和模板衍生的特征模式的结果最终在非常短的波长(~第500个模式)处收敛。
因此,前 200 种模式捕获了皮层几何结构的个体差异对大脑功能的影响,这与最近的工作\cite{chen2022individuality}一致,并且对应于最大可能的特征模式数量的一小部分(~0.6\%)。


\subsection{补充图}

\begin{figure}[!htb] 
	\centering
	\includegraphics[width=0.7\textwidth]{fig/supp_1.pdf}
	\caption{\textbf{使用几何特征模式获得的 47 个 HCP 任务对比图的重建精度}。
		线条根据 7 种广泛的 HCP 任务类型定义的组进行着色(第 S2.1 节和补充表 2)。 wm = 工作记忆。} \label{fig:supp_1}
\end{figure}



\begin{figure}[!htb] 
	\centering
	\includegraphics[width=0.85\textwidth]{fig/supp_2.pdf}
	\caption{\textbf{不同分割分辨率下 7 个关键 HCP 任务对比图和静息态功能耦合的重建精度}。 
		有关对比图的详细信息,请参阅第 S2.1 节和补充表 2。
		wm = 工作记忆。
		从深色到浅色的线条代表递减的分割分辨率(箭头方向);
		即 Schaefer100、Schaefer200、Glasser360、Schaefer400、Schaefer600、Schaefer800 和 Schaefer1000 在两个半球分别有 100、200、360、400、600、800 和 1000 个地块。} \label{fig:supp_2}
\end{figure}



\begin{figure}[!htb] 
	\centering
	\includegraphics[width=0.85\textwidth]{fig/supp_3.pdf}
	\caption{\textbf{使用模板导出和个体特定的几何特征模式重建 7 个关键 HCP 任务对比图和静息态功能耦合的准确性}。
		有关对比图的详细信息,请参阅第 S2.1 节和补充表 2。 
		wm = 工作记忆。
		实线代表由模板表面导出的特征模式所获得的结果(图\ref{fig:supp_1}d)。
		虚线表示通过从各个表面导出的各个特定特征模式所实现的结果。
		插图显示两个结果之间的差异(即模板减去个体特异性)。} \label{fig:supp_3}
\end{figure}




\begin{figure}[!htb] 
	\centering
	\includegraphics[width=0.85\textwidth]{fig/supp_4.pdf}
	\caption{\textbf{使用 200 个模板导出的和个体特定的几何特征模式,对 7 个关键 HCP 任务对比图和静息态耦合进行基于个体的重建精度}。
		有关对比图的详细信息,请参阅第 S2.1 节和补充表 2。
		wm = 工作记忆。
		模板特征模式自模板表面,而个体特定特征模式源自个体表面。
		每个点对应一个个体。
		虚线代表模板准确度 = 个体特定准确度线。
		虚线上方的点表示模板准确度 > 个体特定准确度。
		插图显示了模板导出的特征模式和个体特定的特征模式(即模板减去个体特定的)所实现的重建精度之间差异的直方图。} \label{fig:supp_4}
\end{figure}


\begin{figure}[!htb] 
	\centering
	\includegraphics[width=0.85\textwidth]{fig/supp_5.pdf}
	\caption{\textbf{使用 200 至 500 个模板导出的和个体特定的几何特征模式重建 7 个关键 HCP 任务对比图和静息态功能耦合的准确性}。
		有关对比图的详细信息,请参阅第 S2.1 节和补充表 2。
		wm = 工作记忆。
		实线代表由模板表面导出的本征模所获得的结果(图\ref{fig:supp_1}d)。
		虚线表示通过从各个表面导出的各个特定本征模式所实现的结果。
		插图显示了模板导出的特征模式和个体特定的本征模式(即模板减去个体特定的)所实现的重建精度之间的差异。} \label{fig:supp_5}
\end{figure}



\begin{figure}[!htb] 
	\centering
	\includegraphics[width=0.85\textwidth]{fig/supp_6.pdf}
	\caption{
		\textbf{通过几何特征模式和不同连接组密度的连接组特征模式实现的 7 个关键 HCP 任务对比图和静息态功能耦合的重建精度}。
		有关对比图的详细信息,请参阅第 S2.1 节和补充表 2。
		wm = 工作记忆。
		虚线表示几何特征模式取得的结果(图\ref{fig:supp_1}d)。
		浅色到深色的线条代表使用密度不断增加的连接矩阵(从 1.0\% 到 13.0\%)的连接组本征模式所实现的结果。
	} \label{fig:supp_6}
\end{figure}


\begin{figure}[!htb] 
	\centering
	\includegraphics[width=0.85\textwidth]{fig/supp_7.pdf}
	\caption{
		\textbf{使用不同连接组密度的 200 个连接组本征模式重建 7 个关键 HCP 任务对比图和静息态功能耦合的准确性}。
		有关对比图的详细信息,请参阅第 S2.1 节和补充表 2。
		wm = 工作记忆。
		实线对应于200个几何特征模式所达到的重建精度。
		虚线分别对应于连接组密度 0.1\% 和 1.55\%,用于生成扩展数据图\ref{fig:supp_1}中的连接组和密度匹配的连接组特征模式。
	} \label{fig:supp_7}
\end{figure}


\begin{figure}[!htb] 
	\centering
	\includegraphics[width=0.85\textwidth]{fig/supp_8.pdf}
	\caption{
	\textbf{通过不同分割分辨率的几何特征模式和离散连接体特征模式实现 7 个关键 HCP 任务对比图和静息态功能耦合的重建精度}。
	有关对比图的详细信息,请参阅第 S2.1 节和补充表 2。 wm = 工作记忆。
	对于所有情况,我们最多使用 200 种模式来直接将结果与研究的其余部分进行比较。
	虚线代表几何特征模式取得的结果(图\ref{fig:supp_1}d);
	请参阅插图中的放大版本。
	浅色到深色的线条表示通过使用以递增分辨率分割的连接矩阵的离散连接组本征模式所实现的结果。
	绘制重建精度与相对于每个完整基组尺寸(即可用模式总数)使用的模式百分比的关系,该百分比对应于皮质表面的顶点数量(对于几何本征模式)或 每个半球的地块(对于离散连接组本征模式)。
	因此,对于几何特征模式,我们使用每个半球 32,492 个可用模式中的前 200 个模式,这就是虚线在可用模式的 0.6\% 处终止的原因。
	对于使用 Schaefer400、Schaefer600、Schaefer800 和 Schaefer1000 分区的离散连接组特征模式,我们分别使用每个半球 200、300、400 和 500 个可用模式中的前 200 个模式。
	因此,实线分别在可用模式的 100\%、66.7\%、0.50\% 和 0.40\% 处终止。
	结果强调,几何本征模提供了大脑活动最简约和紧凑的表示。
	} \label{fig:supp_8}
\end{figure}



\begin{figure}[!htb] 
	\centering
	\includegraphics[width=0.85\textwidth]{fig/supp_9.pdf}
	\caption{
		\textbf{通过 fMRI 数据的主成分分析 (principal component analysis, PCA) 获得主成分}。
		PCA 接受了 7 个关键 HCP 任务对比图和 200 个人的静息状态时间序列的训练。
		有关对比图的详细信息,请参阅第 2.1 节和补充表 2。
		wm = 工作记忆。
		主成分 1-5、10、25、50、100 和 200 从上到下显示。
		负-零-正值被着色为蓝-白-红。
	} \label{fig:supp_9}
\end{figure}


\begin{figure}[!htb] 
	\centering
	\includegraphics[width=0.85\textwidth]{fig/supp_10.pdf}
	\caption{
		\textbf{每个区域的峰值激活时间和 T1w:T2w 的比较}。
		一个半球所有 180 个大脑区域的排名活动曲线时间达到峰值与排名 T1w:T2w 值的关系。
		红线表示排名变量与 Spearman 相关系数 $ r $ 和单边旋转测试 $ p $ 值 pspin 的线性拟合,根据 10,000 个排列进行估计。
	} \label{fig:supp_10}
\end{figure}














\nocite{*}
\printbibliography[heading=bibintoc, title=\ebibname]

\appendix
%\appendixpage
\addappheadtotoc

\end{document}
