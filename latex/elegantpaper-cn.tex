%!TEX program = xelatex
% 完整编译: xelatex -> biber/bibtex -> xelatex -> xelatex
\documentclass[lang=cn,a4paper,newtx]{elegantpaper}

\title{大脑功能的几何约束}
%\author{作者1 \\ 某某大学/机构 \and 作者2 \\ 某某大学/机构}
%\institute{\href{https://elegantlatex.org/}{Elegant\LaTeX{} 项目组}}

%\version{0.11}
\date{}

% 本文档命令
\usepackage{array}
\newcommand{\ccr}[1]{\makecell{{\color{#1}\rule{1cm}{1cm}}}}
\addbibresource[location=local]{reference.bib} % 参考文献,不要删除

\begin{document}

\maketitle

\begin{abstract}
脑部的解剖结构必然限制其功能,但具体的限制方式尚不清楚。在神经科学中,传统且占主导地位的范式是,神经元动态是由复杂的轴突纤维网络连接的离散、功能专一的细胞群体之间的交互作用驱动的。然而,从神经场理论(一个用于模拟大脑大规模活动的既定数学框架)的预测来看,大脑的几何形状可能比复杂的区域间连接性更基本地限制了动态性。在这里,我们通过分析在自发条件和多种任务诱发条件下获取的人类磁共振成像数据来确认这些理论预测。具体来说,我们显示,皮质和皮质下活动可以被简明地理解为大脑几何形状(即其形状)的基本共振模式的激发,而不是像传统观念中所假设的复杂区域间连接性的模式。然后,我们使用这些几何模式来显示,在超过10,000个大脑绘图中,任务诱发的激活并不局限于焦点区域(如广泛认为的那样),而是激发了波长超过60毫米的全脑模式。最后,我们证实了几何形状和功能之间紧密关联是由波动活动的主导作用解释的预测,显示波动动力学可以再现自发和诱发记录的许多典型的空间时间属性。我们的发现挑战了现有的观点,并确定了一个以前被低估的几何形状在塑造功能中的作用,正如大脑全局动态的统一和物理原理模型所预测的那样。
%\keywords{Elegant\LaTeX{},工作论文,模板}
\end{abstract}

\section{主题}

许多自然系统的动态特性在根本上受到其潜在结构的限制。例如,鼓的形状影响其声学特性,河床的形态塑造了水下的流动,蛋白质的几何形状决定了它可以与哪些分子进行相互作用。神经系统也不例外,由轴突互连网支持的解剖分布的神经元群体的丰富和复杂的时空动态特性就是这样。有几个研究已经显示出大脑连接性和活动的各种特性之间的相关性,但神经动态的时空模式如何受到相对稳定的神经解剖支架的限制,这一点尚不清楚。

在物理和工程的多个领域中,系统动态的结构约束可以通过系统的特征模式来理解,这些特征模式是对应系统自然共振模式的基本空间模式。在线性阶段,如大脑在正常(即,非癫痫样)条件下的活动,特征模式(以下也称为模式)为连接大脑解剖结构和塑造活动的物理过程提供了一种特别强大和严格的形式主义。通过这个视角,神经动态的时空模式源自大脑结构特征模式的激发,就像拨动的小提琴弦的和音源自其自身共振模式的振动。

关键的是,就像小提琴弦的共振频率由其长度、密度和张力决定一样,大脑的特征模式由其结构——物理、几何和解剖——属性决定。这些特定的结构属性中是否有任何一个对动态产生主导贡献?在这里,我们测试了两种具有影响力且竞争的理论,这两种理论对大脑结构的哪些关键元素塑造功能做出了不同的预测。

一个传统的观点,代表了神经科学中的主导范式,它源于Ramon y Cajal的神经元学说,Brodmann的细胞结构学以及一个世纪以来将功能定位到特定大脑区域的工作。根据这种观点,神经动态的时空模式源于离散、功能专一的细胞群体之间的交互,这些细胞群体由拓扑复杂的短程和长程轴突连接组成。在人类中,这些连接可以通过扩散磁共振成像(dMRI)在宏观尺度上进行估计,以得到一个称为连接组的结构连接矩阵。这种方法已被广泛用于理解大脑组织和动态,最近的工作提出,从这样的离散连接组中得到的特征模式——在这里被称为连接组特征模式——可以用来重构人类皮层的典型功能网络的空间模式,这些网络是通过功能性磁共振成像(fMRI)映射的。

这种基于离散连接组的观点的一个限制是,它依赖于一个不直接考虑大脑解剖学的物理属性和空间嵌入(即,几何和拓扑)的抽象表示。这些特征被明确地纳入到一大类神经场理论(NFTs)中,这些理论描述了在0.5毫米以上的空间尺度上的平均场神经动态(补充信息1)。一种受生理限制的NFT形式已经统一了各种不同的实证现象,它将皮层活动视为通过物理连续的神经组织片层传播的行波的叠加。在这个理论中,不同皮质位置之间的神经交互被近似为一个随距离大致指数衰减的均匀空间核,这与实验证据一致,证据显示,无数种物种的神经系统的组织都受到连接性的指数距离规则(EDR)的普遍管理。

鉴于波动动态和EDR样的连接性,NFT(神经场理论)的一个关键预测是,大脑的内在几何形状物理上塑造并对出现的动态施加边界条件。这个观点的一个显著的推论是,如果我们优先考虑大脑解剖学的空间和物理限制,我们只需要考虑大脑的形状,而不是它的全系列拓扑复杂的轴突互连性,就可以理解空间模式化的活动。更正式地说,这个理论预测,从大脑几何形状中得到的特征模式——在这里被称为几何特征模式——比连接组更基本的解剖学约束动态。这个观点与传统观点形成鲜明对比,传统观点认为区域间解剖连接的复杂模式塑造大脑活动。

在这里,我们测试了这些关于大脑的竞争观点,目的是确定人脑动态的主要结构约束。与来自NFT(神经场理论)的理论预测一致,我们展示了从人类新皮层的自发和任务诱发记录中获取的多样化的实验性fMRI数据,可以更简洁地由从皮层几何形状(几何特征模式)得出的特征模式解释,而不是由大脑连接性(连接组特征模式)的测量得出的那些。我们进一步确认,刺激诱发的活动主要由具有长空间波长的几何特征模式的激发主导,这挑战了传统观点,即这种活动局限于局部、空间隔离的簇。为了直接将这些结构约束与驱动大脑动态的物理过程联系起来,我们使用了一个生成模型,展示了如何在皮层的几何形状上展开的波动动态可以解释功能性大脑组织的多样性特征。最后,我们展示了通过特征模式捕获的几何与功能之间的紧密关系扩展到非新皮层结构,表明这种链接是大脑组织的普遍属性。

\section{几何模型约束皮层激活}
我们首先检查几何特征模式能在多大程度上解释人类新皮层活动的多样性。为了推导出这些特征模式,我们使用一个人口平均模板的新皮层表面的网格表示(图1a和方法中的皮层几何特征模式的推导)。然后,我们从这个表面网格构建拉普拉斯-贝尔特拉米算子(LBO),该算子捕捉了局部顶点到顶点的空间关系和曲率,并解决特征值问题,(编者注:在微分几何中,拉普拉斯算子可以推广为定义在曲面,或更一般地黎曼流形与伪黎曼流形上,函数的算子。这个更一般的算子叫做拉普拉斯-贝尔特拉米算子(Laplace–Beltrami operator))。




\subsection{注意事项}
\textbf{文献部分}:我们将 bibtex 的默认文献编译方式改为 biblatex,不过我们也提供了两个后端,\lstinline{bibend=biber} 和 \lstinline{bibend=bibtex}。特别需要注意的是从 0.10 开始,文献文件改为 \lstinline{reference.bib},与 ElegantBook 保持一致,而参考文献的引文样式等更多格式,请参考后文参考文献部分,更多样式可以参考 biblatex 文档。 

\textbf{字体部分},我们将 newtxtext 宏包的支持方式改为了字体名称设定方式,设定英文字体为 TeX Gyre Terms/Heros,英文字体部分,根据编译方式选择不同字体。对于一般用户而言,不太需要关心这部分内容。

另外,中文请务必使用 \hologo{XeLaTeX} 编译。

\subsection{模板介绍}

此模板基于 \LaTeX{} 的标准文类 article 设计,所以 article 文类的选项也能传递给本模板,比如 \lstinline{a4paper, 11pt} 等等。

\begin{lstlisting}
\documentclass[a4paper,11pt]{elegantpaper}
\end{lstlisting}

\textbf{注意}:Elegant\LaTeX{} 系列模板已经全部上传至 \href{https://www.overleaf.com/latex/templates/elegantpaper-template/yzghrqjhmmmr}{Overleaf} 上,用户可以在线使用。另外,为了方便国内用户,模板也已经传至\href{https://gitee.com/ElegantLaTeX/ElegantPaper}{码云}。


\subsection{全局选项}
此模板定义了一个语言选项 \lstinline{lang},可以选择英文模式 \lstinline{lang=en}(默认)或者中文模式 \lstinline{lang=cn}。当选择中文模式时,图表的标题引导词以及参考文献,定理引导词等信息会变成中文。你可以通过下面两种方式来选择语言模式:
\begin{lstlisting}
\documentclass[lang=cn]{elegantpaper} % or
\documentclass[cn]{elegantpaper} 
\end{lstlisting}

\textbf{注意:} 英文模式下,由于没有添加中文宏包,无法输入中文。如果需要输入中文,可以通过在导言区引入中文宏包 \lstinline{ctex} 或者加入 \lstinline{xeCJK} 宏包后自行设置字体。 
\begin{lstlisting}
\usepackage[UTF8,scheme=plain]{ctex}
\end{lstlisting}

\subsection{数学字体选项}

本模板定义了一个数学字体选项(\lstinline{math}),可选项有三个:
\begin{enumerate}
  \item \lstinline{math=cm}(默认),使用 \LaTeX{} 默认数学字体(推荐,无需声明);
  \item \lstinline{math=newtx},使用 \lstinline{newtxmath} 设置数学字体(潜在问题比较多)。
  \item \lstinline{math=mtpro2},使用 \lstinline{mtpro2} 宏包设置数学字体,要求用户已经成功安装此宏包。
\end{enumerate}

\subsection{中文字体选项}

模板提供中文字体选项 \lstinline{chinesefont},可选项有
\begin{enumerate}
  \item \lstinline{ctexfont}:默认选项,使用 \lstinline{ctex} 宏包根据系统自行选择字体,可能存在字体缺失的问题,更多内容参考 \lstinline{ctex} 宏包\href{https://ctan.org/pkg/ctex}{官方文档}\footnote{可以使用命令提示符,输入 \lstinline{texdoc ctex} 调出本地 \lstinline{ctex} 宏包文档}。
  \item \lstinline{founder}:方正字体选项(\textbf{需要安装方正字体}),后台调用 \lstinline{ctex} 宏包并且使用 \lstinline{fontset=none} 选项,然后设置字体为方正四款免费字体,方正字体下载注意事项见后文,用户只需要安装方正字体即可使用该选项。
  \item \lstinline{nofont}:后台会调用 \lstinline{ctex} 宏包并且使用 \lstinline{fontset=none} 选项,不设定中文字体,用户可以自行设置中文字体,具体见后文。
\end{enumerate}

\subsubsection{方正字体选项}
由于使用 \lstinline{ctex} 宏包默认调用系统已有的字体,部分系统字体缺失严重,因此,用户希望能够使用其它字体,我们推荐使用方正字体。方正的{\songti 方正书宋}、{\heiti 方正黑体}、{\kaishu 方正楷体}、{\fangsong 方正仿宋}四款字体均可免费试用,且可用于商业用途。用户可以自行从\href{http://www.foundertype.com/}{方正字体官网}下载此四款字体,在下载的时候请\textbf{务必}注意选择 GBK 字符集,也可以使用 \href{https://www.latexstudio.net/}{\LaTeX{} 工作室}提供的\href{https://pan.baidu.com/s/1BgbQM7LoinY7m8yeP25Y7Q}{方正字体,提取码为:njy9} 进行安装。安装时,{\kaishu Win 10 用户请右键选择为全部用户安装,否则会找不到字体。}

\begin{figure}[!htb]
\centering
\includegraphics[width=0.9\textwidth]{founder.png}
\end{figure}

\subsubsection{其他中文字体}
如果你想完全自定义字体\footnote{这里仍然以方正字体为例。},你可以选择 \lstinline{chinesefont=nofont},然后在导言区设置即可,可以参考下方代码:
\begin{lstlisting}
\setCJKmainfont[BoldFont={FZHei-B01},ItalicFont={FZKai-Z03}]{FZShuSong-Z01}
\setCJKsansfont[BoldFont={FZHei-B01}]{FZKai-Z03}
\setCJKmonofont[BoldFont={FZHei-B01}]{FZFangSong-Z02}
\setCJKfamilyfont{zhsong}{FZShuSong-Z01}
\setCJKfamilyfont{zhhei}{FZHei-B01}
\setCJKfamilyfont{zhkai}[BoldFont={FZHei-B01}]{FZKai-Z03}
\setCJKfamilyfont{zhfs}[BoldFont={FZHei-B01}]{FZFangSong-Z02}
\newcommand*{\songti}{\CJKfamily{zhsong}}
\newcommand*{\heiti}{\CJKfamily{zhhei}}
\newcommand*{\kaishu}{\CJKfamily{zhkai}}
\newcommand*{\fangsong}{\CJKfamily{zhfs}}
\end{lstlisting}



\subsection{自定义命令}
此模板并没有修改任何默认的 \LaTeX{} 命令或者环境\footnote{目的是保证代码的可复用性,请用户关注内容,不要太在意格式,这才是本工作论文模板的意义。}。另外,本模板可以使用的 4 个额外命令:
\begin{enumerate}
  \item \lstinline{\email}:创建邮箱地址的链接,比如 \email{xxx@outlook.com};
  \item \lstinline{\figref}:用法和 \lstinline{\ref} 类似,但是会在插图的标题前添加 <\textbf{图 n}> ;
  \item \lstinline{\tabref}:用法和 \lstinline{\ref} 类似,但是会在表格的标题前添加 <\textbf{表 n}>;
  \item \lstinline{\keywords}:为摘要环境添加关键词。
\end{enumerate}

\subsection{参考文献}

文献部分,本模板调用了 biblatex 宏包,并提供了 biber(默认) 和 bibtex 两个后端选项,可以使用 \lstinline{bibend} 进行修改:

\begin{lstlisting}
  \documentclass[bibtex]{elegantpaper}
  \documentclass[bibend=bibtex]{elegantpaper}
\end{lstlisting}

关于文献条目(bib item),你可以在谷歌学术,Mendeley,Endnote 中取,然后把它们添加到 \lstinline{reference.bib} 中。在文中引用的时候,引用它们的键值(bib key)即可。

为了方便文献样式修改,模板引入了 \lstinline{bibstyle} 和 \lstinline{citestyle} 选项,默认均为数字格式(numeric),参考文献示例:\cite{cn1,en2,en3} 使用了中国一个大型的 P2P 平台(人人贷)的数据来检验男性投资者和女性投资者在投资表现上是否有显著差异。

如果需要设置为国标 GB7714-2015,需要使用:
\begin{lstlisting}
  \documentclass[citestyle=gb7714-2015, bibstyle=gb7714-2015]{elegantpaper} 
\end{lstlisting}

如果需要添加排序方式,可以在导言区加入
\begin{lstlisting}
  \ExecuteBibliographyOptions{sorting=ynt}
\end{lstlisting}

启用国标之后,可以加入 \lstinline{sorting=gb7714-2015}。


\section{使用 newtx 系列字体}

如果需要使用原先版本的 \lstinline{newtx} 系列字体,可以通过显示声明数学字体:

\begin{lstlisting}
\documentclass[math=newtx]{elegantpaper}
\end{lstlisting}

\subsection{连字符}

如果使用 \lstinline{newtx} 系列字体宏包,需要注意下连字符的问题。
\begin{equation}
  \int_{R^q} f(x,y) dy.\emph{of\kern0pt f}
\end{equation}

\begin{lstlisting}
\begin{equation}
  \int_{R^q} f(x,y) dy.\emph{of \kern0pt f}
\end{equation}
\end{lstlisting}

\subsection{宏包冲突}

有用户反馈模板在使用 \lstinline{yhmath} 以及 \lstinline{esvect} 等宏包时会报错:
\begin{lstlisting}
LaTeX Error:
   Too many symbol fonts declared.
\end{lstlisting}

原因是在使用 \lstinline{newtxmath} 宏包时,重新定义了数学字体用于大型操作符,达到了 {\heiti 最多 16 个数学字体} 的上限,在调用其他宏包的时候,无法新增数学字体。为了减少调用非常用宏包,在此给出如何调用 \lstinline{yhmath} 以及 \lstinline{esvect} 宏包的方法。

请在 \lstinline{elegantpaper.cls} 内搜索 \lstinline{yhmath} 或者 \lstinline{esvect},将你所需要的宏包加载语句\textit{取消注释}即可。


\section{常见问题 FAQ}

\begin{enumerate}[label=\arabic*).]
  \item \textit{如何删除版本信息?}\\
    导言区不写 \lstinline|\version{x.xx}| 即可。
  \item \textit{如何删除日期?}\\
    需要注意的是,与版本 \lstinline{\version} 不同的是,导言区不写或注释 \lstinline{\date} 的话,仍然会打印出当日日期,原因是 \lstinline{\date} 有默认参数。如果不需要日期的话,日期可以留空即可,也即 \lstinline|\date{}|。
  \item \textit{如何获得中文日期?}\\
    为了获得中文日期,必须在中文模式下\footnote{英文模式下,由于未加载中文宏包,无法输入中文。},使用 \lstinline|\date{\zhdate{2019/10/11}}|,如果需要当天的汉化日期,可以使用 \lstinline|\date{\zhtoday}|,这两个命令都来源于 \href{https://ctan.org/pkg/zhnumber}{\lstinline{zhnumber}} 宏包。
  \item \textit{如何添加多个作者?}\\
    在 \lstinline{\author} 里面使用 \lstinline{\and},作者单位可以用 \lstinline{\\} 换行。
    \begin{lstlisting}
    \author{author 1\\ org. 1 \and author 2 \\ org. 2 }
    \end{lstlisting}
  \item \textit{如何添加中英文摘要?}\\
    请参考 \href{https://github.com/ElegantLaTeX/ElegantPaper/issues/5}{GitHub::ElegantPaper/issues/5}
\end{enumerate}


\section{致谢}

特别感谢 \href{https://github.com/sikouhjw}{sikouhjw} 和 \href{https://github.com/syvshc}{syvshc}  长期以来对于 Github 上 issue 的快速回应,以及各个社区论坛对于 ElegantLaTeX 相关问题的回复。特别感谢 ChinaTeX 以及 \href{http://www.latexstudio.net/}{LaTeX 工作室} 对于本系列模板的大力宣传与推广。

如果你喜欢我们的模板,你可以在 Github 上收藏我们的模板。

\nocite{*}
\printbibliography[heading=bibintoc, title=\ebibname]

\appendix
%\appendixpage
\addappheadtotoc

\end{document}
