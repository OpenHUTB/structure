%!TEX program = xelatex
% !BIB program = biber
% 完整编译: xelatex -> biber/bibtex -> xelatex -> xelatex
\documentclass[lang=cn,a4paper,newtx]{elegantpaper}

% 参考:https://weibo.com/ttarticle/p/show?id=2309404908550236537030
\title{大脑功能的几何约束}
%\author{作者1 \\ 某某大学/机构 \and 作者2 \\ 某某大学/机构}
%\institute{\href{https://elegantlatex.org/}{Elegant\LaTeX{} 项目组}}

%\version{0.11}
\date{}

% 本文档命令
\usepackage{array}
\newcommand{\ccr}[1]{\makecell{{\color{#1}\rule{1cm}{1cm}}}}
\addbibresource[location=local]{reference.bib} % 参考文献,不要删除

\begin{document}

\maketitle

\begin{abstract}
脑部的解剖结构必然限制其功能,但具体的限制方式尚不清楚。在神经科学中,传统且占主导地位的范式是,神经元动态是由复杂的轴突纤维网络连接的离散、功能专一的细胞群体之间的交互作用驱动的。然而,从神经场理论(一个用于模拟大脑大规模活动的既定数学框架)的预测来看,大脑的几何形状可能比复杂的区域间连接性更基本地限制了动态性。在这里,我们通过分析在自发条件和多种任务诱发条件下获取的人类磁共振成像数据来确认这些理论预测。具体来说,我们显示,皮质和皮质下活动可以被简明地理解为大脑几何形状(即其形状)的基本共振模式的激发,而不是像传统观念中所假设的复杂区域间连接性的模式。然后,我们使用这些几何模式来显示,在超过10,000个大脑绘图中,任务诱发的激活并不局限于焦点区域(如广泛认为的那样),而是激发了波长超过60毫米的全脑模式。最后,我们证实了几何形状和功能之间紧密关联是由波动活动的主导作用解释的预测,显示波动动力学可以再现自发和诱发记录的许多典型的空间时间属性。我们的发现挑战了现有的观点,并确定了一个以前被低估的几何形状在塑造功能中的作用,正如大脑全局动态的统一和物理原理模型所预测的那样。
%\keywords{Elegant\LaTeX{},工作论文,模板}
\end{abstract}

\section{主题}

许多自然系统的动态特性在根本上受到其潜在结构的限制。例如,鼓的形状影响其声学特性,河床的形态塑造了水下的流动,蛋白质的几何形状决定了它可以与哪些分子进行相互作用。神经系统也不例外,由轴突互连网支持的解剖分布的神经元群体的丰富和复杂的时空动态特性就是这样。有几个研究已经显示出大脑连接性和活动的各种特性之间的相关性,但神经动态的时空模式如何受到相对稳定的神经解剖支架的限制,这一点尚不清楚。

在物理和工程的多个领域中,系统动态的结构约束可以通过系统的特征模式来理解,这些特征模式是对应系统自然共振模式的基本空间模式。在线性阶段,如大脑在正常(即,非癫痫样)条件下的活动,特征模式(以下也称为模式)为连接大脑解剖结构和塑造活动的物理过程提供了一种特别强大和严格的形式主义。通过这个视角,神经动态的时空模式源自大脑结构特征模式的激发,就像拨动的小提琴弦的和音源自其自身共振模式的振动。

关键的是,就像小提琴弦的共振频率由其长度、密度和张力决定一样,大脑的特征模式由其结构——物理、几何和解剖——属性决定。这些特定的结构属性中是否有任何一个对动态产生主导贡献?在这里,我们测试了两种具有影响力且竞争的理论,这两种理论对大脑结构的哪些关键元素塑造功能做出了不同的预测。

一个传统的观点,代表了神经科学中的主导范式,它源于Ramon y Cajal的神经元学说,Brodmann的细胞结构学以及一个世纪以来将功能定位到特定大脑区域的工作。根据这种观点,神经动态的时空模式源于离散、功能专一的细胞群体之间的交互,这些细胞群体由拓扑复杂的短程和长程轴突连接组成。在人类中,这些连接可以通过扩散磁共振成像(dMRI)在宏观尺度上进行估计,以得到一个称为连接组的结构连接矩阵。这种方法已被广泛用于理解大脑组织和动态,最近的工作提出,从这样的离散连接组中得到的特征模式——在这里被称为连接组特征模式——可以用来重构人类皮层的典型功能网络的空间模式,这些网络是通过功能性磁共振成像(fMRI)映射的。

这种基于离散连接组的观点的一个限制是,它依赖于一个不直接考虑大脑解剖学的物理属性和空间嵌入(即,几何和拓扑)的抽象表示。这些特征被明确地纳入到一大类神经场理论(NFTs)中,这些理论描述了在0.5毫米以上的空间尺度上的平均场神经动态(补充信息1)。一种受生理限制的NFT形式已经统一了各种不同的实证现象,它将皮层活动视为通过物理连续的神经组织片层传播的行波的叠加。在这个理论中,不同皮质位置之间的神经交互被近似为一个随距离大致指数衰减的均匀空间核,这与实验证据一致,证据显示,无数种物种的神经系统的组织都受到连接性的指数距离规则(EDR)的普遍管理。

鉴于波动动态和EDR样的连接性,NFT(神经场理论)的一个关键预测是,大脑的内在几何形状物理上塑造并对出现的动态施加边界条件。这个观点的一个显著的推论是,如果我们优先考虑大脑解剖学的空间和物理限制,我们只需要考虑大脑的形状,而不是它的全系列拓扑复杂的轴突互连性,就可以理解空间模式化的活动。更正式地说,这个理论预测,从大脑几何形状中得到的特征模式——在这里被称为几何特征模式——比连接组更基本的解剖学约束动态。这个观点与传统观点形成鲜明对比,传统观点认为区域间解剖连接的复杂模式塑造大脑活动。

在这里,我们测试了这些关于大脑的竞争观点,目的是确定人脑动态的主要结构约束。与来自NFT(神经场理论)的理论预测一致,我们展示了从人类新皮层的自发和任务诱发记录中获取的多样化的实验性fMRI数据,可以更简洁地由从皮层几何形状(几何特征模式)得出的特征模式解释,而不是由大脑连接性(连接组特征模式)的测量得出的那些。我们进一步确认,刺激诱发的活动主要由具有长空间波长的几何特征模式的激发主导,这挑战了传统观点,即这种活动局限于局部、空间隔离的簇。为了直接将这些结构约束与驱动大脑动态的物理过程联系起来,我们使用了一个生成模型,展示了如何在皮层的几何形状上展开的波动动态可以解释功能性大脑组织的多样性特征。最后,我们展示了通过特征模式捕获的几何与功能之间的紧密关系扩展到非新皮层结构,表明这种链接是大脑组织的普遍属性。

\section{几何模型约束皮层激活}
我们首先检查几何特征模式能在多大程度上解释人类新皮层活动的多样性。
为了推导出这些特征模式,我们使用一个群体平均模板的新皮层表面网格表示(图\ref{fig:1}a和方法\ref{sec:derivation}中的皮层几何特征模式的推导)。
然后,我们从这个表面网格构建拉普拉斯-贝尔特拉米算子(Laplace–
Beltrami operator, LBO),该算子捕捉了局部顶点到顶点的空间关系和曲率,并解决特征值问题\footnote{在微分几何中,拉普拉斯算子可以推广为定义在曲面,或更一般地黎曼流形与伪黎曼流形上,函数的算子。这个更一般的算子叫做拉普拉斯-贝尔特拉米算子},

% 有*号就没公式标号
\begin{equation} \label{eq:1}
	\nabla^2 \psi = \Delta\psi = -\lambda \psi,
\end{equation}


其中 $ \nabla $ 是梯度算子,$ \Delta $ 是LBO,$ \psi = \{\psi_1(r), \psi_2(r),...\} $ 是一组几何特征模式,对应的特征值为 $ \lambda = \{ \lambda_1, \lambda_2, ... \} $。
特征值按照每个模式的空间频率或波长的顺序进行排序(图\ref{fig:1}a和扩展数据图\ref{fig:extended_fig_1}),因此$ \psi_1 $是波长最长的模式。
得到的特征模式是正交的,形成一个完整的基础集,可以将在皮层上展开的时空动态分解为具有不同波长的模式的加权和(图\ref{fig:1}b和方法\ref{sec:modal_decomposition}中的大脑活动的模式分解)。
除非另有说明,我们在本研究中使用$ N = 200 $个模式\footnote{"模式" (modes) 是一个技术术语,它是指振动系统的特殊状态,这些状态可以独立地进行简谐振动。在这里,它们是指从大脑结构特性(如几何形状)中推导出的特定空间模式,这些模式可能对大脑活动产生影响。}。)


\begin{figure}[!htb]
	\centering
	\includegraphics[width=0.9\textwidth]{fig/fig_1.pdf}
	\caption{\textbf{使用几何特征模式重建新皮层活动}。
	\textbf{a}, 通过解决特征值问题,$ \Delta \psi = -\lambda \psi $(公式(\ref{eq:1})),从皮层表面网格得到几何特征模式。
	模式$ \psi_1, \psi_2, \psi_3, ... \psi_N $ 按空间频率从低到高(空间波长从长到短)排序。
	负值、零和正值分别用蓝色、白色和红色标记。
	\textbf{b}, 大脑活动数据的模态分解。
	示例展示了如何将给定时间$ t $的空间图 $ y(r,t) $ 分解为加权的模式 $ \psi_j $ 之和。
	\textbf{c}, 左图,我们使用一系列刺激对比的激活空间图重建任务诱发的数据。
	右图,我们通过在每个时间帧分解空间图并生成区域到区域的功能耦合(FC,functional coupling)矩阵来重建自发活动。
	\textbf{d}, 重建七个关键HCP任务对比图(补充信息2.1)和静息状态功能耦合作为模式数量函数的准确性。
	插图显示皮层表面重建,展示了前10个、100个和200个模式所对应的空间尺度,分别对应大约120毫米、40毫米和30毫米的空间波长。
	} \label{fig:1}
\end{figure}




\section{方法} \label{sec:method}

\subsection{皮层几何特征模式求导} \label{sec:derivation}

% Each eigenmode
每个特征模式包括具有特定空间波长的空间图案。
根据参考文献\cite{robinson2016eigenmodes},我们使用理想化的球形情况来近似特征模式波长,因为它在拓扑上与人类皮层相当。
通过在球体上求解方程\ref{eq:1},存在退化解,使得某些特征模式具有相同的特征值和空间波长——这类似于量子物理学中的球谐函数。
事实上,由于特征模式将在皮层折叠消失的极限下接近球谐函数,因此前者可以组合成具有空间波长的本征群,

\begin{equation}
	\text{wavelength} = \frac{2 \pi R_s}{\sqrt{l(l+1)}},
\end{equation}

其中$R_s$是球体的半径(对于本研究中使用的fsaverage总体平均模板,$R_s \approx 67.0 mm$),$l$是本征群数(原子物理学中的角动量量子数)。
前15个本征群的波长和本征群中包括的本征模如补充表1所示。


\subsection{大脑活动的模式分解} \label{sec:modal_decomposition}


\section{扩展数据的图}
\begin{figure}[!htb] 
	\centering
	\includegraphics[width=0.9\textwidth]{fig/extended_fig_1.pdf}
	\caption{特征模式基组} \label{fig:extended_fig_1}
\end{figure}


\nocite{*}
\printbibliography[heading=bibintoc, title=\ebibname]

\appendix
%\appendixpage
\addappheadtotoc

\end{document}
