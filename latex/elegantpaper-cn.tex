%!TEX program = xelatex
% !BIB program = biber
% 完整编译: xelatex -> biber/bibtex -> xelatex -> xelatex
\documentclass[lang=cn,a4paper,newtx]{elegantpaper}

% 参考:https://weibo.com/ttarticle/p/show?id=2309404908550236537030
\title{大脑功能的几何约束}
%\author{作者1 \\ 某某大学/机构 \and 作者2 \\ 某某大学/机构}
%\institute{\href{https://elegantlatex.org/}{Elegant\LaTeX{} 项目组}}

%\version{0.11}
\date{}

% 本文档命令
\usepackage{array}

\usepackage{makecell}

\newcommand{\ccr}[1]{\makecell{{\color{#1}\rule{1cm}{1cm}}}}
\addbibresource[location=local]{reference.bib} % 参考文献,不要删除

\begin{document}

\maketitle

\begin{abstract}
脑部的解剖结构必然限制其功能,但具体的限制方式尚不清楚。
在神经科学中,传统且占主导地位的范式是:神经元动态是由复杂的轴突纤维网络连接的离散、功能专一的细胞群体之间的交互作用驱动的。
然而,从\textit{神经场理论}(一个用于模拟大脑大规模活动的既定数学框架)的预测来看,大脑的几何形状可能比复杂的区域间连接性更基本地限制了动态性。
在这里,我们通过分析在自发条件和多种任务诱发条件下获取的人类核磁共振成像数据来确认这些理论预测。
具体来说,我们指出皮层和皮层下活动可以被简明地理解为大脑几何形状(即其形状)的基本共振模式的激发,而不是像传统观念中所假设的复杂区域间连接性的模式。
然后,我们使用这些几何模式来显示,在超过 10,000 个大脑绘图中,任务诱发的激活并不局限于焦点区域(如广泛认为的那样),而是激发了波长超过 60 毫米的全脑模式。
最后,我们证实了几何形状和功能之间紧密关联是由波动活动的主导作用解释的预测,显示波动动力学可以再现自发和诱发记录的许多典型的空间时间属性。
我们的发现挑战了现有的观点,并确定了一个以前被低估的几何形状在塑造功能中的作用,正如大脑全局动态的统一和物理原理模型所预测的那样。
%\keywords{Elegant\LaTeX{},工作论文,模板}
\end{abstract}

\section{主题}

许多自然系统的动态特性在根本上受到其潜在结构的限制。
例如,鼓的形状影响其声学特性,河床的形态塑造了水下的流动,蛋白质的几何形状决定了它可以与哪些分子进行相互作用。
神经系统也不例外,由轴突互连网支持的解剖分布的神经元群体的丰富和复杂的时空动态特性就是这样。
有几个研究已经显示出大脑连接性和活动的各种特性之间的相关性,但神经动态的时空模式如何受到相对稳定的神经解剖支架的限制,这一点尚不清楚。


在物理和工程的多个领域中,系统动态结构约束可以通过系统的特征模式来理解,这些特征模式是对应系统自然共振模式的基本空间模式。
在线性阶段,如大脑在正常(即不是类癫痫)条件下的活动,特征模式(以下也称为模式)为连接大脑解剖结构和塑造活动的物理过程提供了一种特别强大和严格的形式主义。
通过这个视角,神经动态的时空模式源自大脑结构特征模式的激发,就像拨动的小提琴弦的和音源自其自身共振模式的振动。


关键的是,就像小提琴弦的共振频率由其长度、密度和张力决定一样,大脑的特征模式由其结构——物理、几何和解剖——属性决定。
这些特定的结构属性中是否有任何一个对动态产生主导贡献?
在这里,我们测试了两种具有影响力且竞争的理论,这两种理论对大脑结构的哪些关键元素塑造功能做出了不同的预测。


一个是代表了神经科学主导范式的的传统观点,它源于卡哈尔的神经元学说、布罗德曼的细胞结构学以及一个世纪以来将功能定位到特定大脑区域的工作。
根据这种观点,神经动态的时空模式源于离散、功能专一的细胞群体之间的交互,这些细胞群体由拓扑复杂的短程和长程轴突连接组成。
在人类中,这些连接可以通过磁共振弥散成像(diffusion Magnetic Resonance Imaging, dMRI)在宏观尺度上进行估计,以得到一个称为连接组的结构连接矩阵。
这种方法已被广泛用于理解大脑组织和动态,最近的工作提出,从这样的离散连接组中得到的特征模式——在这里被称为连接组特征模式——可以用来重构人类皮层的典型功能网络的空间模式,这些网络是通过功能性磁共振成像(functional MRI, fMRI)映射的。


% One limitation of
这种基于离散连接组的观点的一个限制是:它依赖于一个不直接考虑大脑解剖学的物理属性和空间嵌入(即几何和拓扑)的抽象表示。
这些特征被明确地纳入到一大类\textit{神经场理论}(Neural Field Theories, NFTs)中,
这些理论描述了在0.5毫米以上的空间尺度上的平均场神经动力学(补充信息\ref{sec:NFT})。
一种受生理限制的\textit{神经场理论}形式已经统一了各种不同的实证现象,它将皮层活动视为通过物理连续的神经组织片层传播的行波的叠加。
在这个理论中,不同皮层位置之间的神经交互被近似为一个随距离大致指数衰减的均匀空间核,这与实验证据一致,证据显示,无数种物种的神经系统的组织都受到连接性的\textit{指数距离规则}(Exponential Distance Rule, EDR)的普遍管理。


% Given wave-like dynamics
鉴于类波动力学和类\textit{指数距离规则}的连接性,\textit{神经场理论}的一个关键预测是,大脑的内在几何形状物理上塑造并对出现的动态施加边界条件。
这个观点的一个显著的推论是,如果我们优先考虑大脑解剖学的空间和物理限制,我们只需要考虑大脑的形状,而不是它的全系列拓扑复杂的轴突互连性,就可以理解空间模式化的活动。
更正式地说,这个理论预测,从大脑几何形状中得到的特征模式(在这里被称为\textit{几何特征模式}),表示了一个比连接组更根本的动力学解剖约束。
这个观点与传统观点形成鲜明对比,传统观点认为区域间解剖连接的复杂模式塑造了大脑活动。


% Here we test
在这里,我们测试了这些关于大脑相互竞争的观点,目的是确定人脑动态的主要结构约束。
与来自\textit{神经场理论}的预测一致,我们展示了从人类新皮层的自发和任务诱发记录中获取的多样化的实验性 fMRI 数据,可以更简洁地由从皮层几何形状(\textit{几何特征模式})得出的特征模式解释,而不是由大脑连接性(连接组特征模式)的测量得出的那些。
我们进一步确认,刺激诱发的活动主要由具有长空间波长的\textit{几何特征模式}的激发主导,这挑战了这种活动局限于局部、空间孤立的团簇的经典观点。
为了直接将这些结构约束与驱动大脑动态的物理过程联系起来,我们使用了一个生成模型,展示了如何在皮层的几何形状上展开的波动力学可以解释功能性大脑组织的多样性特征。
最后,我们展示了通过特征模式捕获的几何与功能之间的紧密关系扩展到非新皮层结构,表明这种联系是大脑组织的普遍属性。


\section{几何模型约束皮层激活}
我们首先检查\textit{几何特征模式}能在多大程度上解释人类新皮层活动的多样性。
为了推导出这些特征模式,我们使用一个群体平均模板的新皮层表面网格表示(图\ref{fig:1}a和方法\ref{sec:derivation}中的皮层\textit{几何特征模式}的推导)。
然后,我们从这个表面网格构建\textit{拉普拉斯-贝尔特拉米算子}(Laplace–
Beltrami operator, LBO),该算子捕捉了局部顶点到顶点的空间关系和曲率,并解决特征值问题\footnote{在微分几何中,拉普拉斯算子可以推广为定义在曲面,或更一般地黎曼流形与伪黎曼流形上,函数的算子。这个更一般的算子叫做拉普拉斯-贝尔特拉米算子},

% 有*号就没公式标号
\begin{equation} \label{eq:1}
	\nabla^2 \psi = \Delta\psi = -\lambda \psi,
\end{equation}


% \nabla (那勃勒)是哈密顿算子(在量子力学中为一个可观测量,对应于系统的总能量);矢量微分算子;
% \psi是振幅
其中 $ \nabla $ 是梯度算子,$ \Delta $ 是\textit{拉普拉斯-贝尔特拉米算子},$ \psi = \{\psi_1(r), \psi_2(r),...\} $ 是一组\textit{几何特征模式},对应的特征值为 $ \lambda = \{ \lambda_1, \lambda_2, ... \} $。
特征值按照每个模式的空间频率或波长的顺序进行排序(图\ref{fig:1}a和扩展数据图\ref{fig:extended_fig_1}),因此$ \psi_1 $是波长最长的模式。
得到的特征模式是正交的,形成一个完整的基础集,可以将在皮层上展开的时空动态分解为具有不同波长的模式的加权和(图\ref{fig:1}b和方法\ref{sec:modal_decomposition}中的大脑活动的模式分解)。
除非另有说明,我们在本研究中使用$ N = 200 $个模式\footnote{"模式" (modes) 是一个技术术语,它是指振动系统的特殊状态,这些状态可以独立地进行简谐振动。在这里,它们是指从大脑结构特性(如几何形状)中推导出的特定空间模式,这些模式可能对大脑活动产生影响。}。)


\begin{figure}[!htb]
	\centering
	\includegraphics[width=0.9\textwidth]{fig/fig_1.pdf}
	\caption{\textbf{使用几何特征模式重建新皮层活动}。
	\textbf{a}, 通过解决特征值问题,$ \Delta \psi = -\lambda \psi $(公式(\ref{eq:1})),从皮层表面网格得到\textit{几何特征模式}。
	模式$ \psi_1, \psi_2, \psi_3, ... \psi_N $ 按空间频率从低到高(空间波长从长到短)排序。
	负值、零和正值分别用蓝色、白色和红色标记。
	\textbf{b}, 大脑活动数据的模态分解。
	示例展示了如何将给定时间$ t $的空间图 $ y(r,t) $ 分解为加权的模式 $ \psi_j $ 之和。
	\textbf{c}, 左图,我们使用一系列刺激对比的激活空间图重建任务诱发的数据。
	右图,我们通过在每个时间帧分解空间图并生成区域到区域的功能耦合(,functional coupling)矩阵来重建自发活动。
	\textbf{d}, 重建七个关键\textit{人类连接组计划}任务对比图(补充信息2.1)和静息状态功能耦合作为模式数量函数的准确性。
	插图显示皮层表面重建,展示了前10个、100个和200个模式所对应的空间尺度,分别对应大约120毫米、40毫米和30毫米的空间波长。
	 \textbf{e}, 使用10个、100个和200个模式得到的7个关键\textit{人类连接组计划}任务对比的群体平均实验任务激活图和重建图(recon)。
	 黑色箭头指示出当使用短波长模式时能更准确地重建的局部激活模式。 
	 \textbf{f}, 使用10个、100个和200个模式的群体平均实验静息状态功能耦合矩阵和重建。
	} \label{fig:1}
\end{figure}


利用这种分解,我们评估了\textit{几何特征模式}在捕获任务诱导大脑活动和自发大脑活动(图\ref{fig:1}c)方面的准确性,这些活动是从人类连接组项目(Human Connectome Project, HCP; 方法和补充信息\ref{sec:sup_2}中的\textit{人类连接组项目}数据)中的255个健康个体中测量出来的。
对于任务诱导的活动,我们绘制了从七种不同任务中获取的 47 种基于任务的对比,这些任务代表了不同的诱导活动模式。
然后,我们使用逐渐增加的模式数量(最多200个)来重建每个个体的激活图(图\ref{fig:1}d)。
对于自发的、无任务的(所谓的静息态)活动,我们在每个时间帧重建活动的空间图,然后生成一个区域到区域的功能耦合矩阵,描述 180 个每半球离散大脑区域之间的活动相关性。
为了使任务诱导和自发记录之间能够直接进行比较,我们将相同的区域划分应用于任务诱导数据(方法中的皮层划分)。
最后,我们通过计算实证和重建任务诱导激活图以及自发功能耦合矩阵之间的相关性来量化重建的准确性(图\ref{fig:1}d-f)。


我们观察到,在所有任务对比和静息状态下,随着模式数量的增加,重构的准确性也随之增加,仅使用$ N=10 $模式,相关性系数$ r $就已经达到或超过0.38(图\ref{fig:1}d)。
不同的任务也会调动不同的大尺度模式,这表明特定的刺激会激发特定的模式(图\ref{fig:1}e)。
在10个模式之后,重构准确性的改善变得缓慢,大约在$ N=100 $模式时达到$ r \geq 0.80 $,之后的重构准确性只有微小的增加。
因为前100个模式的波长都超过40毫米(补充表1),而包含更短波长的模式只会细化局部模式的重构(图\ref{fig:1}e中的箭头),我们的发现表明,数据主要由具有长空间波长的空间模式组成\footnote{在这里,波长是指空间模式或模式的尺度。在这篇文章中,研究者们发现,脑活动的重构主要依赖于具有较长空间波长(大约40毫米或更大)的模式。这个发现表明,大脑的功能模式具有较大的空间尺度,这与以往对于脑活动主要在局部区域发生的观点形成了鲜明的对比。}(下一节将进行更详细的分析)。


这些结果在所有47个\textit{人类连接组项目}任务对比(补充图\ref{fig:supp_1})和各种分辨率的划分(补充图\ref{fig:supp_2})中都是一致的,但是高分辨率划分的数据需要更多的模式才能达到高重建精度,这是由于粗略划分的低通空间滤波效应。
我们的结果也不受使用群体平均皮层表面模板(而不是个体特定的表面)来推导\textit{几何特征模式}的影响(补充图\ref{fig:supp_3}-\ref{fig:supp_5}和补充信息\ref{sec:individual_specific})。
总的来说,这些发现表明皮层\textit{几何特征模式}形成了一个紧凑的表示,捕获了任务引发和自发皮层活动的各种方面。
此外,他们显示出这样的活动主要由长波长、大规模的特征模式主导。


我们接下来测试\textit{几何特征模式}是否比基于图论的连接组近似导出的特征模式提供了更简洁且基础的动态描述。
% geometric
为此,我们将\textit{几何特征模式}的重构精度与三个替代的连接组导出的特征模式基准集进行比较(参见图~\ref{fig:2}a~的示意图)。
% 1. connectome
第一个基准集是根据\textit{磁共振弥散成像}追踪图在顶点分辨率上映射的连接组并经过阈值处理得到的,以获得 0.10\%的连接密度,就像我们之前做过的那样(在方法\ref{sec:connectome_derivation}中连接组特征模式的求导)。
% 2. EDR
第二个基准集是根据均匀的随机接线过程构建的,该过程受到指数级的距离依赖连接概率的控制,以模仿简单的、类似\textit{指数距离规则}的连接性(在方法\ref{sec:derivation}皮层\textit{几何特征模式}推导中)。
% 3. connectome (density matched)
因为实证连接组和\textit{指数距离规则}连接组的连接密度不同,我们评估了一个第三个基准集,该基准集来自经过 1.55\% 阈值处理的实证连接组,以匹配\textit{指数距离规则}连接组的密度。
上述连接组,\textit{指数距离规则}和密度匹配的连接组特征模式是从其各自的连通性矩阵的图拉普拉斯算子(\textit{拉普拉斯-贝尔特拉米算子}的离散对应物)导出的(见图~\ref{fig:2}b~和扩展数据图~\ref{fig:extended_fig_1})。


\begin{figure}[!htb]
	\centering
	\includegraphics[width=0.9\textwidth]{fig/fig_2.pdf}
	\caption{\textbf{几何特征模态与基于连接组的特征模态的对比评估}。
		\textbf{a}, 用于推导皮层几何,连接组和\textit{指数距离规则}连接组的特征模式的解剖属性示意图。
		\textit{几何特征模式}依赖于局部表面网格信息,如相邻表面网格顶点(点)之间的连接(蓝色)和曲率。
		连接组特征模式依赖于网格顶点(蓝色)之间的局部链接,以及从\textit{磁共振弥散成像}实证重构的短程和长程连接(品红色)。
		\textit{指数距离规则}特征模式依赖于从随机接线过程生成的连接(红色),其中顶点之间的连接概率作为它们距离的函数呈指数衰减。
		\textbf{b},连接组和\textit{指数距离规则}特征模式的示例。
		负值,零值和正值分别以蓝色,白色和红色着色。
		尽管存在一些相似之处,这些模式的空间模式与使用皮层几何学导出的模式不同(与图\ref{fig:1}a比较)。
		\textbf{c},由几何、\textit{指数距离规则}和两种变体的连接组特征模式实现的静息态功能耦合矩阵的重构精度:
		一种使用以前的方法\cite{naze2021robustness}定义的连接组,另一种与\textit{指数距离规则}连接组具有相同的连接密度,以便公平比较(对于其他密度,请参见补充图\ref{fig:supp_6}和\ref{fig:supp_7})。
		\textbf{d},由\textit{几何特征模式}和其他基准集实现的所有47个\textit{人类连接组项目}任务对比图的重构精度之间的差异,如每个面板上方的文本所示。
		每一行代表一种不同的任务对比,这里按照广泛的类型进行分组(补充信息~\ref{sec:sup_2_1});
		红色表示\textit{几何特征模式}的性能优越。
		请注意,虽然在与\textit{几何特征模式}相比的重构中,对于包含少于十个模式的连接组特征模式似乎有性能优势,但重构精度通常较低(不同任务的平均$ r = 0.42 $),与100个模式的情况(平均$ r = 0.71 $)相比。
	} \label{fig:2}
\end{figure}


总的来说,\textit{几何特征模式}考虑了皮层表面的内在曲率和表面网格中的局部顶点到顶点的关系;
连接组特征模式并未考虑曲率,但捕捉了网格顶点之间的局部空间关系,以及使用\textit{磁共振弥散成像}测量的短程和长程连接;
而\textit{指数距离规则}特征模式考虑了均匀的,随机的,距离依赖的连接规则的影响,但并未完全捕捉到皮层几何(图~\ref{fig:2}a)。
因此,比较这些不同的基准集使我们能够区分皮层几何与结构连接对大脑动态的贡献。


直接比较这些不同基准集的重构精度显示,\textit{几何特征模式}在自发(图~\ref{fig:2}c)和任务诱发(图~\ref{fig:2}d)数据中始终显示出最高的重构精度。
\textit{指数距离规则}特征模式的表现几乎与\textit{几何特征模式}一样好,而连接组特征模式的精度最低。
无论使用哪种划分(扩展数据图~\ref{fig:extended_fig_2}~和~\ref{fig:extended_fig_3}),用于生成连接组特征模式的特定连接密度(补充图~\ref{fig:extended_fig_6}~和~\ref{fig:extended_fig_7}~以及补充信息~\ref{sec:thresholding_effect}),以及我们是否使用离散的区域划分而不是顶点分辨率生成连接组(补充图\ref{fig:extended_fig_8}和补充信息~\ref{sec:thresholding_effect}),这一发现都成立。
我们还发现,\textit{几何特征模式}显示出比功能数据本身的主成分(通过主成分分析计算;补充图\ref{fig:supp_9},扩展数据图\ref{fig:extended_fig_4}和补充信息~\ref{sec:comparison_eigenmodes_derived})更强的样本外泛化能力,并且比傅立叶空间基准集的表现更好(扩展数据图\ref{fig:extended_fig_5},补充信息~\ref{sec:comparison_fourier}~和方法~\ref{sec:sets_comparisons}~中与统计基集的比较)。


总的来说,这些结果展示了\textit{几何特征模式}作为大脑功能基准集的简洁性、稳健性和普遍性。
它们还支持\textit{神经场理论}的预测,即大脑活动最好用直接从皮层形状导出的特征模式来表示,从而强调了几何形状在约束动态中的基本作用。




% 短波频率高传播不远
\section{长波长主导皮层活动}

使用\textit{几何特征模式}对自发和任务诱发的数据进行重构显示,大脑活动的空间组织主要由空间波长约 40 毫米或更长的模式主导(图~\ref{fig:1}d-f)。
这个结果反驳了神经影像数据分析的经典方法,其中通过阈值化统计图来映射刺激诱发的激活,以识别高度活动的局部、孤立的区域。
这种经典方法基于这样的假设,即焦点区域代表了刺激可能引发的离散的大脑区域,而其他区域的次阈值活动是可以忽略的。
任务激活数据中令人惊讶的长波长内容(图~\ref{fig:1}d-e)表明,经典程序只关注冰山一角,并掩盖了任务诱发的底层空间延展和结构化的活动模式(参见扩展数据图\ref{fig:extended_fig_6}以获取涉及原因的解释)。
这些观察符合\textit{神经场理论}的预测以及先前对任务诱发的脑电图信号的分析\footnote{"冰山"在这里是个比喻,意指经典神经影像数据分析方法只揭示了大脑活动的一小部分(即"冰山的尖端"),而忽略了由任务诱发的底层空间延展和结构化的活动模式,这些可能类似于隐藏在水面下的冰山的大部分。}。


为此,我们分析了使用\textit{人类连接组项目}中的 47 个任务对比中的群体平均未阈值化激活图的几何模式分解获得的空间功率谱(方法\ref{sec:modal_power}中的任务诱发激活图的模态功率谱)。
作为独立的复制,我们还分析了 NeuroVault 存储库中1,178个独立实验的 10,000 个未阈值化激活图,从而提供了在人脑中映射的刺激诱发激活模式多样性的全面绘图。


% Despite the wide
尽管获取这些激活图使用了广泛的刺激、范例和数据处理方法,但我们观察到图中的大部分功率集中在前 50 个模式中,这些模式对应的空间波长大于约 60 毫米(图\ref{fig:3}a;对\textit{人类连接组项目}关键任务对比图的每个单独结果也类似;扩展数据图\ref{fig:extended_fig_7})。
使用替代数据,我们确认这些发现不能由典型的 fMRI 处理流程引发的空间平滑解释,该流程可以过滤出活动的短波长空间模式(扩展数据图~\ref{fig:extended_fig_8}~和补充信息~\ref{sec:modal_power_spectra})。
我们进一步观察到,递增地、顺序地移除长波长模式对重构精度的影响比移除短波长模式的影响要大得多(图~\ref{fig:3}b~和方法~\ref{sec:wavelength_contributions}~中的长波模式和短波模式的贡献)。
例如,在 7 个关键的\textit{人类连接组项目}任务对比中,移除前25\%的长波长模式(模式1-50)导致重构精度下降约 40-60\%,而移除前 25\% 的短波长模式(模式151-200)仅导致下降约 2-4\%(图~\ref{fig:3}b,插图)。
这些结果表明,在 fMRI 可接触的时间和空间尺度上,诱发的皮层活动包括大尺度,几乎是大脑全宽的空间模式,挑战了这样的经典观点,即这种活动应该以离散的、孤立的和解剖学上定位的激活簇来描述。


\begin{figure}[!htb]
	\centering
	\includegraphics[width=0.9\textwidth]{fig/fig_3.pdf}
	\caption{\textbf{任务引发的活动激发了长波长模式}。
	\textbf{a},47 个\textit{人类连接组项目}任务对比图的归一化平均功率谱(左)和 NeuroVault 数据库中的 10,000 个对比图的功率谱(右)。
	插图展示了与前 50、100和200 个模式相关的空间尺度的皮层表面重构,这些模式对应的空间波长分别约为 60、40和30毫米。
	七个关键\textit{人类连接组项目}任务对比的对比特定谱在扩展数据图\ref{fig:extended_fig_7}中展示。 
	\textbf{b},作为去除重构过程中的模式(在200个模式中)百分比函数的 7 个关键\textit{人类连接组项目}任务对比图的重构精度。
	实线和虚线分别对应去除顶部长波长和短波长模式。
	插图展示了群体平均的实证激活图(数据)及其去除25\%模式后的重构。
	负值、零值和正值分别用蓝色、白色和红色着色。
	} \label{fig:3}
\end{figure}



\section{波动力学连接了几何和功能}

皮层的\textit{几何特征模式}是通过解\textit{拉普拉斯-贝尔特拉米算子}特征值问题获得,该问题也被称为\href{https://baike.baidu.com/item/%E4%BA%A5%E5%A7%86%E9%9C%8D%E5%85%B9%E6%96%B9%E7%A8%8B}{\textit{亥姆霍兹方程}}~\ref{eq:1}。
在物理连续系统中,\textit{亥姆霍兹方程}的解对应于更一般的波动方程解的空间投影,使得结果的特征模式本质上代表系统动力学的振动模式,或者称为\href{https://baike.baidu.com/item/%E9%A9%BB%E6%B3%A2}{驻波}。
这个等价性意味着,正如\textit{神经场理论}所预测的那样,\textit{几何特征模式}在重构脑活动不同模式中的卓越性能,来源于波动动力学在塑造这些模式中的基本作用。
这个预测已经通过脑电图记录的模型得到了确认,但在 fMRI 信号中观察到的全脑波动现象仅在最近被观察到,并且到目前为止还缺乏理论解释。
在这里,我们使用\textit{神经场理论}和\textit{几何特征模式}来展示的波动动力学可以提供一个统一的描述,这个描述可以解释在 fMRI 可以得到的尺度上观察到各种实证和生理现象。


% We model neural
我们使用一个等向性阻尼的\textit{神经场理论}波动方程模型来模拟神经活动,该模型没有再生机制\cite{robinson1997propagation}(图~\ref{fig:4}a~和方法~\ref{sec:NFT_model}~中的\textit{神经场理论}的波动力学模型)。
在这个模型下,活动通过白质连接在新皮层的各个点之间传播,其强度随距离呈近似指数衰减(补充图\ref{fig:extended_fig_10}和补充信息\ref{sec:NFT}和\ref{sec:NFT_wave})。
为了模拟静息状态下的神经活动,我们使用白噪声输入来模拟无结构的随机波动(方法\ref{sec:modelling_resting}中的静息态动力学建模)。
我们比较了这个简单的波动力学模型和一个基于生物物理的神经群模型(兴奋抑制平衡模型, Balanced Excitation–Inhibition, BEI模型)的性能,这个神经群体模型已被广泛用于理解静息状态 fMRI 信号(图~\ref{fig:4}a~和方法~\ref{sec:neural_mass}~中的神经群模型)。
神经群模型紧密地与经典的、以连接组为中心的脑功能观念相一致,将动态过程视为由于在离散解剖区域的神经群体间的相互作用的结果,这些神经群体根据实证测量的连接组进行耦合。


% 对于一个波,一个周期总得是完整的,不存在是零碎、被切断的周期
\begin{figure}[!htb]
	\centering
	\includegraphics[width=0.9\textwidth]{fig/fig_4.pdf}
	\caption{\textbf{波动动力学塑造自发和刺激诱发活动的模式}。
	\textbf{a}, 对于波动力学模型,位置 $ r $ 和时间 $ t $ 的活动 $ \phi(r,t) $ 由具有阻尼率 $ \gamma_s $、空间长度尺度 $ r_s $ 和输入 $ Q $ 的波动方程控制。
	对于神经群模型,区域$ i $的活动$ S_i(t) $由函数$ f $描述,该函数依赖于其他区域$ S $的活动、局部群体参数$ \theta_i $和由全局耦合参数$ G $缩放的连接组$ C $。
	模型动态被用来计算一个模拟的功能耦合矩阵(方法)。
	$ M_{fixed} $ 和 $ M_{free} $ 分别对应每个模型的固定参数和自由参数的数量。
	\textbf{b}, 基于各种指标比较数据和模型模拟,从左到右分别是:功能耦合矩阵(为了可视化)、边功能耦合、节点功能耦合和功能耦合的动力学属性(FCD, dynamic properties of FC)。
	对于边功能耦合和节点功能耦合,红线代表皮尔逊相关系数 $ r $ 的线性拟合;
	对于功能耦合动力学属性,使用\textit{\href{https://www.cnblogs.com/jiangkejie/p/11572205.html}{柯尔莫可洛夫-斯米洛夫}}(Kolmogorov-Smirnov, KS)统计量比较数据和模型动态中区域间同步性相似性的概率密度函数(probability density function, pdf)。
	\textbf{c}, 经过 1ms 的初级视觉皮层刺激后 1-2ms 的活动波动传播。
	箭头指示传播的方向(补充视频1)。
	\textbf{d}, 视觉皮层层次中17个区域的活动剖面。
	插图显示了根据其活动剖面颜色的皮层表面区域的空间位置。
	\textbf{e}, 按排名活动剖面达到峰值的时间与 \textbf{d} 中区域的T1w:T2w值的关系。
	红线表示排名变量的线性拟合,带有\textit{斯皮尔曼相关系数} $ r $和单侧Pspin,估计自10,000次排列。
	} \label{fig:4}
\end{figure}


我们首先比较这两种模型在捕捉自发的、无任务的功能耦合的不同且常被研究的属性的效果:即,静态两两耦合(边耦合),静态节点级平均功能耦合(节点功能耦合)和功能耦合的时间解析动力学属性(方法\ref{sec:modelling_resting}中的静息态动力学建模)。
在所有基于功能耦合的基准测量中,与神经群模型相比,波动模型在重建实验数据方面表现出相当或更优的性能(图\ref{fig:4}b)。
波动力学模型也比群模型更准确地捕捉到了实验静息态活动的时滞属性(扩展数据图\ref{fig:extended_fig_9}和方法\ref{sec:dynamics_measurement}中的测量静息态动态的时滞属性)。
波动模型的这种强大性能是非常引人注目的,考虑到其相对简单:波动模型只需要皮层的几何形状(即,表面网格)作为输入,并包括一个固定参数和一个用于拟合数据的自由参数(rs)(扩展数据图\ref{fig:extended_fig_10}),而神经群模型需要一个由\textit{磁共振弥散成像}导出的区域间解剖连通矩阵,并包括15个固定参数和四个自由参数(补充信息\ref{sec:mass_optimization})。
这些考虑表明,波动动力学提供了一种更准确、更简洁的机械解释,能捕捉到由fMRI捕获的宏观尺度的自发皮层动态。


接下来,我们在波动模型中考虑刺激引发的皮层活动。我们分析了初级视觉皮层的感觉刺激引发的皮层反应,因为它引发了一个明确定义的区域皮层反应层次结构(方法\ref{sec:modelling_stimulus}中的建模刺激诱发动力学)。
向初级视觉皮层输入1毫秒的脉冲产生了一个传播的活动波,该波迅速沿着背侧和腹侧视觉处理流程分离(图4\ref{fig:4}(箭头)和补充视频1),这与层次化视觉处理的主流理解一致。
值得注意的是,这个结果表明,对引发活动的传播波的几何约束足以使背侧和腹侧的处理流程分离,这些流程传统上被认为主要是由复杂的层特异性连通性模式驱动的。
此外,视觉系统各处的引发反应的时间轮廓遵循一个明确定义的时间尺度层次,与低阶视觉区相比,高阶联合区显示的峰值反应被延迟并延长(图\ref{fig:4}d)。
因此,这些发现表明,这种层次排序,以前在实验和建模研究中已经确定,自然地从通过皮层介质传播的激发波中出现。
关键的是,这种区域反应的层次化时间排序与基于非侵入性的髓鞘结构估计(T1加权(T1w)和T2加权(T2w)比率)的皮层处理层次的独立解剖学测量强烈相关。
这种相关性在视觉处理层次结构中尤其强烈($ r = -0.72 $,单侧旋转测试$ P $值(Pspin)= 0.003;图4e),但在考虑所有皮层区域时也存在($ r = -0.44 $,$ Pspin = 0.037 $;补充图\ref{fig:supp_11})。
总的来说,我们的建模结果表明,如何在皮层的几何形状上展开的简单波动动态提供了一个统一的生成机制,用于捕捉脑活动的空间时间复杂性质。


\section{几何形状约束了皮层下活动}

我们迄今的分析主要关注了新皮层中几何与动力学的强烈耦合。
接下来,我们将这种耦合研究扩展到非新皮层区域,重点研究丘脑、纹状体和海马,因为这些结构的几何形状可以很容易地使用MRI数据捕捉到,而且它们的功能组织已经被广泛研究过。


我们首先将我们的特征模式分析推广到三维体积(在方法中估计非新皮层结构的\textit{几何特征模式}),得到的\textit{几何特征模式}在每个结构的三个空间维度中空间扩展。
接下来,为了全面捕捉这些非新皮层区域的宏观尺度功能组织,我们将一种广泛使用的流形学习过程应用于体素级功能耦合数据,以获取每个结构中的关键功能梯度(在方法中绘制非新皮层结构的功能组织)。
这些功能梯度描述了由功能耦合相似性决定的空间组织的主要轴,从而代表了功能组织的主要变化模式,按照它们解释的功能耦合相似性方差的百分比排序。


皮层下结构(如丘脑、纹状体和海马)的前三个功能梯度的空间特征(分别解释了功能耦合相似性方差的24\%,50\%和47\%)与第一至第三个\textit{几何特征模式}近乎完美地匹配(图\ref{fig:5}a-c;空间相关性$ r\geq0.93 $)。
这种紧密的关联性推广到每个结构的前20个模式和前20个梯度(前20个梯度分别解释了功能耦合相似性总方差的49\%,70\%和68\%),所有的绝对空间相关性 $ |r| > 0.5 $,除了纹状体和海马的第20个梯度和第20个模式(图\ref{fig:5}d-f)。
这种强烈的关系令人震惊,因为功能梯度是通过对fMRI派生的功能耦合测量应用复杂的处理流程生成的,而特征模式仅从每个结构的几何形状派生,独立于功能数据。
这些发现表明,非新皮层结构的功能组织直接源于其\textit{几何特征模式}。


\begin{figure}[!htb]
	\centering
	\includegraphics[width=0.9\textwidth]{fig/fig_5.pdf}
	\caption{\textbf{几何形状塑造了非新皮层的功能}。
		\textbf{a-c},丘脑(a)、纹状体(b)和海马(c)的前三个\textit{几何特征模式}和基于功能耦合的功能梯度。
		这些模式和梯度以三维坐标空间显示,负值、零和正值分别用蓝色、白色和红色表示。D,背侧;P,后侧;R,向右。散点图显示了模式和梯度之间的关系,红线代表具有皮尔逊相关系数$ r $的线性拟合。
		\textbf{d-f},底部,丘脑(d)、纹状体(e)和海马(f)的前20个\textit{几何特征模式}和功能梯度的绝对相关性($ |r| $)。
		顶部,每个功能梯度获得的最高$ |r| $值(灰色条),考虑到\textit{几何特征模式}的顺序翻转,以及每个功能梯度解释的方差百分比(蓝线)。Max,最大。
	} \label{fig:5}
\end{figure}



\section{讨论}

许多物理系统的动态都受到其几何形状的限制,并可以理解为一小部分结构模式的激发。
我们在这里展示,仅从大脑的几何形状导出的结构特征模式,提供了一种比其他基于连通图模型更紧凑、更准确、更简约的表示宏观级别活动的方式。
这种基于模式的大脑视角进一步表明,fMRI捕获的自发性和诱发性大脑活动都是由相对较长波长的大规模特征模式主导的,这些动态是从生物物理动力学波动方程推导出来的。
这些发现挑战了传统的神经科学范式,即认为复杂的区域间连通性模式是动态的关键解剖学基础,这些连通性模式在离散的、专业化的神经元群体之间形成。
相反,我们的结果表明,一种将大脑视为连续的、空间嵌入的系统的物理方法,提供了一种理解结构对宏观神经功能各个方面的约束的统一框架。


广泛的比较显示,\textit{几何特征模式}与其他解剖(连通性和\textit{指数距离规则}特征模式)和统计(主成分分析和傅里叶)基础集的表现相比,其在捕获大脑皮层宏观活动方面的优越性能并非由基础集扩展的通用数学属性所驱动的。
相反,这个结果表明几何学代表了对动态的基本解剖约束。
另外,从合成网络派生出的\textit{指数距离规则}特征模式的强大表现表明,均匀的、距离依赖的连接性和近指数形式代表了另一个对活动的重要解剖约束。
\textit{指数距离规则}型连通性在等式\ref{eq:1}的\textit{亥姆霍兹方程}中数学嵌入(附加信息\ref{sec:NFT_wave}),所以这种连通性的作用被\textit{几何特征模式}隐含地捕获了。


相较之下,连通性特征模式的性能较差,这表明超越简单的\textit{指数距离规则}形式的拓扑复杂连接,在获取能准确解释用fMRI测量的大脑皮层活动的时空模式的特征模式方面,提供的额外效益极小。
因此,我们的发现对强调复杂解剖连接模式作为协调动态的主要驱动力的传统观点提出了质疑。
实际上,最近的研究表明,长距离的大脑皮层连接相对罕见——它们可能仅对\textit{指数距离规则}型连通性主导的效应产生较小的干扰。
然而,这些连接的拓扑中心性、代谢成本和严格的遗传控制暗示,它们可能在波动动态之外提供重要的功能和进化优势(附加信息\ref{sec:lag_threads})。
\textit{磁共振弥散成像}和fMRI数据的有限分辨率和对预处理流程的敏感性,使得全面揭示这些连接的功能角色变得复杂。
在这方面,高质量的动物追踪和电生理数据可能有助于我们深入理解这一问题。


几何与动态之间的紧密耦合在新皮层结构和非新皮层结构中都很明显,这表明新皮层以外的区域的功能组织也主要由距离依赖的解剖连通性和波动动态主导,正如最近的实验所发现的。
这些观察表明,相比于目前文献中使用的复杂的流形学习过程,\textit{几何特征模式}提供了对非新皮层结构中潜在的功能组织梯度的更简单、更节约和机械性更强的解释。
这是因为这些过程是现象学的,它们提供数据中主要方差来源的统计描述,而结构特征模式的研究则源于一个生成过程。


几何模式分解为研究大脑激活图的空间属性提供了独特的视角。
传统的大脑映射分析主要关注在空间位置孤立集群中超过统计阈值的反应。
相比之下,我们的方法与物理和工程中严谨确立的结果相一致,即空间连续系统的微扰会引发系统范围内的反应,就像小提琴弦的音乐音符是由其全长振动产生的,而不是由受限段的行为产生的。
因此,使用\textit{几何特征模式}表明,在来自基于任务的fMRI研究的超过10,000个多样化绘图中,任务参与主要与激发大约60毫米和更长波长的模式相关。
这个结果与在实验性脑电图和诱发反应电位数据中观察到的长波长激发的观察相符,并且表明,依赖于点统计图的阈值的传统分析掩盖了任务实际引发的空间扩展和复杂的活动模式。


我们的建模结果为理解几何与功能之间紧密联系背后的物理过程提供了洞见。特别是,波动模型的相对简单性和在捕获自发fMRI动态的多样性方面的优越性能,表明该模型比将大脑视为通过连通图(边)连接的离散解剖区域(节点)的图形的复杂神经群模型提供了更节约的解释。这个发现与在人类和动物fMRI数据中观察到的波动动态的实验观察一致。未来的工作可以探索是否将空间异质性或复杂结构输入引入波动模型,进一步提高其解释各种实证现象的准确性。

将波动模型应用于模拟视觉刺激显示,从刺激点传播的波动沿着经典的背部和腹部视觉路径分离,而区域对扰动的反应遵循一个广为描述的时间尺度层次,范围从快速反应的单模态区域到较慢的跨模态区域。
这些视觉处理层次结构的典型特性已经被广泛研究了几十年,通常认为是由复杂的层特异性区域间连接模式驱动的,但我们的分析表明,通过皮层几何形状传播的波动足以产生分离的、层次化的处理流。
因此,虽然我们的发现不能排除复杂的区域间连接在这些宏观动态中的作用,但它们确实表明,这种连接并不是这些宏观动态出现的必要条件。


\textit{几何特征模式}的优越性能提供了即时的实际利益,因为这些模式可以仅使用感兴趣结构的网格表示来估计,这可以很容易地使用针对T1w解剖图像的成熟、自动化处理管道来推导。
相比之下,连通性特征模式需要一个通过复杂数据处理管道应用于T1w和\textit{磁共振弥散成像}图像生成的宏观区域间连接的基于图的模型;
需要定义图节点,这是一个有争议的话题;
以及需要应用一个阈值处理程序来去除可能是虚假连接,我们自己的分析显示这可能会影响到发现(补充图\ref{fig:supp_6}和\ref{fig:supp_7})。
不需要做这些选择就可以得到\textit{几何特征模式}的事实意味着它们可以在人类和其他物种的不同实验环境中稳健和灵活地应用,开辟了新的研究途径。
例如,可以研究\textit{几何特征模式}如何通过神经发展变化或在临床疾病中被破坏。
实际上,我们确定的几何与功能之间的紧密关系暗示,空间和时间动态的物种间差异可能主要由大脑形状的差异驱动。
描述大脑几何形状的变化,无论是在物种内还是在物种间,如何塑造大脑功能将对理解神经活动的物理和解剖约束至关重要。



\section{方法} \label{sec:method}

\subsection{皮层几何特征模式推导} \label{sec:derivation}

% If brain structure
如果大脑结构可以近似为时间常数,则可以通过本征模分解\cite{nowack1995neocortical,robinson2016eigenmodes}分别处理所产生的空间和时间动力学,类似于其他物理系统的处理\cite{melrose1991electromagnetic}。 
特别是,空间方面满足拉普拉斯特征值问题,也称为\textit{亥姆霍兹方程},在方程\ref{eq:1}中定义。

对于大脑皮层,我们将其视为嵌入三维欧几里德空间的二维模型,方程\ref{eq:1}中的\textit{拉普拉斯-贝尔特拉米算子}捕获内在几何形状,其中包括皮层表面的曲率\cite{wachinger2015brainprint},通常定义为\cite{chavel1984eigenvalues,seo2011laplace}:


\begin{equation}\label{key}
	\Delta:=\frac{1}{W} \sum_{i,j} \frac{\partial}{\partial x_i} (g^{ij} W \frac{\partial}{\partial x_j})
\end{equation}

其中 $ x_i $,$ x_j $ 是局部坐标,$ g^{ij} $ 是内积度量张量 $ g^{ij}:=<\frac{\partial}{\partial x_i}, \frac{\partial}{\partial y_i}> $, $ W:=\sqrt{det(G)} $,det 表示行列式,$ G:=(g_{ij}) $。


我们使用安装在大型高性能计算设施\cite{goscinski2014multi}中的LaPy python库\cite{wachinger2015brainprint,reuter2006laplace}来推导人类皮层的几何本征模式。 
具体来说,我们使用了中层人类皮质表面的三角形表面网格表示,每个半球包含 32,492 个顶点,从 FreeSurfer 的 fsaverage 总体平均模板的下采样、\href{https://github.com/ ThomasYeoLab/CBIG/tree/master/data/templates/surface/fs_LR_32k}{左右对称版本}获得\cite{fischl1999high} ()。
该模板独立于我们所有分析中使用的数据样本,从而消除了对循环性的任何担忧。


请注意,连续\textit{拉普拉斯-贝尔特拉米算子}在表面的底层黎曼流形上运行,而不是直接在网格顶点上运行。
LaPy 在表面网格上使用三次有限元方法,在插值光滑流形上实现方程\ref{eq:1}的数值可处理解。
这与离散图 Laplacian \cite{shuman2013emerging} 不同,后者不编码点之间的空间关系。
我们所有的分析都集中在单半球本征模态,但我们的方法可以很容易地扩展到整个大脑,因为双半球本征模态可以表示为来自每个半球的本征模态的对称或反对称组合。
对称组合对应于矢状中平面的镜像对称,不对称组合对应于半球具有相同空间结构但符号翻转的情况。


方程\ref{eq:1}的特征值解根据每个特征模的空间图案的空间频率或波长顺序排序,即$ 0 \leq \lambda_1 \leq \lambda_2 \leq ... $。
请注意,第一个特征值 $ \lambda_1 $ 约等于零(波长远大于大脑大小),相应的特征模 $ \psi_1 $ 是一个没有节点线的常数函数(函数的零集)。
在我们的整个研究中,我们在分析中使用了前 200 个模式(包括恒定模式 $ \psi_1 $),因为使用越来越多的模式时观察到重建精度的改善逐渐减弱(图\ref{fig:1}d)。


% Each eigenmode
每个特征模式包括具有特定空间波长的空间图案。
根据参考文献\cite{robinson2016eigenmodes},我们使用理想化的球形情况来近似特征模式波长,因为它在拓扑上与人类皮层相当。
通过在球体上求解方程\ref{eq:1},存在退化解,使得某些特征模式具有相同的特征值和空间波长——这类似于量子物理学中的球谐函数。
事实上,由于特征模式将在皮层折叠消失的极限下接近球谐函数,因此前者可以组合成具有空间波长的本征群,

\begin{equation}
	\text{wavelength} = \frac{2 \pi R_s}{\sqrt{l(l+1)}},
\end{equation}

其中$R_s$是球体的半径(对于本研究中使用的fsaverage总体平均模板,$R_s \approx 67.0 mm$),$l$是本征群数(原子物理学中的角动量量子数)。
前15个本征群的波长和本征群中包括的本征模如补充表1所示。


\subsection{大脑活动的模式分解} \label{sec:modal_decomposition}

我们使用\textit{几何特征模式}来分解每个个体在空间位置 $ r $ 和时间 $ t $ 测量的时空 fMRI 数据,作为模式的加权和:

\begin{equation}\label{eq:weighted_sum}
	y(r, t) = \sum_{j=1}^{N} a_j(t) \psi_j(r),
\end{equation}

其中 $ a_j  $ 是解释数据时模态 $ j $ 的振幅,$ \psi_j $ 是第 $ j $ 个模态,N 是使用的模态数量;
我们使用 $ N = 200 $ 进行分析。
对于时空数据,即来自无任务 fMRI 的自发动力学记录,数据的每个时间帧都被代入方程 \ref{eq:weighted_sum},从而得出每个模式 $ \psi_j $ 的时间相关幅度 $ a_j(t) $。
对于纯空间数据(即任务诱发的激活图),幅度与时间无关,因此 $ a_j(j) \rightarrow a_j $。
在这两种情况下,幅度都可以通过在皮层表面积分来获得,

\begin{equation}\label{eq:amplitudes}
	a_j(t) = \int y(\boldsymbol{r}, t) \psi_j(\boldsymbol{r}) d\boldsymbol{r}, 
\end{equation}

可以使用本征模态的正交特性从方程\ref{eq:weighted_sum}导出\cite{robinson2021determination,courant2008methods}。
如果没有足够的测量来评估积分,也可以通过统计通用线性模型来估计幅度。


获得幅度后,使用方程\ref{eq:weighted_sum}计算重建数据。
我们通过计算经验数据和重建数据之间的相关性来量化这种重建的准确性。
对于时空无任务数据,我们首先按照该领域的标准做法,通过取离散块/区域内数据的平均值来对经验数据和重建数据进行划分\cite{eickhoff2018imaging},然后通过计算皮尔逊相关系数构建区域间功能耦合矩阵块时间序列对。
对于任务诱发的数据,我们在激活图上应用相同的分区以允许直接比较。
最后,通过采用经验功能耦合和重构功能耦合的上三角元素的相关性来计算无任务数据的重构精度。
对于任务诱发的数据,重建精度是根据分割的经验图和重建图的空间相关性计算的。
然后我们计算所有参与者的平均重建精度。



\subsection{人类连接组项目数据} \label{sec:HCP_data}

我们使用来自\textit{人类连接组项目}\cite{van2013wu}的预处理 fMRI 数据。
我们没有执行任何额外的预处理步骤,例如全局信号去除,因为第一个特征模式(被视为全局恒定模式)已经明确捕获数据中的全局偏差,从而允许其他模式捕获功能相关的非全局活动。
我们分析了 255 名不相关的健康个体(年龄 22-35 岁,132 名女性和 123 名男性)的数据,这是最大的\textit{人类连接组项目}样本(不包括双胞胎或兄弟姐妹),并且所有参与者都完成了任务诱发和无任务静息状态数据 。
所有参与者都是志愿者并提供知情同意书。 开放获取数据由 WU-Minn HCP 联盟获得,并获得当地监督伦理委员会的批准,并根据\textit{人类连接组项目}的数据使用条款与作者共享。
我们的所有程序都是按照这些数据使用条款规定的协议进行的。 有关图像采集协议、预处理流程和道德监督的详细说明,请参阅参考文献\cite{van2013wu,barch2013function}。


在\textit{人类连接组项目}数据集中,我们分析了在七个任务域中测量的任务诱发 fMRI(补充表 2)和无任务静息态 fMRI,这些数据已经经过\textit{人类连接组项目}预处理。
两个数据集都映射到 fsLR-32k CIFTI 空间,每个半球有 32,492 个顶点。
有关数据和预处理的更多详细信息,请参阅补充信息 2.1 和 2.2。
我们还分析了源自\textit{磁共振弥散成像}数据的单个连接组,如参考文献中提供的\cite{tian2021high}。
连接组代表每个半球大小为 32,492 × 32,492 的高分辨率加权矩阵。
有关数据和连接组构建的更多详细信息,请参阅补充信息 2.3。


\subsection{皮层分割} \label{sec:cortical_parcellations}

该研究的主要结果分析了任务诱发的激活图和分割成离散区域的无任务功能耦合数据。
我们使用每个半球 180 个区域的 HCP-MMP1 分区(我们称之为 Glasser360 分区)展示了结果,它反映了基于皮质结构、功能、连通性和地形组合的清晰区域边界\cite{glasser2016multi}。
分区位于\textit{人类连接组项目}提供的 fsLR-32k 空间上。
为了测试我们结果的稳健性(补充图\ref{fig:supp_2}和扩展数据图\ref{fig:extended_fig_2}和\ref{fig:extended_fig_3}),我们还使用 Schaefer 等人 82 在不同分辨率(100、200、400)的 fsLR-32k 空间上提供的分区进行了分析。
600、800 和 1,000 个地块,横跨两个半球;我们分别将这些地块称为 Schaefer100、Schaefer200、Schaefer400、Schaefer600、Schaefer800 和 Schaefer1000 块)。


\subsection{连接组特征模式的求导} \label{sec:connectome_derivation}

连接组特征模式是根据以前的方法\cite{naze2021robustness}导出的,以便能够与以前的发现进行比较。
请注意,在之前的研究中\cite{atasoy2016human,preti2019decoupling,naze2021robustness,rue2021connectome},连接组本征模式被称为连接组谐波,但我们在这里使用术语本征模式,因为术语谐波意味着整数频率比,这对于脑源模式来说不一定得到保证。


我们获得了使用\textit{磁共振弥散成像}纤维束成像测量的高分辨率连接图,如参考文献\cite{tian2021high}中所述。
其中皮层表面网格中 32,492 个顶点中每个顶点的连通性是通过跟踪每个点的流线直到它们终止于其他点来估计的。
顶点(被视为节点)之间的连接权重被估计为互连流线的数量,而无需标准化\cite{tournier2019mrtrix3}。
对来自\textit{人类连接组项目}的个体进行了纤维束成像(有关数据和纤维束成像方法的更多详细信息,请参阅补充信息 2.3)。 
根据纤维束成像数据,我们将大小为 32,492 × 32,492 的个体加权连接矩阵组合起来,生成组平均连接组 Wconnectome,其权重代表流线的平均数量。
然后,我们生成一个二元邻接矩阵 Alocal,它捕获皮质表面网格模型中各点之间局部空间关系的离散表示,该模型是通过连接网格中直接相邻的两个顶点而构建的。
这些链接旨在捕获传统\textit{磁共振弥散成像}纤维束成像无法解析的本地极短距离连接。


以参考文献\cite{naze2021robustness} 所示,对组平均加权连接组 Wconnectome 进行阈值处理以删除最小权重,使得连接数量比 Alocal 大四倍。
将所得阈值矩阵二值化以获得组邻接矩阵 Aconnectome。 
最后,我们生成了一个合并的邻接矩阵 AC = Alocal Aconnectome(矩阵大小为 32,492 × 32,492;|| 是逻辑 OR 运算符),它捕获了本地顶点到顶点的连接以及测量的复杂的短距离和长距离连接 凭经验。
请注意,由上述阈值处理产生的邻接矩阵 AC 的连接密度为 0.10\%。




通过求解特征值问题获得连接组特征模态,


\begin{equation}\label{eq:connectome_eigenmodes}
	L' \psi = - \lambda \psi,
\end{equation}

其中 $ L' $ 是归一化图拉普拉斯算子,是\textit{拉普拉斯-贝尔特拉米算子}的离散对应物。
归一化图拉普拉斯与非归一化图拉普拉斯 $ L $ 相关,因为 $ L = D^{−1/2}LD^{−1/2} $,其中 $ L $ 根据之前的工作\cite{levy2006laplace}定义:

\begin{equation}\label{eq:unnormalized_Laplacian}
	L = \frac{1}{2} [ (D-A_C) + (D-A_C)^T ],
\end{equation}

其中$ D $是对角矩阵,上标$ T $表示矩阵转置。
与几何特征模一样,方程\ref{eq:connectome_eigenmodes}的特征值解形成序列$ 0 \leq \lambda_1 \leq \lambda_2 \leq ... $。
还要注意,使用高分辨率、顶点级连接组会导致连接组本征模式跨越维度(模式数量)等于顶点数量的空间,从而可以与\textit{几何特征模式}进行公平比较。



\subsection{指数距离规则特征模式的求导} \label{sec:EDR_derivation}

\textit{指数距离规则}特征模式也通过求解方程\ref{eq:connectome_eigenmodes}得到。
然而,在这种情况下,非归一化图拉普拉斯 $ L $ 是使用综合构建的 $ 32,492 \times 32,492  $邻接矩阵 $ A_E $ 定义的,该矩阵遵循随机\textit{指数距离规则}。
为了构建 $ A_E $,我们使用了上一节中的组平均、无阈值加权连接组 $ W_{connectome} $。
然后,我们通过 $ e^{-\alpha d} $ 形式的指数函数将权重的变化拟合为皮层表面顶点之间欧几里得距离 $ d $ 的函数,其中 $ \alpha $ 是尺度指数参数。
使用非线性最小二乘法在 MATLAB 2019b 中进行拟合,得到的最佳经验参数值 $ \alpha_{empirical} = 0.12 $ ,与之前基于连接概率与距离函数的估计一致\cite{theodoni2022structural}。
然后,我们按照随机\textit{指数距离规则}连接过程生成一个随机的二元邻接矩阵,其中两个顶点以概率 $ e^{-\alpha d} $ 连接,且 $ \alpha_{empirical} = 0.12 $。


由\textit{指数距离规则}模型实例生成的邻接矩阵 $ A_E $ 的连接密度为 1.55\%,大大高于 $ A_C $ 的 0.10\% 密度。
因此,我们构建了另一个版本的连接组本征模式,对组平均经验连接组进行阈值处理,以实现密度与 $ A_E $ 匹配的最终 $ A_C $,以允许公平比较(我们将这种密度匹配的连接组特征模式称为密度匹配的连接组特征模式)。
补充图\ref{fig:supp_6} 和图 \ref{fig:supp_7} 更全面地评估了作为网络连接密度函数的连接组特征模式的性能。


\subsection{与统计基集的比较} \label{sec:sets_comparisons}

几何、连接组和\textit{指数距离规则}特征模式基础集均源自生成模型,该模型解释了大脑功能如何从解剖学中产生。
这种方法与文献中常用的统计基础集形成对比,后者可以有效地总结数据,但无法深入了解潜在的生成过程。
我们根据两个统计基组评估了\textit{几何特征模式}的性能,一个统计基组源自函数数据本身的主成分分析,另一个基于傅里叶空间基组。
这些分析的更多细节分别在补充信息\ref{sec:comparison_eigenmodes_derived} 和 \ref{sec:comparison_fourier} 中提供。

\subsection{任务诱发激活图的模态功率谱} \label{sec:modal_power}

为了研究任务诱发激活图的频谱内容,我们使用方程\ref{eq:weighted_sum}中的模式分解并取振幅 $ a $ 的绝对平方来计算它们的模态功率谱。
这类似于通过傅里叶分析计算时间功率谱密度。
然后我们将模式 $ j $ 下的功率相对于所有模式下的总功率进行归一化,这样:

\begin{equation}\label{eq:normalized_power}
	P_j = \frac{|a_j|^2}{\sum_{j=1}^{N} |a_j|^2 }.
\end{equation}

我们计算了两组任务诱发数据的模态功率谱。
第一组包括来自人脸连接组计划任务诱发数据的无阈值激活图(补充信息 \ref{sec:sup_2_1})。
在进行谱分析之前,我们取了 255 个个体的激活图的组平均值,然后分析了每个任务对比的组平均激活图的功率谱。
图 \ref{fig:3}a 和扩展数据图 \ref{fig:extended_fig_8}a 中的结果显示了 47 个人类连接组计划任务对比图的平均功率谱。 
扩展数据图 \ref{fig:extended_fig_7} 显示了七个关键人类连接组任务对比的对比特定功率谱。


第二组包含来自 \href{https://neuroVault.org/}{NeuroVault 存储库}\cite{gorgolewski2015neurovault} 的 1,178 个独立实验的 10,000 个激活图。
我们使用 python 模块 Nilearn86 从 NeuroVault 检索激活图,这些激活图没有阈值,并且带有 fMRI-BOLD 模态标签。
我们使用 Nilearn 将激活图从体积投影到 fsLR-32k CIFTI 空间以匹配 HCP 数据(通过函数 $ nilearn.surface.vol_to_surf $)。
然后我们分析了每个激活图的功率谱。
图 \ref{fig:3}a 和扩展数据图 \ref{fig:extended_fig_8}a 中的结果显示了 10,000 张 NeuroVault 图的平均功率谱。


然后,我们将经验图的功率谱与具有不同平滑级别的替代随机图的功率谱进行比较,以进一步研究长波长模式的相关性。
特别是,我们在体积空间中生成了 10,000 个随机映射,我们以半高全宽范围从 0 到 50 mm 的核大小对其进行平滑,并投影到 fsLR-32k CIFTI 空间上。
然后,我们计算了经验图和替代图之间功率谱的均方对数误差 (mean square logarithmic error, MSLE)。
补充信息 \ref{sec:modal_power_spectra} 中提供了该分析和 均方对数误差测量的更多详细信息。


\subsection{长波模式和短波模式的贡献} \label{sec:wavelength_contributions}

为了了解长波长和短波长几何本征模式在任务诱发激活图重建中的贡献,我们在执行重建过程之前顺序删除模式。
具体来说,我们首先使用 200 个几何本征模式重建七个关键的人类连接组计划对比图,并计算重建精度(即经验图和重建图之间的相关性),作为我们的基线。
然后,我们从长波长模式开始进行增量、连续的模式去除(即去除模式 1、模式 1-2、模式 1-3、模式 1-4、……、模式 1-200)并计算重建每个增量的准确性。
我们重复相同的过程,但从短波长模式开始(即去除模式 200、模式 199-200、模式 198-200、模式 197-200、……、模式 1-200)。


\subsection{神经场理论的波动力学模型} \label{sec:NFT_model}

如上所述,正如\textit{神经场理论}所预测的那样,\textit{几何特征模式}的优越性能意味着神经元活动由波动力学主导。
为了研究波动力学是否可以解释神经元活动的复杂时空模式,我们实现了一个简单的\textit{神经场理论}波动模型,其中动力学由各向同性阻尼波方程描述,无需再生\cite{jirsa1996field,robinson1997propagation},

\begin{equation}\label{eq:NFT_wave_model}
	[\frac{1}{\gamma_s^2 \partial t^2} + \frac{2}{\gamma_s} \frac{\partial}{\partial t} + 1 - r_s^2 \nabla^2] \phi(\boldsymbol{r}, t) = Q(\boldsymbol{r}, t),
\end{equation}

其中 $ \phi(\boldsymbol{r}, t) $ 是位置 $ r $ 和时间 $ t $ 处的神经活动,$ Q $ 是外部输入,$ \gamma_s $ 是阻尼率,$ r_s $ 是波传播的空间长度尺度(概念上与长度为$ \alpha $ 尺度相关) \textit{指数距离规则}特征模式推导中的随机\textit{指数距离规则}连接过程)。
这种形式告诉我们,脉冲输入将产生以 $ \gamma_s $ 速率消散并以 $ \gamma_s r_s $ 速度传播的活动。
在这里,我们将 $ \gamma_s $ 视为固定参数,其值为从电生理学估计中获取的 $ 116^{-1} $\cite{robinson2005multiscale},将 $ r_s $ 视为自由参数。
我们将该模型应用于每个半球 32,492 个顶点的皮质中层表面网格,以求解每个顶点的活动。
请注意,点之间活动的传播受其白质连接性的控制,其强度随距离呈指数衰减(补充图\ref{fig:supp_10})。
当方程\ref{eq:NFT_wave_model}转换为其等效积分形式时,这种距离依赖性更加明显(详细信息请参见补充信息\ref{sec:NFT_wave})。


\subsection{神经群模型} \label{sec:neural_mass}

我们将简单\textit{神经场理论}波动模型的性能与生物物理大规模神经群模型进行了比较,其中介观动力学是通过经验解剖连接耦合的神经群体(即神经群)的相互作用产生的。
在典型的神经群模型中,每个大脑区域 $ i $ 都有自己的平均场群体动力学,其时间活动 $ S_i $ 由以下一般方程定义:

\begin{equation}\label{eq:temporal_activity}
	\frac{dS_i}{dt} = f(\boldsymbol{S}, \theta_i, C, G), 
\end{equation}

其中 $ f $ 是描述该区域活动演变的函数。
该函数取决于其他区域 $ \boldsymbol{S} $ 的活动、局部总体参数 $ \theta_i $、区域 $ C $ 之间的解剖连接性(顶点分辨率连接组 $ W_{connectome} $ 的分割版本,用于导出连接组本征模式)以及缩放连接性的全局耦合参数 $ G $ 区域之间。


我们可以使用文献中的几种全脑神经群模型\cite{sanz2013virtual},从简单的相位振荡器模型(例如,Kuramoto 模型\cite{breakspear2010generative})到更复杂的生物物理群体模型(例如,Wilson-Cowan 模型\cite{wilson1972excitatory})。
所有这些模型都遵循方程\ref{eq:temporal_activity}的形式,特别是它们对解剖区域间连接矩阵 $ C $ 的依赖。
这里我们关注一种广泛使用的神经群模型,兴奋抑制平衡模型\cite{deco2014local,deco2021dynamical,demirtacs2019hierarchical,aquino2022intersection}。
兴奋抑制平衡模型使用平均场方法来近似每个大脑区域的局部群体动态,这些区域通过源自\textit{磁共振弥散成像}的解剖连接矩阵进行耦合。
每个大脑区域 $ i $ 包含相互作用的兴奋性和抑制性神经元群,受以下非线性随机微分方程控制:

\begin{equation}\label{eq:synaptic_gating_E}
	\frac{dS_i^{(E)}}{dt} = - \frac{S_i^{(E)}}{\tau_E} + (1-S_i^{(E)} \gamma r_i^(E) + \sigma \mu_i(t)),
\end{equation}

\begin{equation}\label{eq:synaptic_gating_I}
	\frac{dS_i^{(I)}}{dt} = - \frac{S_i^{(I)}}{\tau_I} + r_i^{I} + \sigma \mu_i(t),
\end{equation}

\begin{equation}\label{eq:firing_rate_E}
	r_i^{E} = H^{(E)} (I_i^{(E)}) = \frac{a_E I_i^{(E)} - b_E}{1 - exp[-d_e (a_E I_i^{(E)} - b_E)]},
\end{equation}

\begin{equation}\label{eq:firing_rate_I}
	r_i^{(I)} = H^{I} (I_i^{I}) = \frac{a_I I_i^{(I)} - b_I}{1 - exp[-d_E (a_E I_i^{(E)}) - b_I]'},
\end{equation}

% Excitatory
\begin{equation}\label{eq:input_current_E}
	I_i^{(E)}(t) = I^{ext} + W_E I_0 + w_{EE} S_i^{(E)} (t) + G J \sum_j C_{ij} S_j^{(E)}(t) - w_{IE} S_i^{(I)} (t), 
\end{equation}

% Inhibition
\begin{equation}\label{eq:input_current_I}
	I_i^{I} (t) = W_I I_0 +
				  W_{EI} S_i^{(E)} (t) - 
				  S_i^{(I)} (t),
\end{equation}

其中 $ S_i^{E,I} $、$ r_i^{E, I} $ 和 $ I_i^{E, I} $ 分别代表 $ E $ 和 $ I $ 群体的突触门控变量、放电率和总输入电流。
参数 $ \tau_{E,I} $ 是时间常数,
$ \gamma $ 是动力学速率常数,
$ v_i (t) $ 是具有标准偏差 $ \sigma $ 的时变随机高斯输入。
函数 $ H^{E,I} $ 是 S 型神经元响应函数,将总输入电流 $ I_i^{E,I} $ 转换为放电速率 $ r_i^{E,I} $,其由增益因子 $ a_{E,I} $、阈值电流 $ b_{E,I} $ 和 曲率参数 $ d_{E,I} $。
在方程\ref{eq:input_current_E}和\ref{eq:input_current_I}中,$ I_i^{ext} $ 是外部输入,$ I_0 $ 是兴奋性和抑制性群体的局部输入电流,分别由 $ W_E $ 和 $ W_I $ 缩放,
$ w_{EE} $ 是兴奋性-兴奋性强度,
$ w_{EI} $ 是兴奋-抑制强度,
$ w_{IE} $ 是抑制-兴奋强度,
$ G $ 是全局耦合参数,
$ J $ 是有效 N-甲基-d-天冬氨酸 (NMDA) 电导,$ C_{ij} $ 是从\textit{磁共振弥散成像}估计的区域 $ i $ 和 $ j $ 之间的结构连接强度。


总体而言,兴奋-抑制平衡模型有 15 个固定参数和 4 个自由参数,如补充表 3 所示。
固定参数的值取自参考文献\cite{demirtacs2019hierarchical}。
为了进行直接比较,我们还使用了参考文献\cite{demirtacs2019hierarchical}提供的离散连接组数据。
定义区域间结构连接矩阵 $ C $。
这些连接组数据源自 334 名无关人类连接组计划受试者的最低限度预处理的\textit{磁共振弥散成像}数据,并通过 FMRIB 软件库 (FSL)\cite{behrens2003characterization} 的概率纤维束成像工具构建。 
我们建议读者参考参考文献\cite{demirtacs2019hierarchical}了解数据处理和连接组构建的更多细节。
当使用连接组数据的分割版本时,我们的结果没有改变。
我们注意到,\textit{神经场理论}方程数值解(例如方程\ref{eq:temporal_activity})也在空间上的离散皮层,但将其离散为一个非常精细的点阵列,然后将每个点上选择的任何时间动力学与连续极限中正确的连通性进行积分。
因此,神经群模型近似于更通用的\textit{神经场理论}方法\cite{deco2008dynamic,spiegler2013systematic}。 
我们还重申,局部动力学不会影响\textit{神经场理论}波动模型的空间特征函数。



\subsection{血液动力学模型} \label{sec:haemodynamic_model}

为了模拟功能磁共振成像数据,我们使用完善的 Balloon-Windkessel 血液动力学模型将\textit{神经场理论}波动力学模型和兴奋抑制平衡神经群模型生成的神经活动转换为血氧水平依赖 (blood oxygen-level dependent, BOLD) 信号\cite{stephan2007comparing}。
请注意,虽然该模型是对 BOLD 信号\cite{aquino2012hemodynamic,pang2017effects,pang2018biophysically}基础的生理血液动力学过程的更详细模型的简单近似,但我们在这里使用这种近似是为了与文献中的绝大多数建模研究进行直接比较\cite{deco2021dynamical,demirtacs2019hierarchical,aquino2022intersection,deco2009key,cabral2014exploring}。
每个顶点或大脑区域 $ i $ 中的 BOLD-fMRI 信号由以下微分方程控制:

\begin{equation}\label{eq:vasodilatory_signal}
	\frac{dz_i}{dt} = N_i (t) - 
					  \kappa z_i(t) - 
					  \gamma [f_i(t) - 1], 
\end{equation}


\begin{equation}\label{eq:blood_inflow}
	\frac{df_i}{dt} = z_i(t),
\end{equation}


\begin{equation}\label{eq:blood_volume}
	\frac{dv_i}{d_t} = \frac{1}{\tau} [f_i(t) - v_i^{1/\alpha} (t)],
\end{equation}


\begin{equation}\label{eq:deoxyhaemoglobin}
	\frac{dq_i}{dt} = \frac{1}{\tau}
					\{
					\frac{f_i(t)}{\rho}
					[1 - (1-\rho)^{1/f_i(t)}]
						- v_i^{1/\alpha - 1} (t)
					\},
\end{equation}


\begin{equation}\label{eq:BOLD_signal}
	\frac{dy_i}{dt} = V_0
					\{
						k_1 [1 - q_i(t)] +
						k_2 [1 - \frac{q_i(t)}{v_i(t)}] + 
						k_3 [1 - v_i(t)]
					\},
\end{equation}


其中 $ z_i $、$ f_i $、$ v_i $、$ q_i $ 和 $ y_i $ 分别是血管舒张信号、血流量、血容量、脱氧血红蛋白含量和 BOLD 信号变量。
变量 $ N_i $ 表示由波动力学模型和兴奋抑制平衡神经群模型生成的神经活动。
对于波浪模型,$ N(t) $ 为 $ \phi (t) $,对于兴奋抑制平衡神经群模型,$ N_i (t) $ 为 $ S_i^{(E)} (t) $。
模型参数及其值取自之前的工作\cite{stephan2007comparing,heinzle2016hemodynamic}如下:
$ \kappa = 0.65^{-1} $ 是信号衰减率,
$ \gamma = 0.41^{-1} $ 是流量相关的消除率,
$ \tau = 0.98 s $ 是血流动力学传输时间,
$ \alpha = 0.32 $ 为格鲁布指数,
$ \rho = 0.34 $ 为静息氧提取分数,
$ V_0 = 0.02 $ 为静息血容量分数,
$ k_1 = 3.72 $、
$ k_2 = 0.53 $ 
和 $ k_3 = 0.53 $ 为 3T fMRI 参数。


\subsection{静息态动力学建模} \label{sec:modelling_resting}

% textit 不变斜体
我们使用\textit{神经场理论}波动力学模型和\textit{兴奋抑制平衡}模型以及血液动力学模型来估计各种自发功能耦合特性。
对于波动力学模型,我们根据之前的研究\cite{robinson2005multiscale,sanz2018nftsim},使用白噪声输入来模拟没有任何结构化刺激的情况下求解方程\ref{eq:NFT_wave_model}。
我们将所得解与方程 (\ref{eq:vasodilatory_signal}-\ref{eq:BOLD_signal}) 中的血液动力学模型相结合,以模拟 BOLD-fMRI 信号。
模拟的 BOLD 信号被下采样到 1,200 个时间帧的 0.72 秒采样间隔,以匹配补充信息 2.2 中描述的静息态人类连接组项目数据。
最后,我们使用 HCP-MMP1 分割对模拟的 BOLD 信号进行分割,并计算代表模型功能耦合的相关矩阵。


对于兴奋抑制平衡模型,我们可以使用上述方法从每个大脑区域的方程\ref{eq:synaptic_gating_E}的数值解开始计算模型功能耦合。
然而,已经表明可以将兴奋抑制平衡动力学模型和血液动力学模型的方程线性化以获得模型功能耦合\cite{demirtacs2019hierarchical}的解析近似。
由于\textit{兴奋抑制平衡}模型中有大量模型参数,我们在本研究中使用了这种解析近似,因为它允许更全面且计算效率更高的模型拟合。
参见参考文献\cite{demirtacs2019hierarchical}了解更多详情。


我们通过拟合每个模型的自由参数,将模型\textit{功能耦合}拟合到所有分析中使用的相同\textit{人类连接组项目}参与者的经验功能耦合(也使用 HCP-MMP1 分区进行分区)。
我们将参与者样本分为训练集和测试集,每个集有 125 人,以便能够对模型性能进行样本外评估并避免过度拟合。
具体来说,我们在训练集数据上拟合模型参数,并使用拟合的模型参数来预测测试集数据。
模型拟合和性能评估基于三个广泛使用的功能耦合相关指标\cite{deco2021dynamical,aquino2022intersection}:边功能耦合、节点功能耦合和\textit{功能耦合的动力学属性}。


边\textit{功能耦合}度量是通过采用 $ z $ 变换模型的上三角元素(即 边功能耦合的强度)和经验功能耦合的皮尔逊相关性来计算的。
我们只采用上三角元素,因为功能耦合值相对于对角线对称。
较高的相关性代表模型和数据之间更好的拟合。


节点功能耦合指标是通过模型中每个大脑区域的平均功能耦合强度与经验功能耦合的皮尔逊相关性来计算的。
脑区$ i $的平均功能耦合强度定义为$ \frac{1}{n} \sum_{j=1}^{n} FC_{ij} $,其中$ n $是脑区的总数。
再次强调,相关性越高表明模型拟合效果越好。


\textit{功能耦合的动力学属性}指标捕获静息状态活动的时空统计数据,计算如下\cite{aquino2022intersection}。
每个区域 $ i $ 的时间序列均使用二阶巴特沃斯滤波器在 0.04 至 0.07 Hz 之间进行滤波;
该乐队基于参考文献\cite{deco2017dynamics}及其纳入的动机是其与大脑的功能相关性\cite{glerean2012functional,pang2019power}。
然后对时间序列进行希尔伯特 ($ H $) 变换以计算数量 $ y_i(t) = x_i(t) + jH_i (t) $,其中 $ j $ 是虚数。
然后计算瞬时复数参数 $ \theta_i(t) = tan^{-1} [H_i(t) / x_i(t)] $。
在时间 $ \Delta(i,j,t) $ 时区域 $ i $ 和 $ j $ 之间的同步水平计算如下:

\begin{equation}\label{eq:synchrony}
	\Delta(i,j,t) = cos [\theta_i (t) - \theta_j (t)].
\end{equation}

请注意,我们并不像文献中有时那样将 $ \theta_i(t) $ 称为相位,因为我们的信号是宽带的,而这种解释仅适用于几乎单色的信号。
我们计算这个量是为了与之前的工作进行比较\cite{deco2021dynamical,aquino2022intersection},然后我们计算两个时间实例$ \tau_u $ 和 $ \tau_v $ 之间的全局同步相似度 $ \phi_{uv} $,其中 $ \phi_{uv} $ 定义为:

\begin{equation}\label{eq:synchrony_similarity}
	\phi_{uv} = \frac{1}{d_u d_v} 
				\sum_{i>j} \Delta(i,j,\tau_u) \Delta(i,j,\tau_v),
\end{equation}

其中
\begin{equation}\label{key}
	d_x = \sqrt{\sum_{i>j}
					[
					\Delta(i,j,\tau_x)
					]^2
				}.
\end{equation}

这里,$ \phi_{uv} $ 是\textit{功能耦合的动力学属性},它是一个对称的时间$ \times $时间矩阵,解释为时间$ \tau_u $ 和 $ \tau_v $ 之间全局同步的相似性。
然后,我们比较了模型上三角元素的分布和经验\textit{功能耦合的动力学属性}估计,按照参考文献将所有个体或模型实现连接起来\cite{aquino2022intersection}。
更好地捕捉数据的总体波动。
使用KS统计量比较模型和数据的\textit{功能耦合的动力学属性}分布,KS统计量越低表明模型拟合越好。


波模型有一个自由参数,即波传播的空间长度尺度 $ r_s $,该参数经过优化以适合经验功能磁共振成像数据。
具体来说,我们构建了一个由 20 个 $ r_s $ 值组成的向量,这些值均匀分布在 10 到 100 mm 之间。
对于每个 $ r_s $ 值,计算上述功能耦合指标。
扩展数据图\ref{fig:extended_fig_10} 显示了优化情况。
我们采用最小化 \textit{功能耦合的动力学属性} KS 统计量的 $ r_s $ 值,因为该指标已被发现是上述三个指标中模型数据比较的最严格基准\cite{aquino2022intersection}。
该过程产生了 $ r_s = 28.9 mm $ 的优化值,该值小于之前在脑电图数据分析中获得的值\cite{robinson2005multiscale},可能是由于血流动力学过程限制了神经活动传播到邻近区域,或者我们在此关注皮质-皮质动力学\cite{aquino2012hemodynamic,pang2017effects}。


\textit{兴奋抑制平衡}神经群模型有 15 个固定参数和 4 个自由参数,即 $ w_{EE} $、$ w_{EI} $、$ w_IE $ 和 $ G $,这些参数经过优化以拟合数据,如下参考文献\cite{demirtacs2019hierarchical}所示。
该程序得出的优化参数为 $ w_{EE} = 9.80 $ 、$ w_{EI} = 1.48 $、$ w_{IE} = 7.13 $ 和 $ G = 6.87 $。
优化过程的更多细节可以在补充信息\ref{sec:mass_optimization}中找到。



\subsection{静息态动力学时滞特性的测量} \label{sec:dynamics_measurement}

除了用于模型拟合和模型性能评估的基于功能耦合的指标之外,我们还研究了波动力学模型和神经群模型是否也可以捕获传播活动的时间特性。
特别是,我们分析了静息态 BOLD-fMRI 时间过程的滞后结构(或滞后线程),如参考文献\cite{mitra2015lag,mitra2014lag}所提出的。
我们在补充信息\ref{sec:lag_threads}中简要讨论了经验和模拟 fMRI 数据的滞后结构的计算算法,并请读者参阅之前的文章\cite{mitra2015lag,mitra2014lag} 以了解更多详细信息。
请注意,还有几种其他方法可以表征静息状态活动的时空特性\cite{bolt2022parsimonious,cabral2017functional,kashyap2019dynamic},但当前选择的方法足以满足我们的目的。


\subsection{建模刺激诱发动力学} \label{sec:modelling_stimulus}

我们还评估了波模型捕捉诱发神经反应的经典特性的程度。
具体来说,我们使用优化的波模型($ r_s = 28.9 mm $)来研究响应施加到左侧初级视觉皮层的刺激的神经动力学。
特别是,刺激 $ Q(\boldsymbol{r},t) $ 是一个 1 ms 脉冲($ t = 1–2 ms $),幅度为 $ 20 s^{-1} $(结果对幅度变化很稳健),仅限于皮质表面的顶点 由 HCP-MMP1 分区定义的 初级视觉皮层区域。
我们以 0.1 毫秒的分辨率在 100 毫秒的时间段内进行了仿真。 
最后,我们使用 HCP-MMP1 分区将活动 $ \phi(\boldsymbol{r},t) $ 分区为 180 个区域。


我们比较了每个大脑区域的活动概况(幅度与时间),重点关注活动达到峰值幅度的时间(即达到峰值的时间)。
具体来说,我们研究了活动的时间优先顺序是否遵循从视觉到额叶皮层的人类视觉皮层层次结构,其中包括以下 17 个大脑区域:V1、V4、7m、7Am、TE1p、7AL、24dd、2、24dv、8BM、10r 、10v、8BL、10pp、10d、9-46v 和 9-46d。
这些区域与先前使用束追踪数据和非线性网络建模在猕猴新皮质视觉层次中识别的区域非常相似\cite{chaudhuri2015large}。


我们还将峰值响应时间的区域估计值与 T1w:T2w 值进行了比较,这是一种对皮质内髓磷脂含量敏感的非侵入性测量\cite{glasser2011mapping},也是皮质层次等级\cite{burt2018hierarchy}的良好代表。
fsLR-32k 空间中的髓磷脂图是从\textit{人类连接组项目} 数据集27,28 中获得的,然后使用 HCP-MMP1 分区进行分区。
我们通过斯皮尔曼等级相关性量化了达到峰值时间的区域值与髓磷脂含量之间的关系。
通过与使用空间约束旋转测试方法获得的 10,000 个相关值的零分布进行比较来评估相关性的统计显着性\cite{vavsa2018adolescent,alexander2018testing}。
该方法计算一张地图与另一张地图的随机空间旋转版本之间的相关性,从而保留地图中地块的空间关系。
所得 $ P $ 值 Pspin 是大于经验相关值的零相关值的分数。
最后,我们重复了上述统计测试,包括所有大脑区域(即,不限于视觉层次大脑区域),以确保我们的发现不是由我们对感兴趣区域(ROI)的特定选择驱动的。
这是一个更为保守的测试,因为并非所有大脑区域都对视觉刺激表现出强烈的诱发反应。


\subsection{非新皮层结构几何特征模式的估计} \label{sec:geometric_estimation}

我们将特征模式分析扩展到新皮质以外的区域,特别关注皮质下层(丘脑和纹状体)和古皮质(海马体)。
与可以建模为 2D 片的皮质带不同,这些结构是实体 3D 对象。 
因此,我们使用四面体网格而不是基于表面的三角形网格来计算\textit{几何特征模式},以解释非新皮质结构的完整 3D 几何结构\cite{wachinger2015brainprint},如下所述。


我们首先使用概率\href{https://fsl.fmrib.ox.ac.uk/fsl/fslwiki/Atlases}{哈佛-牛津皮质下图集}在每个半球生成丘脑、纹状体和海马体的体积二元掩模。
掩模中包含有 25\% 或更高概率属于这些结构的体素。
我们使用 FreeSurfer 的 mri\_mc 函数(该函数实现了 marching-cubes 算法)通过对体积掩模进行细分来构造 2D 表面,然后使用\href{https://gmsh.info/}{Gmsh 软件}将 2D 表面转换为 3D 四面体网格 。
然后,我们使用 LaPy python 库在每个非新皮质结构的四面体网格上求解方程\ref{eq:1},以获得特征模及其相应的特征值。
最后,我们通过插值将四面体空间中的本征模投影回自然体积空间。
因此,所产生的本征模式在空间上通过构成每个非新皮层结构体积的 3D 体素而变化。
在这部分研究中,我们放弃了第一个恒定模式,并在分析中使用了接下来的 20 个模式。


\subsection{非新皮层结构的功能组织映射} \label{sec:functional_mapping}

非新皮质结构中功能磁共振成像的信噪比通常比新皮质结构中的信噪比弱\cite{uugurbil2013pushing}。
此外,由于功能磁共振成像的分辨率有限以及常见功能磁共振成像处理方法引起的空间平滑,细粒度任务激活通常很难在这些较小的结构中解决。
因此,为了有效地绘制丘脑、纹状体和海马体的功能组织,我们使用了每个结构中静息状态fMRI信号的连接映射\cite{haak2018connectopic} 以获得它们的主要功能模式(通常称为梯度,尽管这些模式捕捉了FC图谱逐点相似性的空间变化)。
这项技术和相关程序已在过去的工作中广泛用于研究这些结构的功能组织\cite{vos2018anatomical,yang2020thalamic,oldehinkel2022mapping}。


我们按照参考文献中描述的程序将连接映射应用于\textit{人类连接组项目}个体的体积体素静息态 fMRI 数据\cite{haak2018connectopic}。
具体来说,对于每个个体和非新皮质 ROI(即丘脑、纹状体和海马体),我们构建了一个 ROI 时间序列数据矩阵 $ A $,大小为 $ T \times N $,其中 $ T $ 是时间帧数,$ N $ 是 ROI 中的体素数量。
类似地,对于每个个体,我们构建了大小为 $ T \times M $ 的灰质时间序列数据矩阵 $ B $,其中 $ M $ 是 ROI 之外的灰质体素的数量。
由于 $ M $ 通常很大(103 或以上),因此我们使用奇异值分解来降低 $ B $ 的维数,以构造大小为 $ T \times (T – 1) $ 的矩阵 $ B $。
然后使用 $ A $ 和 $ B $ 的皮尔逊相关性计算 ROI 内每个体素的连通性指纹,以获得大小为 $ N \times (T – 1) $ 的矩阵 $ C $。
使用 $ \eta^2 $ 系数 \cite{alexander2018testing} 计算每对 ROI 体素之间连接指纹的相似性,从而得到大小为 $ N \times N $ 的矩阵 $ S $。
$ \eta^2 $ 系数表示一个连接剖面中的方差由另一个连接剖面中的方差所占的比例。
然后我们计算所有个体的平均 $ S $。 
使用拉普拉斯特征图算法\cite{belkin2003laplacian}的非线性流形学习过程应用于每个 ROI 的 $ S $ 矩阵,以计算其特征向量和特征值。
特征向量表示 ROI 的功能模式(称为功能梯度),根据它们解释的 功能耦合相似性的方差进行排序。
我们仅分析了前 20 个非常量梯度,以便能够与几何本征模进行直接比较。
基于功能耦合的梯度分析的典型应用很少考虑超过前五个梯度\cite{haak2018connectopic,margulies2016situating}。


假设模式和梯度的符号是任意的,我们通过获取它们的绝对空间相关性来分析每个非新皮质结构中的几何本征模式和函数梯度之间的对应关系。
此外,同一本征群内几何本征模的顺序可以改变(补充表1;例如,第一个本征群可以在模2-4之间进行顺序翻转),因此几何本征模和 不保证函数梯度。
因此,我们检查了模式和梯度之间所有可能的成对相关性,并评估了与每个梯度最大相关的模式的对应关系。
对于丘脑、纹状体和海马体,几何模式和功能梯度之间观察到的最大阶数差异分别为 20 中的 1、8 和 5。


\subsection{数据获取} \label{sec:data_availability}

原始和预处理的 HCP 数据可以通过 \href{https://db. human connectome.org/}{HCP} 访问。
NeuroVault 数据可通过\href{https://neurovault.org/}{链接} 访问。
用于复制研究结果的源数据可在\href{https://github.com/NSBLab/BrainEigenmodes}{github}  和 \href{https://osf.io/xczmp/}{osf} 上公开获取。

\subsection{代码获取} \label{sec:code_availability}

用于计算特征模式、分析结果和重现研究数据的计算机代码可在\href{https://github.com/NSBLab/BrainEigenmodes}{github}上公开获取。



\section{扩展数据的图}


\begin{figure}[!htb] 
	\centering
	\includegraphics[width=0.9\textwidth]{fig/extended_fig_1.pdf}
	\caption{\textbf{特征模式基组}。
		从左到右的基组是\textit{几何特征模式}、连接组特征模式、使用与\textit{指数距离规则}特征模式使用的密度相匹配的连接矩阵的连接组特征模式,和距离指数规则特征模式。 
		负值-零值-正值被着色为蓝-白-红。
	} \label{fig:extended_fig_1}
\end{figure}


\begin{figure}[!htb] 
	\centering
	\includegraphics[width=0.9\textwidth]{fig/extended_fig_2.pdf}
	\caption{对于不同的分割分辨率,通过不同的基集实现静息状态功能耦合的重建精度。 
		基组是\textit{几何特征模式}、连接组特征模式、使用密度与\textit{指数距离规则}特征模式使用的连接矩阵相匹配的连接组特征模式和指数距离特征模式。
		Schaefer100、Schaefer200、Glasser360、Schaefer400、Schaefer600、Schaefer800 和 Schaefer1000 分别在两个半球拥有 100、200、360、400、600、800 和 1000 个块。
	} \label{fig:extended_fig_2}
\end{figure}


\begin{figure}[!htb] 
	\centering
	\includegraphics[width=0.9\textwidth]{fig/extended_fig_3.pdf}
	\caption{\textbf{在不同分割分辨率下,通过\textit{几何特征模式}和连接组特征模式实现的所有47个人类连接组项目任务对比图的重建精度差异}。
	每行代表不同的任务对比,此处按广泛类型进行分组(补充信息 2.1)。
	wm = 工作记忆。
	红色表示\textit{几何特征模式}的卓越性能。
	Schaefer100、Schaefer200、Glasser360、Schaefer400、Schaefer600、Schaefer800 和 Schaefer1000 分别在两个半球拥有 100、200、360、400、600、800 和 1000 个块。}
	\label{fig:extended_fig_3}
\end{figure}


\begin{figure}[!htb] 
	\centering
	\includegraphics[width=0.9\textwidth]{fig/extended_fig_4.pdf}
	\caption{\textbf{\textit{几何特征模式}和主成分分析实现的重建精度}。
	(\textbf{a}) 静息态功能耦合的重建精度。
	(\textbf{b}) 比较所有 47 个 \textit{人类连接组项目} 任务对比图的重建精度,这些图已按广泛类型进行了分组(补充信息 2.1)。
	wm = 工作记忆。
	每行代表不同的任务对比。
	红色表示\textit{几何特征模式}的卓越性能。
	星号表示用于训练主成分分析相关任务中的对比(即七个关键 \textit{人类连接组项目}任务对比)。}
	\label{fig:extended_fig_4}
\end{figure}


\begin{figure}[!htb] 
	\centering
	\includegraphics[width=0.9\textwidth]{fig/extended_fig_5.pdf}
	\caption{\textbf{几何特征模式与傅立叶基集的比较}。
	\textbf{a},具有单位系数的六个不同傅里叶基组的模式1、2、3、4、10、50和100的空间图。
	术语 reg 和 irreg 表示 $ x $、$ y $ 和 $ z $ 方向上的模式的空间波长分别以规则和不规则增量间隔开。
	详情请参阅补充信息 6。
	\textbf{b},七个关键\textit{人类连接组项目}任务对比图和静息状态功能耦合的重建准确性。
	有关对比图的详细信息,请参阅补充信息 2.1。
	wm = 工作记忆。}
	\label{fig:extended_fig_5}
\end{figure}


\begin{figure}[!htb] 
	\centering
	\includegraphics[width=0.9\textwidth]{fig/extended_fig_6.pdf}
	\caption{\textbf{阈值统计图的经典神经成像方法}。
	\textbf{a},简单的一维示例,说明不同阈值如何仅捕获激活的焦点簇并忽略激活的底层结构化模式。
	\textbf{b},使用未阈值和二值化阈值地图描述的概念的空间嵌入式演示。}
	\label{fig:extended_fig_6}
\end{figure}


\begin{figure}[!htb] 
	\centering
	\includegraphics[width=0.9\textwidth]{fig/extended_fig_7.pdf}
	\caption{\textbf{七个关键\textit{人类连接组项目}任务对比图中每一个的归一化功率谱}。
	有关对比图的详细信息,请参阅补充信息 2.1。
	wm = 工作记忆。}
	\label{fig:extended_fig_7}
\end{figure}


\begin{figure}[!htb] 
	\centering
	\includegraphics[width=0.9\textwidth]{fig/extended_fig_8.pdf}
	\caption{\textbf{经验任务激活映射和代理映射的功率谱}。
	\textbf{a},47 个\textit{人类连接组项目}任务对比图(上)和来自 NeuroVault 数据库的 10,000 个对比图(下)的归一化平均功率谱。
	彩色线对应于应用具有不同半高全宽 (full-width at
	half-maximum, FWHM) 的空间平滑滤波器后的替代数据的功率谱。
	\textbf{b},平均均方对数误差 (MSLE) 作为\textit{人类连接组项目}和 NeuroVault 对比图的归一化平均功率谱与平滑替代数据之间的 FWHM 的函数。
	\textbf{c},分别在 47 个\textit{人类连接组项目}和 10,000 个 NeuroVault 对比图和平滑替代数据之间的功率谱之间获得 MSLE。
	每行代表不同的任务对比图。
	这些线对应于每张图的 MSLE 最小的 FWHM。}
	\label{fig:extended_fig_8}
\end{figure}


\begin{figure}[!htb] 
	\centering
	\includegraphics[width=0.9\textwidth]{fig/extended_fig_9.pdf}
	\caption{\textbf{波动力学模型和神经群模型在捕捉功能磁共振成像数据时滞特性方面的比较}。
	\textbf{a},来自实证数据、使用波模型的模拟数据和使用左半球神经群模型的模拟数据的时滞矩阵。
	负-零-正值被着色为蓝-白-红。
	\textbf{b},投影到皮层表面的 a 矩阵(每列的平均值)的平均滞后。
	负-零-正值被着色为蓝-白-红。
	散点图显示了来自 180 个大脑区域的两个模型的经验数据和模拟数据的平均滞后的关系。
	红线表示与皮尔逊相关系数 $ r $ 和单边旋转测试 $ p $ 值 pspin 的线性拟合,根据 10,000 个排列进行估计。 
	\textbf{c},与 b 类似,但基于 a 中矩阵的第一个主成分 (PC1)。
	表面上方的数字 (var) 对应于 PC 解释的方差。
	\textbf{d},与 c 类似,但位于 a 中矩阵的第二个 PC (PC2) 上。}
	\label{fig:extended_fig_9}
\end{figure}


\begin{figure}[!htb] 
	\centering
	\includegraphics[width=0.9\textwidth]{fig/extended_fig_10.pdf}
	\caption{\textbf{波动力学模型的优化}。
	该模型在 125 名\textit{人类连接组项目}个体上进行训练,以找到参数 $ r_s $ 的最佳值(以毫米为单位)。
	优化性能根据以下指标对数据和模型功能耦合进行比较:边缘 功能耦合相关性、节点功能耦合相关性和 FCD KS 统计量。
	较高的边缘功能耦合相关性、较高的节点功能耦合相关性和较低的 FCD KS 统计量对应于更好的模型拟合。
	我们将 $ r_s = 28.9 mm $ 作为最佳参数,因为它会导致 FCD KS 统计值最小。}
	\label{fig:extended_fig_10}
\end{figure}



\section{补充信息}

\subsection{神经场理论} \label{sec:NFT}

\textit{神经场理论}是一类生物物理模型,它将中尺度到宏观尺度的大脑动力学(从约 0.5 毫米到整个大脑)解释为空间扩展、随时间变化的神经活动场的结果\cite{beurle1956properties,da1976models,wright1995simulation,deco2008dynamic,jirsa1996field,robinson1997propagation,robinson2005multiscale}。
一般 \textit{神经场理论} 关注神经群体的局部平均动态,例如平均放电率和体细胞电压。
重要的生物物理过程,例如树突和突触过程、轴突传导延迟以及细胞体中树突电流的总和,通过一系列原则性的数学简化被纳入局部群体平均值。


\textit{神经场理论} 将尺度大于 0.5 毫米的神经组织视为具有局部特性和点对点白质连接性的空间连续体,该连接性随着距离的增加而平滑减小。
虽然可以通过 \textit{神经场理论} 处理更一般的情况,包括与皮层下结构的相互作用\cite{robinson2005multiscale,sanz2018nftsim},但最简单的版本仅包括皮层,并假设其连接性是均匀且各向同性的,连接强度仅取决于点之间的距离,并且通常会递减,它们的间隔大约呈指数关系\cite{deco2008dynamic,braitenberg2013cortex,henderson2014relations,robinson2019physical}。
这种关系在功能 MRI (fMRI) 的背景下尤其重要,其中动态相对较慢,并且活动连接的强度具有大约指数\cite{braitenberg2013cortex,robinson2019physical,robinson2012interrelating}的空间依赖性。
由此产生的活动场的时空演化已被 Robinson 及其同事开发的 \textit{神经场理论} 的生理约束变体有效地描述了\cite{beurle1956properties,da1976models,jirsa1996field,robinson1997propagation,nunez1974brain},该变体包含受外部输入和本地刺激后穿过皮层片传播的阻尼波。
皮质或皮质丘脑动力学。
在过去的二十年中,这种 \textit{神经场理论} 公式成功地以统一的方式解释和统一了各种实验现象,包括但不限于脑电图 (EEG) 频谱\cite{robinson2001prediction,pang2018neural}、诱发电位\cite{rennie2002unified,mukta2020evoked}、唤醒状态\cite{abeysuriya2015physiologically,assadzadeh2018necessity} 、全脑有效连接\cite{robinson2012interrelating}、皮层活动波\cite{gabay2018dynamics} 和睡眠状态重组\cite{robinson2005multiscale}。
由于中尺度到宏观尺度的大脑动力学在正常条件下(不包括癫痫样动力学)\cite{robinson2019physical}大约在线性状态下运行,Robinson 等人。
\textit{神经场理论}表明,活动的空间本征模式(模式)自然地出现在大脑中\cite{wang2016brain,roberts2017consistency},其属性由大脑的内在几何形状和拓扑结构决定\cite{preti2019decoupling,gabay2018dynamics}。
这些模式是本研究感兴趣的主题。
我们建议读者查阅大量 \textit{神经场理论}文献以进行更详细的讨论(参见示例\cite{wright1995simulation,jirsa1996field,robinson1997propagation,robinson2016eigenmodes,gabay2017cortical,wang2016brain,honey2007network,van2013wu,glasser2016multi,naze2021robustness}以及其中引用的参考文献)。



\subsection{人类连接组项目数据} \label{sec:sup_2}

\subsubsection{任务诱发的数据} \label{sec:sup_2_1}

我们分析了在 7 个任务域中测量的任务诱发功能磁共振成像数据,这些数据已被证明可以可靠地组成各种神经系统\cite{barch2013function}。
这 7 个任务是:社交、运动、赌博、工作记忆、语言、情感和关系。
补充表 2 显示了每个任务领域涉及的具体对比以及本研究中调查的关键对比。
我们总共分析了 47 个对比,其中包括 7 个关键对比。
关键对比代表了文献中常用的那些来映射任务引发的主要激活模式。
有关每项任务和对比的详细信息,请参阅\cite{barch2013function}。
该分析是在通过\href{https://fsl.fmrib.ox.ac.uk/}{FSL} 交叉运行(2 级)FEAT 分析\cite{woolrich2004multilevel}计算的各个任务激活图上进行的。
我们使用\textit{人类连接组项目}提供的任务图,通过多模态表面匹配\cite{robinson2018multimodal},将最小平滑 (2 mm) 映射到 fsLR-32k CIFTI 空间,每个半球有 32,492 个顶点。


\subsubsection{无任务静息态数据} \label{sec:sup_2_2}
我们使用从左到右 (LR) 编码方向分析了在一次扫描过程中获得的无任务静息态 fMRI。
扫描持续 14.4 分钟,总共 1200 个时间帧。
简而言之,静息态 fMRI 采集参数为:各向同性体素大小为 2 mm,重复时间 (TR) 为 720 ms,回波时间 (TE) 为 33.1 ms。
所有其他采集参数都可以在\cite{van2013wu}中找到。
每个人的数据均由\textit{人类连接组项目}通过其最小预处理管道进行预处理\cite{glasser2013minimal},并且还经过 ICA-FIX 来纠正结构噪声和残余混杂\cite{salimi2014automatic}。
没有执行额外的平滑。
与任务诱发数据类似,静息态数据被映射到 fsLR-32k CIFTI 空间。
因此,每个人的数据被表示为每个半球上大小为 32,492 个顶点 × 1200 个时间帧的矩阵。


\subsubsection{连接组数据} \label{sec:sup_2_3}

为了导出连接组本征模式,我们使用了通过概率纤维束成像从\textit{磁共振弥散成像}数据导出的单个连接组,如\cite{tian2021high}中提供的。
简而言之,\textit{磁共振弥散成像}采集参数为:各向同性体素大小为 1.25 mm,TR 为 5520 ms,TE 为 89.5 ms,b 权重为 1000、2000、3000 s/mm2 和 6 次 b0 扫描。
所有其他采集参数均可在\cite{van2013wu}中找到。
每个人的数据均由\textit{人类连接组项目}通过其扩散预处理流程 (v3.19.0) \cite{glasser2013minimal}进行预处理。
为了生成连接组,使用 MRtrix 和概率纤维束描记术生成纤维束图,500 万条流线连接解剖学上不同的流线 脑区域、多壳多组织 (MSMT) 约束球形反卷积 (CSD) 解剖约束纤维束成像 (ACT) 和纤维方向分布算法的二阶积分 (iFOD2)。
详细信息请参见\cite{tian2021high}。
标准 MNI 空间中的 fsLR-32k 皮层表面网格用于定义灰质-白质界面。
对于每个个体,每个半球生成的流线被映射到表面网格上最近的顶点,以构建高分辨率加权连接组(32,492 × 32,492 矩阵大小和代表流线总数的权重)。
详细信息请参见\cite{tian2021high}。


\subsection{个体特异性皮层特征模式} \label{sec:individual_specific}
\textit{亥姆霍兹方程}~\ref{eq:1}可用于求解任何皮层表面网格模型的特征模式,但网格几何形状的变化可能会改变生成的特征值 $ \lambda $ 和特征模式本征模态 $ \psi $。
因此,从各个对象表面导出的特征值和特征模式,特别是在非常短的波长下,将有所不同,并且不能直接比较\cite{henderson2022empirical,chen2022individuality}。
为了简单起见,这里我们使用从公共模板表面生成的特征模式(参见方法\ref{sec:derivation}中的“皮层几何本征模式的推导”)。
虽然这允许比较不同个体的数据重建,但它掩盖了与皮层几何个体差异相关的任何潜在影响。
然而,补充图 \ref{fig:supp_3} 显示,使用从各个皮层表面导出的\textit{几何特征模式}不会改变研究的总体结果,这表明,就目前的目的而言,从总体平均模板表面导出的特征模式代表了基本几何特征模态的良好近似。
尽管如此,为了强调个体几何结构的某些细微差别,我们在补充图\ref{fig:supp_4} 中表明,在某些个体中,200 个个体特定的特征模式的表现略好于模板导出的本征模式,特别是在重建任务激活图方面,但在重建静息状态活动方面则不然。 。 
然而,补充图\ref{fig:supp_5} 显示,个体特定和模板衍生的特征模式的结果最终在非常短的波长(~第500个模式)处收敛。
因此,前 200 种模式捕获了皮层几何结构的个体差异对大脑功能的影响,这与最近的工作\cite{chen2022individuality}一致,并且对应于最大可能的特征模式数量的一小部分(~0.6\%)。


\subsection{连接组阈值对连接组特征模式的影响} \label{sec:thresholding_effect}

许多网络属性取决于网络连接密度,而我们生成组平均连接组 $ A_C $ 的特定阈值程序有些任意。
因此,我们还推导了 $ A_C $ 变体的连接组本征模式,这些阈值范围从 1\% 到 13\%(即未阈值连接组的密度),增量为 2\%。
补充图 \ref{fig:supp_6} 中的结果表明,连接密度确实影响连接组本征模式的重建精度,因此稀疏矩阵与改进的重建相关。
鉴于远程连接往往具有较低的权重,因此更有可能首先被阈值化,更保守的阈值的应用强调局部的、类似\textit{指数距离规则}的连接,直到最终收敛于表面网格。
因此,这些结果与以下观点一致:当主要关注局部连接性时,重建性能通常会得到改善。
值得注意的是,从 0.01\% 到 5\% 的精细采样连接密度揭示了 0.4\% 到 0.7\% 之间的假定最佳阈值,这导致 200 个连接组本征模式的峰值重建精度(补充图 \ref{fig:supp_7})。
这个最佳阈值可能代表了\textit{磁共振弥散成像}管道重建中灵敏度和特异性之间生理上合理的权衡。
然而,重建精度从未超过几何本征模态的精度。
与几何本征模相比,连接组本征模获得的重建精度对密度阈值的敏感性进一步强调了该方法的复杂性,几何本征模不需要选择特定阈值。


我们对高分辨率、顶点级连接组的关注是为了与之前检查连接组特征模式在重建功能数据中的功效的工作进行公平比较\cite{mitra2015lag,felleman1991distributed}。
然而,许多其他关于连接组的研究首先对数据应用离散分区,并研究通常映射在约 102 至约 103 分区大脑区域水平的连接组的属性\cite{chaudhuri2015large,goodale1992separate}。
为了完整起见,我们还使用以不同分辨率分割的连接组导出本征模,这些连接组是使用 Schaefer400、Schaefer600、Scaefer800 和 Schaefer1000 分割生成的。
选择这些分区是因为它们每个半球至少有 200 个分区,使我们能够计算至少 200 个连接组特征模式,以便与研究中使用的 200 个几何本征模直接进行比较。
但请注意,这种方法忽略了大脑的空间嵌入,导致与几何本征模相比重建精度较差(补充图\ref{fig:supp_8}
)。
这一结果再次证实了在为大脑功能推导适当的解剖基础集时捕获局部连接性和区域几何形状的重要性。




\subsection{几何特征模式与函数导出基集的比较} \label{sec:comparison_eigenmodes_derived}

为了进一步评估几何特征模在表示大脑活动方面的功效,我们将它们的性能与通过功能数据本身的主成分分析导出的基础集进行了比较。
由于主成分分析定义了功能数据的线性最优分解,我们试图通过将主成分分析应用于 200 个人的训练集并在保留的测试集上验证其重建准确性来评估该方法的样本外泛化性。 55 人。 
选择训练集的个体数量来产生 200 个主成分,与研究中分析的几何本征模的数量相匹配。
请注意,主成分是按照它们解释的数据方差的顺序排列的。


对于任务诱发数据,我们构建了 7 个训练数据集,每个训练数据集对应补充表 2 中的 7 个关键任务对比。
每个训练数据集都是通过使用每个任务的激活数据形成一个大小为 32,492 个顶点 $ \times $ 200 个个体的矩阵来构建的。
然后,我们应用空间主成分分析获得 200 个空间主成分,描述个体间空间激活模式的常见方差模式(补充图\ref{fig:supp_9})。


对于无任务静息态数据,可以实现类似的方法,其中通过暂时连接训练集中个体的数据来构建训练数据集,产生大小为 32,492 $ \times $ 240,000 的矩阵(即 240,000 = 1200 次) 帧 × 200 个人)。
然而,在这个非常大的数据矩阵上执行标准主成分分析会带来很高的计算负担。
因此,我们使用了 MELODIC Incremental Group-PCA (MIGP) 方法,该方法已被证明可以使用计算效率更高的方法\cite{markov2014anatomy}在完全连接的数据集中生成与标准主成分分析非常接近的输出。
总而言之,MIGP 暂时连接 少量个体(这里我们使用两个个体)的数据构建矩阵$ W $。然后使用$ W $构建时间帧×时间帧协方差矩阵,并应用特征值分解来提取前$ m $个时间特征向量(这里我们 使用$ m = 1200 $),并将特征向量与$ W $相乘,得到$ m $个加权空间特征向量,产生新的$ W $。
然后,通过将$ W $与下一个个体的数据连接来增量更新$ W $,重复上述整个过程以获得新的$ W $ 表示 $ m $ 个空间特征向量。
执行这种迭代方法,直到考虑了训练集中的所有个体。
我们保留了 $ W $ 的前 200 名空间主成分。


然后使用从任务诱发和无任务静息状态数据的训练集中获得的 主成分来分解测试集中个体的数据并计算相应的样本外重建精度。
请注意,\textit{几何特征模式}的性能本质上是样本外的,因为这些模式是从独立于此处考虑的功能磁共振成像数据集的皮质表面网格表示导出的。


我们强调,从主成分分析派生的组件仅捕获数据的统计特性(即最大方差),并且不提供任何机制见解;
即,它们是数据的统计或现象学描述,并且根据所解释的方差排序,而不考虑空间波长。
因此,它们与几何本征模有根本的不同,几何本征模与系统的结构特性直接相关,由大脑结构如何产生功能\cite{wang2016brain}的生成模型告知,并根据其空间频率或波长进行排列。 
如扩展数据图\ref{fig:extended_fig_4}所示,这种区别导致主成分分析相对于几何本征模具有早期性能优势,其中前几个 主成分 可以比前几个几何本征模更准确地捕获数据。
这是因为 主成分 不具有与几何本征模相同的排序约束;
也就是说,早期的 主成分 可以自由地跨越任何空间波长,这由数据的最佳分解决定,而几何本征模则通过构造从长波长到短波长排序。
然而,扩展数据图 \ref{fig:extended_fig_4} 还表明,主成分 重建样本外数据的准确性存在限制,这可能是因为高阶 主成分 捕获了训练数据的特殊属性。
相比之下,添加更多几何本征模可以提高模型解析训练数据和测试数据的可重复短波长特征的能力,使得样本外重建在大多数比较中超过 主成分 的准确性约 40-50 个模式 。


\subsection{几何特征模式与傅立叶基集的比较} \label{sec:comparison_fourier}


为了进一步确认\textit{几何特征模式}的性能并不是由任何基组展开的数学方式简单地驱动的,我们将几何本征模与基于正弦和/或余弦组合的六个简单空间傅里叶基组进行了比较。
傅里叶基组是使用以下实值函数构建的:

\begin{equation}\label{eq:real_functions_1}
	F_1 := cos(
				\frac{2 \pi (j_x - 1) x}{L_x} + 
				\frac{2 \pi (j_y - 1) y}{L_y} + 
				\frac{2 \pi (j_z - 1) z}{L_z}
			  ),
\end{equation}

\begin{equation}\label{eq:real_functions_2}
	F_2 := C_1 cos(
				   \frac{2 \pi (j_x - 1)x}{L_x}
				  ) + 
		   C_2 cos(
		   			\frac{2 \pi (j_y - 1)y}{L_y}
		   		  ) + 
		   C_3 cos(
		   			\frac{2 \pi (j_z - 1)z}{L_z}
		   		  ),
\end{equation}

\begin{equation}\label{eq:real_functions_3}
	F_3 := C_1 cos(
				\frac{2 \pi (j_x - 1) x}{L_x} + 
				\frac{2 \pi (j_y - 1) y}{L_y} + 
				\frac{2 \pi (j_z - 1) z}{L_z}
				) + 
		   C_2 sin(
		   		\frac{2 \pi (j_x - 1)x}{L_x} + 
		   		\frac{2 \pi (j_y - 1)y}{L_y} + 
		   		\frac{2 \pi (j_z - 1)z}{L_z}
		   		),
\end{equation}

其中 $ j_x $,$ j_y $,$ j_z $ 是整数常量,$ x $,$ y $,$ z $ 是皮层球面网格表示上每个点的空间位置,$ L_x $,$ L_y $,$ L_z $ 是每个方向上的周期,$ C_1 $,$ C_2 $,$ C_3 $ 是拟合常数。
对于每个函数,我们构建了一个规则版本和一个不规则版本,从而产生了六个基组。
常规版本对应于当 $ j_x=j_y=j_z=j $ 时,模式 $ j $ 在 $ x $、$ y $ 和 $ z $ 方向上的空间波长规则间隔并增加 $ \frac{2 \pi}{L_x} $ 、$\frac{2 \pi}{L_y}$ 、 $\frac{2 \pi}{L_z}$ 随着模式数的增加,这是傅立叶分析中的标准实现。
不规则版本对应于当$ j_x $、$ j_y $、$ j_z $ 为整数组合且不一定相等时;
因此,$ x $、$ y $ 和 $ z $ 方向上的空间波长是不规则间隔的。
我们实现这个版本是因为它为傅里叶基组提供了更大的自由度,从而提供了表现良好的最佳机会。
为了将 $ (j_x,j_y,j_z) $ 组合关联到单个模式 j,我们按 $ j_x+j_y+j_z $ 递增的顺序排列组合。
例如,前十个模式的 $ (j_x,j_y,j_z) $ 组合遵循集合 $ \{(1,1,1),(1,1,2),(1,2,1),(2,1, 1),(1,2,2),(2,1,2),(2,2,1),(1,1,3),(1,3,1),(3,1,1) \} $。
考虑到几何本征模的空间波长是不规则间隔的(参见补充表 1),这两个版本使我们能够探索空间波长如何影响傅里叶基组的分解。


接下来,我们将 $ i=x, y, z $ 的周期定义为 $ L_i := [ max(i) - min(i) ] $,以实现尊重大脑形状的启发。 
此外,$ (j_i-1) $中的$ -1 $确保第一模式是恒定的,以使其与第一几何模式具有可比性。
方程式\ref{eq:real_functions_2}和\ref{eq:real_functions_3}中的形式具有更多自由度,拟合常数 $ C_1 $、$ C_2 $、$ C_3 $ 分别估计,允许我们对余弦和/或正弦函数进行不同的加权。
因此,模式 $ j $ 现在在模式分解期间需要估计多个幅度,而不是只有一个。
相对于\textit{几何特征模式},这些增加的自由度导致模型复杂性增加;
更具体地说,由方程\ref{eq:real_functions_2}和\ref{eq:real_functions_3}形成的傅立叶基组分别涉及估计每个模式 2 个和 3 个系数,而几何本征模式只需要拟合每个模式一个系数。


尽管可以采用其他形式和复杂的选择来构建傅里叶基组,但上述选择因其简单性而受到很好的启发,产生了独特的实值空间模式,并涵盖了人们可以做出的关键实现选择。
扩展数据图 \ref{fig:extended_fig_5}a 显示了具有单位系数的结果模式的空间分布,我们用它来重建任务激活图和静息状态数据(类似于图 \ref{fig:1}d)。
扩展数据图 \ref{fig:extended_fig_5}b 显示,\textit{几何特征模式}在重建任务诱发和静息状态数据方面显着优于傅里叶基组,进一步强调了几何本征模提供的准确表示并不是任何基组扩展的微不足道的结果。
此外,无论 $ x $、$ y $ 和 $ z $ 方向上的模式的空间波长如何定义,该结论仍然成立。
重申一下,尽管傅里叶基组是基于方程的,
与几何本征模相比,\ref{eq:real_functions_2}和\ref{eq:real_functions_3}有更多的自由度来拟合数据,\textit{几何特征模式}仍然表现出优越的性能,强调了它们在考虑大脑动力学方面的简约性。


我们提出了傅立叶基组的完整性分析,并证明此类函数无法适应不规则形状物体的边界条件,例如皮层表面(例如,内侧壁的狄利克雷或诺依曼条件)\cite{hasson2008hierarchy}。
这是因为傅立叶基组只能针对具有规则形状(例如矩形)的对象正确构建,并且更适合分析在此类形状(例如二维图像)上定义的函数。
因此,对于黎曼流形上定义的皮质表面,使用傅立叶基组通常没有很好的动机,而使用\textit{拉普拉斯-贝尔特拉米算子}的特征模式更为合适\cite{murray2014hierarchy,glasser2011mapping}。
事实上,黎曼流形上的\textit{拉普拉斯-贝尔特拉米算子}特征模式分析在形式上被认为是傅里叶分析\cite{gao2020neuronal}的推广。
这同样适用于离散网络,例如连接组;
出于这个原因,图拉普拉斯的特征分解(如工作中所做的)通常在谱图理论领域用作经典傅立叶变换的替代方案\cite{tian2020topographic}。
因此,虽然可以分解大脑活动的空间图 使用傅立叶基组,它们的效率非常低并且不太适合当前的问题。
因此我们不提倡使用它们。
我们还注意到,傅里叶基组无法深入了解大脑活动背后的生成过程。
我们的主要重点是比较生理原理和解剖学约束的基础集(即几何和连接组本征模式),以揭示大脑动力学的关键限制,而不是确定统计上最佳的基础集。





\subsection{代理映射的模态功率谱} \label{sec:modal_power_spectra}


典型的图像处理管道会引起一定程度的空间平滑,从而滤除高频(短波长)活动的空间模式。
为了确保我们在分析中确定的主要长波长功率不仅仅是功能磁共振成像预处理引起的空间平滑的产物,我们分析了具有不同平滑级别的替代随机数据的功率谱。
我们首先在体积空间中生成 10,000 个取自零均值高斯分布的随机图。
然后,我们以半高全宽范围为 0 到 50 mm 的核尺寸对替代图进行平滑处理。
最后,我们将代理图投影到 fsLR-32k CIFTI 空间(通过 Nilearn)并分析它们的模态功率谱。
请注意,我们在体积而不是表面空间中进行平滑,以模仿流行的神经成像处理实践,其中数据处理通常应用于体积数据,然后将结果重新采样到皮质表面\cite{haak2018connectopic}。
因此,该过程也包含由重采样程序引起的任何平滑。


我们通过计算均方对数误差来比较经验激活图和替代图的模式功率谱:

\begin{equation}\label{eq:MSLE}
	\text{MSLE} = \sqrt{
				\frac{1}{N}
					\sum_{j=1}^{N=200}
					[ log_10 (P_j^{empirical}) - 
					log_10 (P_j^{surrogate}) ] ^2
				},
\end{equation}

其中 $ P_j^{empirical} $ 和 $ P_j^{surrogate} $ 分别是经验映射和代理映射中模式 $ j $ 的幂。
作为距离测量,MSLE 比典型的均方误差更有效,因为长波长到短波长模式的功率具有多个数量级的变化。


扩展数据图 \ref{fig:extended_fig_8}a 中的彩色线显示了替代数据的模态功率谱,扩展数据图 \ref{fig:extended_fig_8}b 显示了跨对比平均的经验谱和替代谱之间的平均距离(即 MSLE),扩展数据图 \ref{fig:extended_fig_8}c 显示了 MSLE 为每个任务对比单独获得(而不是对对比进行平均)。
我们发现小于 10 mm 的平滑核不能充分捕获经验图的长空间波长中的光谱功率集中度(图 \ref{fig:3}a 和扩展数据图 \ref{fig:extended_fig_8}a)。
事实上,替代数据需要 20 毫米的平均内核尺寸(从扩展数据图 \ref{fig:extended_fig_8}b 中的最小值获得的最佳拟合)才能获得与经验数据相当的频谱,该数据比实验中使用的内核尺寸大得多。
神经影像处理(体积空间 49 中约 8 毫米,表面空间 50 约 15 毫米)。
当单独分析任务对比而不是平均值时,得到了类似的结果,最小 MSLE 显示出围绕平均值的微小变化(扩展数据图 \ref{fig:extended_fig_8}c 中的实线)。
因此,这些发现证实,经验激活图的长波长内容不能仅通过预处理来解释。
此外,虽然也可以使用简单的平滑函数并避免统计阈值来恢复任务激活的空间扩展模式,但\textit{几何特征模式}方法可以更深入地理解激活背后的机制,因为这些模式可以直接与严格的大脑生物物理模型联系起来 由\textit{神经场理论}建立的动态(参见示例\cite{robinson1997propagation,robinson2016eigenmodes,jones1999golgi,wang2016brain})。
它们还提供数据的自然多尺度表征。



\subsection{神经场理论的波动力学模型} \label{sec:NFT_wave}

由无再生的各向同性阻尼波动方程描述的简单\textit{神经场理论}波动力学模型为:

\begin{equation}\label{eq:NFT_wave}
	[\frac{1}{\gamma_s^2} \frac{\partial ^2}{\partial t^2} +
	\frac{2}{\gamma_s} \frac{\partial}{\partial t} + 
	1 - r_s^2 \nabla^2
	]
	\phi(\boldsymbol{r}, t)
	= Q(\boldsymbol{r}, t),
\end{equation}


其中 $ \phi(\boldsymbol{r},t) $ 是位置 $ r $ 和时间 $ t $ 处的神经活动,$ Q $ 是外部输入,$ \gamma_s $ 是阻尼率,$ r_s $ 是波传播的空间长度尺度。
该模型假设点之间的活动传播受其白质连接性的控制,其强度随距离呈指数衰减。
当方程\ref{eq:NFT_wave}转换为其等价积分形式如下时,这种距离依赖性更加明显。
考虑新皮质上通过白质束连接的位置 $ \boldsymbol{r}' $ 和 $ \boldsymbol{r} $ 处的两个点(参见补充图 \ref{fig:extended_fig_10} 中的视觉示意图)。
位置 $ \boldsymbol{r} $ 和时间 $ t $ 处的活动 $ \phi(\boldsymbol{r},t) $ 可以被视为位置 $ \boldsymbol{r}' $ 和时间 $ t' $ 处的源 $ Q(\boldsymbol{r}',t') $ 与基于白质的连接核 $ W(\boldsymbol{r}, t, \boldsymbol{r}', t') $ 的时空卷积 ;比如:

\begin{equation}\label{eq:location_activity}
	\phi(\boldsymbol{r}, t) = 
		\sum
			W(\boldsymbol{r}, t; \boldsymbol{r}', t')
			Q(\boldsymbol{r}', t')
			d^2
			\boldsymbol{r}'
			dt'.
\end{equation}


在各向同性情况下,$ W $ 仅取决于点之间的空间间隔和时间差,使得 $ W(\boldsymbol{r}, t; \boldsymbol{r}', t') := W(\boldsymbol{r} - \boldsymbol{r}', t-t') $。
核 $ W $ 也称为格林函数,是点源引起的活动 $ \phi $。
使用格林函数方法\cite{pang2021stochastic},先前的工作已经表明 $ W $ 的适当表达式,使其成为方程\ref{eq:NFT_wave}中阻尼波动方程的解,具有强点源 
$ Q(\boldsymbol{r}, t; \boldsymbol{r}', t') = \delta(\boldsymbol{r} - \boldsymbol{r}') $ 且 
$ |\boldsymbol{r} - \boldsymbol{r}' \leq \gamma_s r_s (t-t')| $:\cite{wright1995simulation,jirsa1996field,robinson1997propagation,nozari2020brain}

\begin{equation}\label{eq:connectivity_kernel}
	W(\boldsymbol{r} - \boldsymbol{r}') = 
		\frac{\gamma_s}{r_s}
		\frac{exp(\frac{-|\boldsymbol{r}-\boldsymbol{r}'|}{r_s})}{\sqrt{\gamma_s^2 r_s^2 (t-t')^2 - |\boldsymbol{r} - \boldsymbol{r}'|^2}}
		\Theta
		[\gamma_s r_s (t-t') - |\boldsymbol{r} - \boldsymbol{r}'|],
\end{equation}

其中 $ |\boldsymbol{r} - \boldsymbol{r}'| $ 是点之间的白质束距离,$ \Theta $ 是 Heaviside 阶跃函数,$ \gama_s $ 和 $ r_s $ 分别是阻尼率和空间长度尺度。
方程\ref{eq:NFT_wave}的详细推导。 
因此,方程\ref{eq:connectivity_kernel}展示了方程\ref{eq:NFT_wave}定义的波浪动力学如何可以与潜在的各向同性解剖连接直接相关,该连接随距离呈指数衰减。
特别是,求解方程中微分形式的阻尼波动方程\ref{eq:NFT_wave}相当于求解积分方程,

\begin{equation}\label{eq:integral_equation}
	\phi(\boldsymbol{r}, t) = 
		\frac{\gama_s}{r_s} \sum
			\frac{exp(\frac{-|\boldsymbol{r}-\boldsymbol{r}'|}{r_s})}{\sqrt{\gamma_s^2 r_s^2 (t-t')^2 - |\boldsymbol{r} - \boldsymbol{r}'|^2}}
			\Theta
			[\gamma_s r_s (t-t') - |\boldsymbol{r} - \boldsymbol{r}'|]
			Q(\boldsymbol{r}', t')
			d^2 \boldsymbol{r}'
			dt',
\end{equation}

后者明确显示了 $ r_s $ 特征范围(~84 mm\cite{robinson1997propagation})的白质连接性引起的时空效应。
方程\ref{eq:NFT_wave}中的波动方程还有其他更复杂的变化。
但是这个简单版本的模型评估了可以解释经验数据的基本物理过程。
将空间异质性或结构化输入纳入方程中的波动方程\ref{eq:NFT_wave}是未来研究的主题。


尽管模型很简单,但方程\ref{eq:NFT_wave}的完整时空解在 32,492 个顶点上的计算成本很高。
因此,我们使用以下方法来有效地求解在皮层上的方程\ref{eq:NFT_wave}。
首先,我们假设 $ \psi $ 是时空可分离的,因为系统的结构在时间上是恒定的,这样

\begin{equation}\label{eq:time_space}
	\phi(\boldsymbol{r}, t) = 
		\psi(t) \phi(\boldsymbol{r}).
\end{equation}

通过将方程\ref{eq:time_space}代入方程\ref{eq:NFT_wave},项$ \nabla^2 \phi(\boldsymbol{r}, t) $ 变为 $ \phi(t) \nabla^2 \phi
(\boldsymbol{r}) $。
请注意,$ \nabla^2 \phi(\boldsymbol{r}) $与方程 \ref{eq:1} 中\textit{亥姆霍兹方程}的左侧相同。
因此,皮层表面上行波解的空间分量 $ \psi(\boldsymbol{r}) $ 可以写成\textit{几何特征模式} $ \psi_j(\boldsymbol{r}) $ 的组合。
因此,方程\ref{eq:time_space}可以写成:

\begin{equation}\label{eq:eigenmodes_combination}
	\phi(\boldsymbol{r}, t) = 
		\sum_{j=1}^{N}
			\phi_j(t) \psi_j(\boldsymbol{r}),
\end{equation}

其中 $ \psi_j(\boldsymbol{r}) $ 是模式 $ j $,$ \phi_j(t) $ 是模式 $ j $ 的时变分量,$ N $ 是模式数量。
其次,我们对输入 $ Q(\boldsymbol{r},t) $ 采用模式分解,使得:

\begin{equation}\label{eq:Q_decomposition}
	Q(\boldsymbol{r}, t) = 
		\sum_{j=1}^{N}
			q_j(t) \psi_j(\boldsymbol{r}),
\end{equation}


其中 $ q_j (t) $ 是通过模式分解获得的时变幅度(参见方法\ref{sec:modal_decomposition}中的“大脑活动的模式分解”)。
上述时间和空间因素分离的重要结果是,空间模式对局部时间动态不敏感,也不对 $ r_s $ 的值敏感,只要它比表面的曲率半径小。


将方程\ref{eq:eigenmodes_combination}、\ref{eq:Q_decomposition}和\ref{eq:1}代入方程\ref{eq:NFT_wave},我们得到:

\begin{equation}\label{eq:NFT_expand}
	\sum_{j=1}^{N}
		[
		\frac{1}{\gamma _s^2 \frac{\partial^2}{\partial t^2}} +
		\frac{2}{\gamma _s} \frac{\partial}{\partial t} +
		1 + r_s^2 \lambda_j
		]
		\phi_j(t)
		\psi_j(\boldsymbol{r})
	=
		\sum_{j=1}^{N}
			q_j(t)
			\psi_j(\boldsymbol{r}),
\end{equation}

其中 $ λ_j $ 是对应于模式 $ j $ 的特征值。
因此,求解模态 $ j $ 的时变分量 $ \phi_j (t) $ 的常微分方程为:

\begin{equation}\label{eq:ordinary_differential}
	[
		\frac{1}{\gamma _s^2}
		\frac{d^2}{dt^2}
		+
		\frac{2}{\gamma _s}
		\frac{d}{dt}
		+ 1 + r_s^2 \lambda_j
	]
	\phi_j(t) 
	=
	q_j(t).
\end{equation}


可以通过数值方法解方程\ref{eq:ordinary_differential},但它可以通过傅里叶变换与时间的关系来更简单、更准确地求解,从而产生代数方程:

\begin{equation}\label{eq:algebraic_equation}
	[
		- \frac{\omega ^2}{\gamma_s^2}
		- \frac{2 i \omega}{\gamma_s}
		+ 1
		+ r_s^2 \lambda_j
	]
	\phi_j(\omega)
	=
	q_j(\omega),
\end{equation}

其中 $ ω $ 是时间角频率, $ \phi_j (\omega) $ 是 $ \phi_j (t) $ 的傅里叶变换,$ q_j (\omega) $ 是 $ q_j (t) $ 的傅里叶变换。 因此,解 $ \phi_j (t) $ 具有以下形式:

\begin{equation}\label{eq:solution_form}
	\phi_j(t)
	=
	F^{-1} \{\phi_j(\omega)\}
	=
	F^{-1}
	\{
		\frac{
				\gamma_s^2 q_j(\omega)
			 }
		 	 {
		 	 	- \omega^2
		 	 	- 2 i \omega \gamma_s
		 	 	+ \gamma_s^2
		 	 		(1 + r_s^2 \lambda_j)
		 	 }
	\},
\end{equation}

其中 $ F^{-1)} $ 是傅里叶逆变换运算符。
最后,方程\ref{eq:solution_form}可代入式\ref{eq:1}。
\ref{eq:eigenmodes_combination}获取每个顶点的神经活动 $ \phi(\boldsymbol{r},t) $。



如上所述,方程\ref{eq:NFT_wave}的空间部分满足方程 \ref{eq:1}中的\textit{亥姆霍兹方程}。
这个微分方程相当于积分形式(方程\ref{eq:integral_equation}),其中给定点的活动会引起其他地方的活动,其权重代表\textit{神经场理论}中隐含的连接性。
在皮层情况下,这对应于随距离呈指数下降——即类似\textit{指数距离规则}的空间依赖性——与实验证据 6 一致。
因此,\textit{几何特征模式}的使用隐含地结合了类似\textit{指数距离规则}的连接,其中还包括长范围连接,但不直接考虑不符合简单指数规则的拓扑复杂连接。



\subsection{兴奋抑制平衡神经群模型的优化} \label{sec:mass_optimization}

\textit{兴奋抑制平衡}神经群模型具有15个固定参数和4个自由参数,即$ w_{EE} $、$ w_{EI} $、$ w_{IE} $和 $ G $,它们被优化以拟合数据,如下\cite{rosen2022estimation}。
简而言之,实施反馈抑制控制\cite{van2012high}以调整$ w_{IE} $以设置所有区域的发射率 $ r_i^{(E)} $ 约为 3 Hz。
因此,在稳态条件 $ <S^{(E)} \approx 0.17 \text{nA} $ 和 $ <I^{(E)}> \approx 0.38 \text{nA} $ 时,$ w_{IE} $ 的解析表达式为:

\begin{equation}\label{eq:analytic_expression}
	W_{IE} = 
		\frac{
			W_E I_0 + W_{EE} <S^{(E)}> + GJ <S^{(E)}> - <I^{(E)}>
			}
			{
				<S^{(I)}>
			},
\end{equation}

其中稳态值 $ <S^{(I)}> = H^{(I) (<I^{(I)}>) \tau_I } $ 使用以下表达式进行数值求解:

\begin{equation}\label{eq:solved_expression}
	W_I I_0 
	+ W_{EI}<S^{(E)}> 
	- H^{(I)} (<I^{I}>) \tau_l
	- <I^{I}>
	= 0.
\end{equation}

因此,$ w_{IE} $ 的值作为其他自由参数的函数动态变化:$ w_{EE} $、$ w_{EI} $ 和 $ G $。
由于在多维参数空间上找到最优参数集的计算成本很高,因此解析功能耦合(参见方法\ref{sec:modelling_resting}中的“静态动力学建模”))被使用。
模型拟合是使用近似贝叶斯计算和分层人口蒙特卡罗 \cite{arnatkeviciute2021genetic,pang2022evolutionary} 方法完成的,可最大限度地减少经验功能耦合和模型功能耦合之间的距离。
有关模型优化过程的更多详细信息,请参阅\cite{rosen2022estimation}。


\subsection{滞后线程算法} \label{sec:lag_threads}

滞后线程算法首先计算大脑区域之间时间序列的滞后互协方差函数。
假设 BOLD-fMRI 时间序列是非周期性的\cite{aquino2020identifying},则获得互协方差函数呈现极值(通常在 0 到 2 s 之间)的区域之间的时间滞后(或延迟)\cite{gajwani2022can}。
这导致了反对称时间- 延迟矩阵TD,元素$ \tau_{ij} $对应于区域$ i $和$ j $之间的时间延迟,第$ i $列表示系统相对于区域i的滞后图,且$ \tau_{ij}=- \tau_{ji} $。
因此,$ \tau_{ij}>0 $ 意味着区域 $ j $ 落后于区域 $ i $。
底层滞后结构通过两种方式量化:
(i)取每个区域的平均时间滞后(TD 每列的平均值); 
(ii) 对 $ TD_z $ 应用主成分分析,$ TD_z $ 是每列为零均值的 $ TD $。
第一种方法获得大脑区域的平均时间顺序,假设单个滞后过程控制大脑,这已在过去的几项研究中使用过\cite{gajwani2022can,hamid2021wave,yousefi2021propagating}。
然而,对于像大脑这样具有多个滞后过程的系统,平均时间滞后无法捕获所有基本滞后模式,这可以通过主成分分析\cite{coalson2018impact}恢复。
在这里,我们使用前两个主要主成分,解释了经验方差的 74\% 数据。
因此,我们计算了三个滞后预测:
(i)平均滞后; 
(ii)第一个主成分(PC1 滞后); (iii) 第二个主成分(PC2 滞后)。


$ TD $ 矩阵是根据使用 HCP-MMP1 分区进行分区的经验和模拟静息态 BOLD-fMRI 时间过程计算的(参见方法\ref{sec:cortical_parcellations}中的“皮质分区”)。
对于经验数据,计算 255 名\textit{人类连接组项目}个体中每人的 $ TD $ 矩阵,然后计算个体之间的平均 $ TD $ 矩阵。
对于模拟数据,我们使用各自的原始优化参数生成了波和神经群模型的 255 次试验(以匹配经验数据的 255 个\textit{人类连接组项目}个体)的时间序列(参见“方法\ref{sec:modelling_resting}中的静息态动力学建模”)。
为每个试验计算 $ TD $ 矩阵。
然后,计算各个试验的平均 $ TD $ 矩阵。
平均 $ TD $ 矩阵如扩展数据图 \ref{fig:extended_fig_9}a 所示。
最后,从经验和模拟平均 $ TD $ 矩阵中获得了三个滞后预测(扩展数据图 \ref{fig:extended_fig_9}b-d)。


扩展数据图 \ref{fig:extended_fig_9}b 中的结果表明,波浪模型的平均滞后模式与经验模式显着相关,而神经群模型的平均滞后模式则不然。
对于 PC1 滞后,波浪模型的性能略有下降,但仍然优于神经群模型(扩展数据图 \ref{fig:extended_fig_9}c)。
PC2 滞后的模型和经验模式之间的相关性(扩展数据图 \ref{fig:extended_fig_9}d)并不显着,但波浪模型的相关性仍然较高。
这些结果表明,无论采用哪种计算滞后投影的方法,波模型都比神经群模型更好地捕捉经验 fMRI 数据的时滞特性。
这进一步强调了波浪动力学可以提供宏观尺度、静息态动力学的准确物理机制解释,与之前的研究一致\cite{robinson2021determination,majeed2011spatiotemporal,matsui2016transient}。
我们还强调,这些模型在捕获滞后结构方面的性能可能是保守估计,因为这些模型依赖于简单且空间均匀的血流动力学正向模型,该模型没有考虑神经血管耦合的区域变化\cite{deco2021dynamical,henderson2022empirical,fischl2012freesurfer}。
这种变化可能会强烈影响凭经验观察到的滞后。


\subsection{补充讨论} \label{sec:supplementary_discussion}

连接组特征模式的相对较差的性能表明,拓扑复杂的长程连接在准确解释功能磁共振成像测量的皮层活动方面提供的益处微乎其微。
然而,大量证据表明,这种连接可能提供重要的功能和进化优势\cite{oldham2020efficacy,arslan2018human,tokariev2019large}。
正文的图 \ref{fig:2}d 暗示了这些优点,其中低阶连接组本征模式在低频重建任务激活图方面表现稍好。
低阶连接组特征模式比低阶\textit{几何特征模式}包含更复杂的空间模式,这可以在捕获任务激活的空间复杂模式方面提供更大的灵活性。
然而,连接组特征模式的这种优势仅在前 25-30 个模中持续存在,其中重建精度通常较低(即 $ r < 0.50 $),并且与\textit{指数距离规则}特征模式没有太大区别。


长程连接可能发挥作用的另一个潜在方面是皮层功能梯度的形成。
我们的工作表明,皮层下结构的几何本征模式与每个结构的功能耦合导出的函数梯度近乎完美匹配。
然而,我们没有观察到单个几何本征模式和先前描述的 功能耦合衍生的新皮层功能梯度之间存在相同的一对一空间对应关系,其中最主要的功能捕获了功能\cite{chen2022individuality} 的分层感觉-离去轴。
功能梯度 因此,新皮层可能反映了远程连接(例如,皮层下-皮层连接)或几何模式\cite{bolt2022parsimonious}的叠加的复杂贡献,就像音乐和弦从各个音符的组合中出现一样。



\subsection{补充图}

\begin{figure}[!htb] 
	\centering
	\includegraphics[width=0.7\textwidth]{fig/supp_1.pdf}
	\caption{\textbf{使用几何特征模式获得的 47 个人类连接组项目任务对比图的重建精度}。
		线条根据 7 种广泛的\textit{人类连接组项目}任务类型定义的组进行着色(第 S2.1 节和补充表 2)。 wm = 工作记忆。} \label{fig:supp_1}
\end{figure}



\begin{figure}[!htb] 
	\centering
	\includegraphics[width=0.85\textwidth]{fig/supp_2.pdf}
	\caption{\textbf{不同分割分辨率下 7 个关键人类连接组项目任务对比图和静息态功能耦合的重建精度}。 
		有关对比图的详细信息,请参阅第 S2.1 节和补充表 2。
		wm = 工作记忆。
		从深色到浅色的线条代表递减的分割分辨率(箭头方向);
		即 Schaefer100、Schaefer200、Glasser360、Schaefer400、Schaefer600、Schaefer800 和 Schaefer1000 在两个半球分别有 100、200、360、400、600、800 和 1000 个地块。} \label{fig:supp_2}
\end{figure}



\begin{figure}[!htb] 
	\centering
	\includegraphics[width=0.85\textwidth]{fig/supp_3.pdf}
	\caption{\textbf{使用模板导出和个体特定的几何特征模式重建 7 个关键人类连接组项目任务对比图和静息态功能耦合的准确性}。
		有关对比图的详细信息,请参阅第 S2.1 节和补充表 2。 
		wm = 工作记忆。
		实线代表由模板表面导出的特征模式所获得的结果(图\ref{fig:supp_1}d)。
		虚线表示通过从各个表面导出的各个特定特征模式所实现的结果。
		插图显示两个结果之间的差异(即模板减去个体特异性)。} \label{fig:supp_3}
\end{figure}




\begin{figure}[!htb] 
	\centering
	\includegraphics[width=0.85\textwidth]{fig/supp_4.pdf}
	\caption{\textbf{使用 200 个模板导出的和个体特定的几何特征模式,对 7 个关键人类连接组项目任务对比图和静息态耦合进行基于个体的重建精度}。
		有关对比图的详细信息,请参阅第 S2.1 节和补充表 2。
		wm = 工作记忆。
		模板特征模式自模板表面,而个体特定特征模式源自个体表面。
		每个点对应一个个体。
		虚线代表模板准确度 = 个体特定准确度线。
		虚线上方的点表示模板准确度 > 个体特定准确度。
		插图显示了模板导出的特征模式和个体特定的特征模式(即模板减去个体特定的)所实现的重建精度之间差异的直方图。} \label{fig:supp_4}
\end{figure}


\begin{figure}[!htb] 
	\centering
	\includegraphics[width=0.85\textwidth]{fig/supp_5.pdf}
	\caption{\textbf{使用 200 至 500 个模板导出的和个体特定的几何特征模式重建 7 个关键人类连接组项目任务对比图和静息态功能耦合的准确性}。
		有关对比图的详细信息,请参阅第 S2.1 节和补充表 2。
		wm = 工作记忆。
		实线代表由模板表面导出的本征模所获得的结果(图\ref{fig:supp_1}d)。
		虚线表示通过从各个表面导出的各个特定本征模式所实现的结果。
		插图显示了模板导出的特征模式和个体特定的本征模式(即模板减去个体特定的)所实现的重建精度之间的差异。} \label{fig:supp_5}
\end{figure}



\begin{figure}[!htb] 
	\centering
	\includegraphics[width=0.85\textwidth]{fig/supp_6.pdf}
	\caption{
		\textbf{通过几何特征模式和不同连接组密度的连接组特征模式实现的 7 个关键人类连接组项目任务对比图和静息态功能耦合的重建精度}。
		有关对比图的详细信息,请参阅第 S2.1 节和补充表 2。
		wm = 工作记忆。
		虚线表示\textit{几何特征模式}取得的结果(图\ref{fig:supp_1}d)。
		浅色到深色的线条代表使用密度不断增加的连接矩阵(从 1.0\% 到 13.0\%)的连接组本征模式所实现的结果。
	} \label{fig:supp_6}
\end{figure}


\begin{figure}[!htb] 
	\centering
	\includegraphics[width=0.85\textwidth]{fig/supp_7.pdf}
	\caption{
		\textbf{使用不同连接组密度的 200 个连接组本征模式重建 7 个关键人类连接组项目任务对比图和静息态功能耦合的准确性}。
		有关对比图的详细信息,请参阅第 S2.1 节和补充表 2。
		wm = 工作记忆。
		实线对应于200个\textit{几何特征模式}所达到的重建精度。
		虚线分别对应于连接组密度 0.1\% 和 1.55\%,用于生成扩展数据图\ref{fig:supp_1}中的连接组和密度匹配的连接组特征模式。
	} \label{fig:supp_7}
\end{figure}


\begin{figure}[!htb] 
	\centering
	\includegraphics[width=0.85\textwidth]{fig/supp_8.pdf}
	\caption{
	\textbf{通过不同分割分辨率的几何特征模式和离散连接组特征模式实现 7 个关键人类连接组项目任务对比图和静息态功能耦合的重建精度}。
	有关对比图的详细信息,请参阅第 S2.1 节和补充表 2。 wm = 工作记忆。
	对于所有情况,我们最多使用 200 种模式来直接将结果与研究的其余部分进行比较。
	虚线代表\textit{几何特征模式}取得的结果(图\ref{fig:supp_1}d);
	请参阅插图中的放大版本。
	浅色到深色的线条表示通过使用以递增分辨率分割的连接矩阵的离散连接组本征模式所实现的结果。
	绘制重建精度与相对于每个完整基组尺寸(即可用模式总数)使用的模式百分比的关系,该百分比对应于皮层表面的顶点数量(对于几何本征模式)或 每个半球的地块(对于离散连接组本征模式)。
	因此,对于\textit{几何特征模式},我们使用每个半球 32,492 个可用模式中的前 200 个模式,这就是虚线在可用模式的 0.6\% 处终止的原因。
	对于使用 Schaefer400、Schaefer600、Schaefer800 和 Schaefer1000 分区的离散连接组特征模式,我们分别使用每个半球 200、300、400 和 500 个可用模式中的前 200 个模式。
	因此,实线分别在可用模式的 100\%、66.7\%、0.50\% 和 0.40\% 处终止。
	结果强调,几何本征模提供了大脑活动最简约和紧凑的表示。
	} \label{fig:supp_8}
\end{figure}



\begin{figure}[!htb] 
	\centering
	\includegraphics[width=0.85\textwidth]{fig/supp_9.pdf}
	\caption{
		\textbf{通过 fMRI 数据的主成分分析 (principal component analysis, PCA) 获得主成分}。
		PCA 接受了 7 个关键\textit{人类连接组项目}任务对比图和 200 个人的静息状态时间序列的训练。
		有关对比图的详细信息,请参阅第 2.1 节和补充表 2。
		wm = 工作记忆。
		主成分 1-5、10、25、50、100 和 200 从上到下显示。
		负-零-正值被着色为蓝-白-红。
	} \label{fig:supp_9}
\end{figure}


\begin{figure}[!htb] 
	\centering
	\includegraphics[width=0.85\textwidth]{fig/supp_10.pdf}
	\caption{\textbf{神经场论中信号传播的示意图}。
		皮层上位置 $ r' $ 和 $ r $ 处的两个点通过白质束(红色曲线)连接。 对于各向同性介质,活动 phi(r,t) 是源 $ Q(r',t') $ 和白质连接施加的连接核 $ W(r,t;r',t') $ 的卷积 ,仅取决于空间间隔 $ r-r' $ 和时间间隔$  t-t' $。
	} \label{fig:supp_10}
\end{figure}


\begin{figure}[!htb] 
	\centering
	\includegraphics[width=0.85\textwidth]{fig/supp_11.pdf}
	\caption{
		\textbf{每个区域的峰值激活时间和 T1w:T2w 的比较}。
		一个半球所有 180 个大脑区域的排名活动曲线时间达到峰值与排名 T1w:T2w 值的关系。
		红线表示排名变量与 Spearman 相关系数 $ r $ 和单边旋转测试 $ p $ 值 pspin 的线性拟合,根据 10,000 个排列进行估计。
	} \label{fig:supp_11}
\end{figure}



\subsection{补充表}

\begin{table}[htbp]
	\centering
	\small
	\caption{特征模式得空间波长}
	\begin{tabular}{ccc}
		\toprule
		特征群         &        波长(纳米)  & 特征群中包含的特征模式     \\
		\midrule
		0      &   -      &      1  \\
		1      &   297.7      &      2-4  \\
		2      &   171.9      &      5-9  \\
		3      &   121.5      &      10-16  \\
		4      &   94.1      &      17-25  \\
		5      &   76.9      &      26-36  \\
		6      &   65.0      &      37-49  \\
		7      &   56.3      &      50-64  \\
		8      &   49.6      &      65-81  \\
		9      &   44.4      &      82-100  \\
		10      &   40.1      &      101-121  \\
		11      &   35.5      &      122-144  \\
		12      &   33.7      &      145-169  \\
		13      &   31.2      &      170-196  \\
		14      &   29.1-      &      197-225  \\

		\bottomrule
	\end{tabular}%
	\label{tab:spatial_wavelength}%
\end{table}%



\begin{table}[htbp]
	\centering
	\small
	\caption{HCP任务对比}
	\begin{tabular}{cccc}
		\toprule
		任务类型         &       对比数  & 对比  & 关键对比     \\
		\midrule
		社交      &   3      &  random; tom; tom\_random  & tom\_random  \\
		运动      &   13      & \makecell{ cue; lf; lh; rf; rh; t; avg; lf\_avg; lh\_avg; \\rf\_avg; rh\_avg; t\_avg; cue\_avg  & cue\_avg}  \\
		赌博       &   3      &  punish; reward; punish\_reward  & punish\_reward  \\
		工作记忆      &   19      & \makecell{ 2bk\_body; 2bk\_face; 2bk\_place; \\ 2bk\_tool; 0bk\_body; 0bk\_face; \\ 0bk\_place; 0bk\_tool; 2bk; 0bk; body; \\ face; place; tool; body\_avg; face\_avg; \\ place\_avg; tool\_avg; 2bk\_0bk  & 2bk\_0bk }  \\
		语言      &   3      &  math; story; math\_story  & math\_story  \\
		情感      &   3      &  faces; shapes; faces\_shapes  & faces\_shapes  \\
		关系      &   3      &  match; rel; match\_rel  & match\_rel  \\
		
		\bottomrule
	\end{tabular}%
	\label{tab:task_contrasts}%
\end{table}%



\begin{table}[htbp]
	\centering
	\small
	\caption{兴奋抑制平衡神经群模型的固定和自由参数}
	\begin{tabular}{ccc}
		\toprule
		符号         &        值  & 符号     \\
		\midrule
		$\tau_E$      &   0.1 s      &      $ W_{EE} $  \\
		$\tau_I$      &   0.01 s      &      $ W_{IE} $  \\
		$\gamma$      &   0.641 s      &      $ W_{EI} $  \\
		$\sigma$      &   0.01 nA      &      $ G $  \\
		$a_E$      &   310 $ nC^{-1} $      &       \\
		$b_E$      &   125 $ s^{-1} $      &       \\
		$d_E$      &   0.16 s      &       \\
		$a_I$      &   615 $ nC^{-1} $      &       \\
		$b_I$      &   177 $ s^{-1} $      &       \\
		$d_I$      &   0.087 s      &       \\
		$I^{ext}$      &   0 nA       &       \\
		$I_0$      &   0.382 nA      &       \\
		$W_E$      &   1.0       &       \\
		$W_I$      &   0.7       &       \\
		$J$      &   0.15 nA      &       \\
		
		\bottomrule
	\end{tabular}%
	\label{tab:BEI_parameters}%
\end{table}%




\nocite{*}
\printbibliography[heading=bibintoc, title=\ebibname]

\appendix
%\appendixpage
\addappheadtotoc

\end{document}
